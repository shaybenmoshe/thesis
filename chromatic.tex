\section{Overview of Chromatic Homotopy Theory}

Our goal is to give a quick overview of chromatic homotopy theory.
There are various point of views and approaches to the topic, and we shall highlight some of them.
One of our main goals is to introduce Morava K-theory $\K[n]$ and Morava E-theory of different flavors $\E{n}$ and $\E{k,\Gamma}$, and their connection to formal group laws.
Our motivation will be the Balmer spectrum of the sphere spectrum.
We will follow the construction of Morava K-theory and Morava E-Theory, and other related spectra, from the point of view of formal group laws, and use them to describe the Balmer spectrum of the sphere spectrum.



\subsection{The Balmer Spectrum}

We will start with an algebraic motivation.
Let $R$ be a noetherian ring.
Consider the symmetric monoidal stable $\infty$-category $\Ch{R}$ of chain complexes on $R$.
It is then natural to ask how much information about $R$ is encoded in the category $\Ch{R}$.
We will try to recover $\spec R$, as a topological space, from $\Ch{R}$.

\begin{remark}
	Balmer's work \cite[6.3]{BalSpec} actually recovers the structure sheaf as well, but we will not consider the structure sheaf.
\end{remark}

\begin{definition}
	A \emph{perfect complex} is a complex that is quasi-isomorphic to a bounded complex of finitely-generated projective modules.
	These objects are the compact objects in $\Ch{R}$, thus they can be defined categorically.
	Their full subcategory is denoted by $\Chperf{R}$.
\end{definition}

\begin{definition}
	Let $\mcl{C}$ be some symmetric monoidal stable $\infty$-category.
	A full subcategory $\mcl{T}$ is \emph{thick} if:
	\begin{itemize}
		\item $0 \in \mcl{T}$,
		\item it is closed under cofibers,
		\item it is closed under retracts.
	\end{itemize}
\end{definition}

\begin{example}\label{chperf-tk}
	Consider the case $\mcl{C} = \Chperf{R}$ (e.g. over $\ZZ$, chain complexes quasi-isomorphic to bounded chain complexes of finitely-generated free abelian groups).
	Let $K \in \Ch{R}$, and define
	$\mcl{T}_K = \left\{ A \in \Chperf{R} \mid A \otimes K \cong 0 \right\}$.
	We claim that $\mcl{T}_K$ is thick.
	Clearly $0 \in \mcl{T}_K$.
	Let $A \to B$ be a morphism between two complexes in $\mcl{T}$. Since tensor is left, tensoring the cofiber with $K$ is given by
	$\cofib\left(A \to B\right) \otimes K \cong \cofib\left(A \otimes K \to B \otimes K\right) \cong \cofib\left(0 \to 0\right) \cong 0$, therefore the cofiber is indeed in $\mcl{T}_K$.
	Lastly, if $A \to B \to A$ is the identity and $B \otimes K \cong 0$, we get that $\id[A \otimes K]$ factors through $0$, which implies that $A \otimes K$ is $0$, so that $A \in \mcl{T}_K$.
\end{example}

\begin{definition}
	A thick subcategory $\mcl{T}$ is an \emph{ideal} if $A \in \mcl{T}, B \in \mcl{C} \implies A \otimes B \in \mcl{T}$.
	Furthermore, it is a \emph{prime ideal} if it is a proper subcategory, and $A \otimes B \in \mcl{T} \implies A \in \mcl{T} \textrm{ or } B \in \mcl{T}$.
	The \emph{spectrum} of the category is defined similarly to the classical spectrum of a ring:
	As a set, $\spec \mcl{C} = \left\{ \mcl{P}\textrm{ prime ideal} \right\}$.
	For any family of objects $S \subseteq \mcl{C}$ we define $V\left(S\right) = \left\{ \mcl{P} \in \spec \mcl{C} \mid S \cap \mcl{P} = \emptyset \right\}$.
	We topologize $\spec \mcl{C}$ with the Zariski topology by declaring those to be the closed subsets.
	We also denote $\supp\left(A\right) = V\left(\left\{ A \right\}\right)$.
\end{definition}

\begin{example}
	We continue the example of $\mcl{T}_K$.
	Clearly if $A \otimes K \cong 0$ then also $A \otimes B \otimes K \cong 0$, so it is an ideal.
	
	Let $\mfk{p}$ be a prime ideal in $R$ in the usual sense, and take $K = R_\mfk{p}$ (concentrated at degree $0$), and we wish to show that $\mcl{T}_{R_\mfk{p}}$ is a prime ideal.
	Let $A,B \in \Chperf{R}$ be complexes such that $A \otimes B \in \mcl{T}_{R_\mfk{p}}$, that is $A \otimes B \otimes R_\mfk{p}$ is quasi-isomorphic to $0$.
	We need to show that one of $A \otimes R_\mfk{p}$ and $B \otimes R_\mfk{p}$ is quasi-isomorphic to $0$.
	Note that $A \otimes B \otimes R_\mfk{p} = \left(A \otimes R_\mfk{p}\right) \otimes_{R_\mfk{p}} \left(B \otimes R_\mfk{p}\right)$.
	Denote $S = R_\mfk{p}$, which is a local ring with maximal ideal $\mfk{m} = \mfk{p} R_\mfk{p}$ and residue field $k = S/\mfk{m}$, we are then reduced to showing that for perfect complexes $X = A \otimes R_\mfk{p}, Y = B \otimes R_\mfk{p}$, such that $H_*\left(X \otimes_S Y\right) = 0$, either $H_*\left(X\right) = 0$ or $H_*\left(Y\right) = 0$.
	
	Note that if we base-change to $k$, this statement is trivial since by K\"unneth, $0 = H_*\left(X/\mfk{m} \otimes_k Y/\mfk{m}\right) = H_*\left(X/\mfk{m}\right) \otimes_k H_*\left(Y/\mfk{m}\right)$, and the tensor product of vector spaces is $0$ if and only if one of them is $0$.
	Therefore, it suffices to prove that for a perfect complex $Z$ over $S$, $H_*\left(Z\right) = 0$ if and only if $H_*\left(Z/\mfk{m}\right) = 0$.
	
	For the first direction, it is clear that if $Z$ is quasi-isomorphic to $0$, then the (derived) tensor product $Z \otimes_S k = Z/\mfk{m}$ is also quasi-isomorphic to $0$.
	
	For the other direction, assume that $Z/\mfk{m}$ is quasi-isomorphic to $0$.
	Since $Z$ is perfect, we can choose it to be bounded
	$$
	Z = \cdots \to 0 \to Z_1 \xrightarrow{d_1} \cdots \xrightarrow{d_{n-2}} Z_{n-1} \xrightarrow{d_{n-1}} Z_n \to 0 \to \cdots,
	$$
	where each $Z_i$ is finitely-generate projective, and over the local ring $S$, this also implies that each $Z_i$ is in fact free.
	We will show that $Z$ is quasi-isomorphic to a complex with only $n-1$ non-trivial terms, and by induction we conclude that $Z$ is indeed quasi-isomorphic to $0$.
	We are given that $H_*\left(Z/\mfk{m}\right) = 0$, so in particular it is exact at $Z_{n-1}/\mfk{m} \xrightarrow{d_{n-1}} Z_n/\mfk{m} \to 0$, i.e. $\im d_{n-1} = Z_n \mod \mfk{m}$.
	By Nakayama's lemma we conclude that $\im d_{n-1} = Z_n$, i.e. $Z_{n-1} \xrightarrow{d_{n-1}} Z_n$ is surjective.
	By the freeness, we conclude that $Z_{n-1} \cong M \oplus Z_n$, under which $d_{n-1} = 0 \oplus \id_{Z_n}$.
	Since $d_{n-1} d_{n-2} = 0$, the image of $d_{n-1}$ must land in $M$.
	Therefore, $Z$ is quasi-isomorphic to
	$$
	\cdots \to 0 \to Z_1 \xrightarrow{d_1} \cdots \xrightarrow{d_{n-2}} M \to 0 \to 0 \to \cdots,
	$$
	which has only $n-1$ non-trivial terms, and we are done.
\end{example}

\begin{theorem}[{\cite[5.6]{BalSpec}}]
	The map $\spec R \to \spec\left(\Chperf{R}\right)$,
	given by $\mfk{p} \mapsto \mcl{T}_\mfk{p} = \left\{ A \mid A_\mfk{p} = 0 \right\}$
	is a homeomorphism.
\end{theorem}

\begin{proposition}[Prime ideals pullback]\label{primes-pullback}
	Let $F: \mcl{C} \to \mcl{D}$ be an exact symmetric monoidal functor, between two symmetric monoidal stable $\infty$-categories.
	Let $\mcl{P} \in \spec \mcl{D}$ be a prime ideal, then $F^*\mcl{P} = F^{-1}\left(\mcl{P}\right) = \left\{ A \in \mcl{C} \mid F\left(A\right) \in \mcl{P} \right\}$ is a prime ideal.
	Moreover, the function we obtain, $F^*: \spec \mcl{D} \to \spec \mcl{C}$, is continuous.
\end{proposition}

\begin{proof}
	We first prove that for $\mcl{P} \in \spec \mcl{D}$, $F^*\mcl{P} \in \spec \mcl{C}$.
	
	Clearly $F\left(0\right) = 0 \in \mcl{P}$ since $F$ is exact, so $0 \in F^*\mcl{P}$.
	Since $F$ preserves cofibers, for $A,B \in F^*\mcl{P}$, i.e. $F\left(A\right), F\left(B\right) \in \mcl{P}$, and a map $A \to B$ we get
	$
	F\left(\cofib\left(A \to B\right)\right)
	= \cofib\left(F\left(A\right) \to F\left(B\right)\right)
	\in \mcl{P}
	$.
	Let $A \to B \to A$ be a retract, that is the composition is the identity, s.t. $B \in F^*\mcl{P}$. We know that $F\left(A\right) \to F\left(B\right) \to F\left(A\right)$ is also a retract by functoriality, thus $F\left(A\right) \in \mcl{P}$, that is $A \in F^*\mcl{P}$.
	We conclude that $F^*\mcl{P}$ is indeed a thick subcategory.
	
	Let $A \in F^*\mcl{P}, B \in \mcl{C}$, since $F$ is monoidal, $F\left(A \otimes B\right) = F\left(A\right) \otimes F\left(B\right) \in \mcl{P}$, so $A \otimes B \in F^*\mcl{P}$, that is $F^*\mcl{P}$ is an ideal.
	
	We claim that $F^*\mcl{P}$ is a proper subcategory, because an ideal is proper if and only if it doesn't contain $1$, and since $F$ is symmetric monoidal it sends $1$ to $1$.
	
	Lastly, assume that $A \otimes B \in F^*\mcl{P}$, again since $F$ is monoidal, $F\left(A \otimes B\right) = F\left(A\right) \otimes F\left(B\right) \in \mcl{P}$, so $A \in F^*\mcl{P}$ or $B \in F^*\mcl{P}$, that is $F^*\mcl{P}$ is a prime idea.
	
	Now we show that $F^*: \spec \mcl{D} \to \spec \mcl{C}$ is continuous.
	So let $V\left(S\right) \subseteq \spec \mcl{C}$ be a closed subset.
	We have:
	\begin{align*}
		\left(F^*\right)^{-1}\left(V\left(S\right)\right)
		&= \left\{ \mcl{P} \in \mcl{D} \mid F^*\mcl{P} \in V\left(S\right) \right\}\\
		&= \left\{ \mcl{P} \in \mcl{D} \mid F^{-1}\left(\mcl{P}\right) \cap S = \emptyset \right\}\\
		&= \left\{ \mcl{P} \in \mcl{D} \mid \mcl{P} \cap F\left(S\right) = \emptyset \right\}\\
		&= V\left(F\left(S\right)\right)
	\end{align*}
	So $\left(F^*\right)^{-1}\left(V\left(S\right)\right)$ is indeed also closed, which shows that $F^*$ is continuous.
\end{proof}

Now, recall that $\Ch{R} \cong \Mod{\HH{R}}$, therefore we can reinterpret the above theorem as $\spec R \cong \spec\left( \Mod{\HH{R}}^\mrm{comp} \right)$ (where the $\mrm{comp}$ denotes the compact objects in the category).
We shall turn this theorem into a definition:

\begin{definition}
	Let $R$ be an $\mbb{E}_\infty$-ring.
	We define the \emph{spectrum} of $R$ to be
	$\spec R = \spec\left( \Mod{R}^\mrm{comp} \right)$.
\end{definition}

A natural question to ask then is what is the topological space $\spec \SS$.
Recall that $\Mod{\SS} = \Sp$, the category of spectra, and that the compact objects in spectra are the finite spectra $\Spfin$.
So, unwinding the definitions, the question can rephrased as finding the prime ideals in $\Spfin$, and their topology.
Chromatic homotopy theory provides an answer to this question.



\subsection{\texorpdfstring{$\MU$}{MU} and Complex Orientations}

Throughout this section, let $E$ be a multiplicative cohomology theory (that is, equipped with a map $E \otimes E \to E$ and $1 \in E_0$, which is associative and unital after taking homotopy groups).

Consider the map $S^2 \to \BU[1]$ classifying the universal complex line bundle.
Concretely, under the identifications $S^2 \cong \CP{1}$ and $\BU[1] \cong \CP{\infty}$, this map can be realized as the inclusion $\CP{1} \subseteq \CP{\infty}$.
This map induces a map
$$
	\tilde{E}^2\left(\BU[1]\right)
	\to \tilde{E}^2\left(S^2\right)
	\cong \tilde{E}^0\left(S^0\right)
	\cong E^0\left(*\right)
	= E_0.
$$
Since $E$ is unital, there is a canonical generator $1 \in E_0$.

\begin{definition}
	$E$ is called \emph{complex oriented} if the map $\tilde{E}^2\left(\BU[1]\right) \to E_0$ is surjective, equivalently, if $1$ is in the image of that map.
	A choice of a lift $x \in \tilde{E}^2\left(\BU[1]\right)$ of $1 \in E_0$ is called a \emph{complex orientation}.
	(Note that $\left|x\right| = -2$ as it is in cohomological degree $2$.)
\end{definition}

\begin{example}\label{HR-1}
	Let $R$ be some ring, and consider $\HH{R}$.
	It is known that
	$\HH{R}^*\left(\CP{n}\right) \cong R \left[x\right] / \left(x^{n+1}\right)$
	and
	$\HH{R}^*\left(\CP{\infty}\right) \cong R \formal{x}$,
	where $\left|x\right| = -2$,
	and the maps induced by the inclusions of projective spaces maps $x$ to $x$.
	Therefore we see that $x \in \HH{R}^2\left(\BU[1]\right)$ is mapped to $x \in \HH{R}^2\left(S^2\right) = R\left\{x\right\}$, which is mapped to $1 \in \HH{R}_0 = R$.
	Hence, $x$ is a complex orientation.
\end{example}

\begin{example}[$\K$-Theory Saga: Complex Orientation]\label{k-thy-oriented}
	Let $\K$ be complex K-theory, then we know that $\K_* = \ZZ\left[\beta^{\pm 1}\right]$ where $\beta$ is the Bott element, with $\left|\beta\right| = 2$.
	It is also known (by the Atiyah-Hirzebruch spectral sequence) that
	$\K^*\left(\CP{n}\right) \cong \K_* \left[t\right] / \left(t^{n+1}\right)$
	and
	$\K^*\left(\CP{\infty}\right) \cong \K_* \formal{t}$
	(here $\left|t\right| = 0$),
	where the maps induced by the inclusions of projective spaces maps $t$ to $t$.
	We deduce that $\beta^{-1} t \in \K^2\left(\BU[1]\right)$ is mapped to $\beta^{-1} t \in \K^2\left(S^2\right) = \ZZ\left\{\beta^{-1} t\right\}$, which is indeed the generator.
	Therefore $x = \beta^{-1} t$ is complex orientation for $\K$.
\end{example}

\begin{example}
	Recall that $\MU$ is constructed as the colimit $\MU = \colim \MU[n]$.
	Also, $\MU[1] \cong \Sigma^{\infty-2} \BU[1]$.
	Therefore we get a canonical map $\Sigma^{\infty-2} \BU[1] \to \MU$, which gives a cohomology class $x_{\MU} \in \MU^2\left(\BU[1]\right)$.
\end{example}

\begin{proposition}[{\cite[4.1.3]{Rav86}}]
	$x_{\MU}$ is a complex orientation for $\MU$.
\end{proposition}

\begin{theorem}[{\cite[4.1.13]{Rav86}}]
	$\MU$ is the universal complex oriented cohomology theory, in the following sense:
	For any multiplicative cohomology theory $E$, there is a bijection between (homotopy classes of) multiplicative maps $\MU \to E$ and complex orientations on $E$.
	The bijection is given in one direction by pulling back $x_{\MU}$ along a multiplicative map.
\end{theorem}

Assume that $E$ is complex oriented with a complex orientation $x$.

\begin{proposition}[{\cite[4.1.4]{Rav86}}]
	As $E_*$-algebras,
	$E^*\left(\BU[1]\right) \cong E^*\formal{x}$
	and
	$E^*\left(\BU[1] \times \BU[1]\right) \cong E^*\formal{y,z}$.
\end{proposition}

There is a multiplication map for the group $\UU[1]$, i.e. $\UU[1] \times \UU[1] \to \UU[1]$.
We can take the $\BB$ of this map, and since it commutes with products we get a map $\BU[1] \times \BU[1] \to \BU[1]$, which is the universal map that classifies the tensor product of vector bundles.
Therefore we get a map $E^*\left(\BU[1]\right) \to E^*\left(\BU[1] \times \BU[1]\right)$, which is completely determined by the image of $x \in E^*\formal{x}$ in $E^*\formal{y,z}$ as above.
We conclude that a choice of a complex orientation on $E$ gives rise to an element $F_E\left(y,z\right) \in E^*\formal{y,z}$.

\begin{proposition}[{\cite[4.1.4]{Rav86}}]
	$F_E$ is a formal group law on $E_*$.
\end{proposition}

\begin{definition}
	The \emph{height} of $E$ is simply the height of $F_E$.
\end{definition}

\begin{example}\label{HR-2}
	We continue with $\HH{R}$ from \cref{HR-1}.
	It is known that the tensor of complex line bundles induces the map
	$$
	R\formal{x}
	= \HH{R}^*\left(\BU[1]\right)
	\to \HH{R}^*\left(\BU[1] \times \BU[1]\right)
	= R\formal{y,z},
	$$
	given by $x \mapsto y + z$.
	This is the additive formal group law.
	It is immediate that $\left[p\right]\left(x\right) = p x$.
	So for $R = \QQ$ we get that the height of $\HH{\QQ}$ is 0, while for $R = \Fp$ we have $p x = 0$ so the height of $\HH{\Fp}$ is $\infty$.
\end{example}

\begin{example}[$\K$-Theory Saga: Formal Group Law]\label{k-thy-fgl}
	We return to complex K-theory from \cref{k-thy-oriented}.
	It is known that the tensor of complex line bundles induces the map
	$$
	\K_*\formal{t}
	= \K^*\left(\BU[1]\right)
	\to \K^*\left(\BU[1] \times \BU[1]\right)
	= \K_*\formal{u,v},
	$$
	given by $t \mapsto u + v + u v$.
	Note that to comply with the definition of the formal group law, we should use the isomorphism
	$\K^*\left(\BU[1]\right) \cong \K_* \formal{x}$,
	i.e. the element $x = \beta^{-1	} t$.
	By multiplying by $\beta^{-1}$ (recall that the map is of $\K_*$-modules) we get that
	$$
	x
	= \beta^{-1} t \mapsto \beta^{-1} u + \beta^{-1} v + \beta^{-1} u v
	= y + z + \beta y z
	= F_{\K}\left(y,z\right).
	$$
	By induction we prove that the $n$-series is $\left[n\right]\left(x\right) = \beta^{-1} \left(1 + \beta x\right)^n - \beta^{-1}$.
	This is clear for $n = 1$, and we have:
	\begin{align*}
		\left[n+1\right]\left(x\right)
		&= x + \left[n\right]\left(x\right) + \beta x \left[n\right]\left(x\right)\\
		&= x + \beta^{-1} \left(1 + \beta x\right)^n - \beta^{-1} + x \left(1 + \beta x\right)^n - x\\
		&= \beta^{-1} \left(1 + \beta x\right) \left(1 + \beta x\right)^n - \beta^{-1}\\
		&= \beta^{-1} \left(1 + \beta x\right)^{n+1} - \beta^{-1}
	\end{align*}
\end{example}

\begin{example}[$\K$-Theory Saga: mod-$p$]\label{k-thy-modp-height}
	By taking the cofiber of the multiplication-by-$p$ map, we get a spectrum $\K/p$, mod-$p$ K-theory, with coefficients $\left(\K/p\right)_* = \mathbb{F}_p \left[\beta^{\pm 1}\right]$.
	It is evident that $F_{\K/p}\left(y,z\right) = y + z + \beta y z$ as well.
	From the result above, it follows that
	$$
	\left[p\right]\left(x\right)
	= \beta^{-1} \left(1 + \beta x\right)^p - \beta^{-1}
	= \beta^{-1} \left(1^p + \beta^p x^p\right) - \beta^{-1}
	= \beta^{p-1} x^p,
	$$
	which shows that the height is exactly $1$.
\end{example}

A formal group law on $E_*$ is the same data as a map from the Lazard ring $L$, so the complex orientation gives a map $L \to E_*$.
In particular, since $\MU$ is complex oriented, there is a canonical map $L \to \MU_*$.

\begin{theorem}[{Quillen, \cite[4.1.6]{Rav86}}]\label{quillen-theorem}
	The canonical map $L \to \MU_*$ is an isomorphism.
\end{theorem}



\subsection{\texorpdfstring{$\BP$}{BP}, Morava K-Theory and Morava E-Theory}

A good principle in homotopy theory (and in many other areas in math) is to study it one prime at a time.
This is possible in homotopy theory due to the arithmetic square.
So, let us fix a prime $p$.
We can form $\MU_{\left(p\right)}$, the $p$-localization of $\MU$.

\begin{theorem}[{\cite[II 15]{Ada}}]
	There exists a unique map of ring spectra $\varepsilon: \MU_{\left(p\right)} \to \MU_{\left(p\right)}$ (depending on the prime $p$) satisfying:
	\begin{itemize}
		\item $\varepsilon$ is an idempotent, i.e. $\varepsilon^2 = \varepsilon$,
		\item $\varepsilon_*$ sends $\left[\CP{n}\right] \in \pi_*\left(\MU_{\left(p\right)}\right)$ to itself if $n = p^r-1$ and to $0$ otherwise.
	\end{itemize}
\end{theorem}

The map $\varepsilon: \MU_{\left(p\right)} \to \MU_{\left(p\right)}$ gives a cohomology operation, for every $X$ we have $\varepsilon^*: \MU_{\left(p\right)}^*\left(X\right) \to \MU_{\left(p\right)}^*\left(X\right)$.
Denote by $\BP_{\left(p\right)}^*\left(X\right)$ the image of $\varepsilon^*$.

\begin{theorem}[{\cite[II 16]{Ada}, \cite[4.1.12]{Rav86}}]
	$\BP$ is a cohomology theory, represented by an associative commutative ring spectrum $\BP$ (depending on the prime $p$), which is a retract of $\MU_{\left(p\right)}$.
	The homotopy groups of $\BP$ are $\BP_* = \ZZ_{\left(p\right)}\left[v_1, v_2, \dotsc\right]$ where $\left|v_n\right| = 2\left(p^n-1\right)$.
\end{theorem}

For convenience we denote $v_0 = p$ (and indeed $\left|v_0\right| = 2\left(p^0-1\right) = 0$).
Since $\BP$ is a retract of $\MU$, it comes with a map $\MU \to \BP$, that is, a complex orientation.

\begin{proposition}[{\cite[4.1.12 combined with A2.1.25 and A2.2.4]{Rav86}}]\label{bp-p-series}
	The $p$-series of the formal group law associated to $\BP$ is
	$\left[p\right]\left(x\right) = \sum^F v_n x^{p^n}$ (note that the sum on the right hand side is in the formal group law).
\end{proposition}

\begin{remark}[{\cite[B.5]{Rav92}}]
	The formal group law on $\BP$ has a similar interpretation to that of $\MU$, namely it is the universal $p$-typical formal group law.
	Moreover, the idempotent $\varepsilon: \MU_{\left(p\right)} \to \MU_{\left(p\right)}$ induces an idempotent on homotopy groups, which can be described as the map that takes a formal group law to the canonical $p$-typical formal group law isomorphic to it.
\end{remark}

Once we have $\BP$, we can turn to the definition of Morava K-theory and Johnson-Wilson spectrum (a variant of Morava E-theory).

\begin{definition}
	Let $0 < n < \infty$.
	\emph{Morava K-theory} at height $n$ and prime $p$, denoted by $\K[p,n]$ or $\K[n]$ when the prime is clear from the context, is the spectrum obtained by killing $p=v_0, \dotsc, v_{n-1}, v_{n+1}, \dotsc$ in $\BP$ and inverting $v_n$.
	Therefore $\K[n]_* = \Fp\left[v_n^{\pm 1}\right]$.
	We also define $\K[0] = \HH{\QQ}$ and $\K[\infty] = \HH{\Fp}$.
	Similarly, \emph{Johnson-Wilson spectrum} (sometimes also called Morava E-theory) at height $n$ and prime $p$, denoted by $\E{p,n}$ or $\E{n}$, is the spectrum obtained by killing $v_{n+1}, v_{n+2}, \dotsc$ in $\BP$ and inverting $v_n$.
	Therefore $\E{n}_* = \ZZ_{\left(p\right)}\left[v_1, \dotsc v_{n-1}, v_n^{\pm 1}\right]$.
\end{definition}

Since Morava K-theory and E-theory are obtained from $\BP$ by cofibers and filtered colimits, they are equipped with a map from $\BP$, hence also with a complex orientation.
Then, from \cref{bp-p-series}, we get:

\begin{corollary}\label{k-e-p-series}
	The $p$-series associated to the formal group laws of $\K[n]$ and $\E{n}$ are $v_n x^{p^n}$ and $v_0 x +_F \dotsc +_F v_n x^{p^n}$ respectively.
	Therefore the height of $\K[n]$ is exactly $n$.
	(Note that by \cref{HR-2}, this is also true for $\K[0]$ and $\K[\infty]$.)
\end{corollary}

We want to describe some properties of Morava K-theory.
To do so we first need some definitions.

\begin{definition}
	Let $R$ be an evenly graded ring.
	$R$ is called a \emph{graded field} if it satisfies one of the equivalent conditions:
	\begin{itemize}
		\item every non-zero homogenus element is invertible,
		\item it is a field $F$ concentrated at degree 0, or $F\left[\beta^{\pm1}\right]$ for $\beta$ of positive even degree.
	\end{itemize}
	An $\mbb{A}_\infty$-ring $E$ is a \emph{field} if $E_*$ is a graded field.
\end{definition}

\begin{example}
	$K\left(n\right)$ is a field for $0 \leq n \leq \infty$.
\end{example}

\begin{proposition}
	A field $E$ has K\"unneth, i.e. $E_*\left(X\otimes Y\right)\cong E_*\left(X\right)\otimes_{E_*}E_*\left(Y\right)$ for any spectra X,Y.
\end{proposition}

\begin{proposition}[{\cite[24]{Lur}}]
	Let $E \neq 0$ be a complex oriented cohomology theory, whose formal group law has height exactly $n$, then $E \otimes \K[n] \neq 0$.
	Let $E$ be a field s.t. $E \otimes \K[n] \neq 0$, then $E$ admits the structure of a $\K[n]$-module.
	(Here $0 \leq n \leq \infty$.)
\end{proposition}

\begin{example}[K-Theory Saga: Morava K-Theory]\label{k-thy-modp-morava}
	As we have seen in \cref{k-thy-modp-height}, mod-$p$ K-theory, $\K/p$, has height exactly $1$ and coefficients $\left(\K/p\right)_* = \mathbb{F}_p \left[\beta^{\pm 1}\right]$.
	It is also known that $\K$ and $\K/p$, are $\mbb{A}_\infty$-ring spectra, from which it follows that $\K/p$ is a field.
	We deduce that $\K/p$ is a $\K[1]$-module.
	Since $\left|\beta\right| = 2$ and $\left|v_1\right| = 2\left(p-1\right)$ it is free of rank $p-1$.
\end{example}

From this we also deduce some form of uniqueness for Morava K-theory:

\begin{corollary}
	Let $E$ be a field with $E_* \cong \Fp\left[v_n^{\pm 1}\right]$, which is also complex oriented of height exactly $n$.
	Then $E \cong \K[n]$ (as spectra).
\end{corollary}



\subsection{\texorpdfstring{$\spec \SS_{\left(p\right)}$}{spec S(p)} and \texorpdfstring{$\spec \SS$}{spec S}}

We are now in a position to describe the topological space $\spec \SS$.
However, it will be easier to state it first for $\spec \SS_{\left(p\right)}$, and then pullback prime ideals.
We know that $\Mod{\SS_{\left(p\right)}} = \Sp_{\left(p\right)}$, and its compact objects are $\Spfin_{\left(p\right)}$, the $p$-localizations of finite spectra.

\begin{proposition}\label{T_E-thick}
	Let $\mcl{T}_E$ be the $E$-acyclics, i.e.
	$$
	\mcl{T}_E
	= \ker E_*
	= \left\{ X \in \Spfin_{\left(p\right)} \mid E_*\left(X\right)=0 \right\}
	= \left\{ X \in \Spfin_{\left(p\right)} \mid X \otimes E=0 \right\}.
	$$
	Then $\mcl{T}_E$ is thick.
\end{proposition}

\begin{proof}
	The proof follows the same lines of \cref{chperf-tk} for the case $\Chperf{R}$.
\end{proof}

\begin{definition}
	We define $\mcl{C}_{p, n} = \mcl{T}_{\K[n]}$, the $\K[n]$-acyclics.
	By the above proposition, it is thick.
	Also, $\mcl{C}_{p, \infty} = \left\{ 0\right\}$, which is trivially thick.
	When the prime is clear from the context, we write $\mcl{C}_n$ in place of $\mcl{C}_{p, n}$.
\end{definition}

\begin{proposition}[{\cite[26]{Lur}}]
	For $X \in \Spfin_{\left(p\right)}$, if $\K[n]_*\left(X\right) = 0$ then $\K[n-1]_*\left(X\right) = 0$.
\end{proposition}

\begin{definition}
	We say that a spectrum $X \in \Spfin_{\left(p\right)}$ is of \emph{type} $n$ (possibly $\infty$) if its first non-zero Morava K-theory homology is $\K[n]$.
\end{definition}

\begin{corollary}
	$\mcl{C}_n$ is the full subcategory of finite $p$-local spectra of type $> n$, that is $\mcl{C}_n = \left\{ X \in \Spfin_{\left(p\right)} \mid \forall m \leq n: \K[m]_*\left(X\right) = 0 \right\}$.
	Therefore, we also conclude that $\mcl{C}_{n+1} \subseteq \mcl{C}_n$.
\end{corollary}

\begin{proposition}
	The inclusions $\mcl{C}_{n+1} \subset \mcl{C}_n$ are proper.
\end{proposition}

\begin{remark}
	The modern proof of this result relies on the periodicity theorem \cite[1.5.4]{Rav92}.
	Using it, we can construct generalized Moore spectra, which give an example of spectra of type $n$ for every $n$.
\end{remark}

\begin{proposition}
	If $X \in \Spfin_{\left(p\right)}$ is not contractible, then it is of finite type.
	Therefore $\bigcap_{n < \infty} \mcl{C}_n = \left\{0\right\} = \mcl{C}_\infty$.
\end{proposition}

\begin{proof}
	Let $X$ be non-contractible.
	Then $\HH{\ZZ}_*\left(X\right) \neq 0$.
	Let $m$ be the first non-zero degree.
	Using the universal coefficient theorem and the fact that the spectrum is $p$-local we get that $\left(\HH{\Fp}\right)_m\left(X\right) \neq 0$, thus $\left(\HH{\Fp}\right)_*\left(X\right) \neq 0$.
	Since $X$ is finite, $\left(\HH{\Fp}\right)_*\left(X\right)$ is bounded.
	The Atiyah-Hirzebruch spectral sequence for $X$ with cohomology $\K[n]$ has $E^2$-page given by
	$
	E_{t,s}^2
	= 
	H_t\left(X; \K[n]_s\right)
	$.
	Since $\K[n]_s = \Fp$ for $s = 0 \mod 2\left(p^n-1\right)$ and $0$ otherwise,
	we see that the rows $s = 0 \mod 2\left(p^n-1\right)$ are $\left(\HH{\Fp}\right)_*\left(X\right)$, and the others are $0$.
	Therefore if we take $n$ such that the period $2\left(p^n-1\right)$ is larger then the bound on $\left(\HH{\Fp}\right)_*\left(X\right)$, then all differentials have either source or target $0$.
	Thus, the spectral sequence collapses at the $E^2$-page, and since $\left(\HH{\Fp}\right)_*\left(X\right) \neq 0$, we get that $\K[n]\left(X\right) \neq 0$, i.e. $X$ has type $\leq n$.
\end{proof}

\begin{proposition}
	$\mcl{C}_n$ is a prime ideal.
\end{proposition}

\begin{proof}
	Recall from \cref{T_E-thick} that we already know that it is thick.
	For $X,Y \in \Spfin_{\left(p\right)}$, by K\"unneth we have 
	$$
	\K[n]_*\left(X \otimes  Y\right)
	= \K[n]_*\left(X\right) \otimes \K[n]_*\left(Y\right).
	$$
	Assume that $X \in \mcl{C}_n$, that is $\K[n]_*\left(X\right) = 0$.
	It follows that $\K[n]_*\left(X \otimes  Y\right) = 0$, i.e. $X \otimes Y \in \mcl{C}_n$, so $\mcl{C}_n$ is an ideal.
	Assume that $X \otimes Y\in \mcl{C}_n$, that is $\K[n]_*\left(X \otimes  Y\right) = 0$, therefore one of the terms in the RHS of the equation must vanish (since they are graded vector spaces), so $\mcl{C}_n$ is a prime ideal.
\end{proof}

\begin{theorem}[{Thick Subcategory Theorem \cite[theorem 7]{HS}}]\label{thick-subcategory-thm}
	If $\mcl{T}$ is a proper thick subcategory of $\Spfin_{\left(p\right)}$, then $\mcl{T} = \mcl{C}_n$ for some $0 \leq n \leq \infty$.
\end{theorem}

\begin{remark}
	The proof relies on a major theorem called the Nilpotence Theorem.
\end{remark}

\begin{corollary}
	$\spec \SS_{\left(p\right)} = \left\{ \mcl{C}_0, \mcl{C}_1, \dotsc, \mcl{C}_\infty \right\}$,
	and the closed subsets in the topology are chains
	$\left\{ \mcl{C}_k, \mcl{C}_{k+1}, \dotsc, \mcl{C}_\infty \right\}$
	for some $0 \leq k \leq \infty$.
\end{corollary}

\begin{proof}
	Follows immediately from the previous results.
\end{proof}

We now want to describe $\spec \SS$.
Note that the $p$-localization functor $L_{\left(p\right)}$ is a Bousfield localization.
As such, it is left (its right adjoint is the inclusion), and in particular preserves cofibers.
It is also reduced, i.e. sends $0$ to $0$.
Now, $L_{\left(p\right)}$ is smashing, that is $L_{\left(p\right)} X = X \otimes \SS_{\left(p\right)}$, so it is also symmetric monoidal.
As we have seen in \cref{primes-pullback}, under these conditions we can pullback primes.
Since $L_{\left(p\right)}$ is smashing and $\K\left(p,n\right)$ is $p$-local for every $0 \leq n \leq \infty$, we have that $\K[n]_*\left(X_{\left(p\right)}\right) = \K[n]_*\left(X\right)$.
Therefore
$$
\mcl{P}_{p, n}
= L_{\left(p\right)}^* \mcl{C}_{p, n}
= \left\{
	X \in \Spfin
	\mid
	\K[p,n]_*\left(X\right) = 0
\right\}
$$
and
$$
\mcl{P}_{p, \infty}
= L_{\left(p\right)}^* \mcl{C}_{p, \infty}
= \left\{
	X \in \Spfin
	\mid
	X_{\left(p\right)} = 0
\right\}
$$
are prime ideals.
Moreover, by definition $\K[p,0] = \HH{\QQ}$, which implies that
$
\mcl{P}_{p, 0}
= \left\{
	X \in \Spfin
	\mid
	{\HH{\QQ}}_*\left(X\right) = 0
\right\}
$.
We see that $\mcl{P}_{p, 0}$ is independent of $p$, and we denote it by $\Spfintor$.
Again, by \cref{primes-pullback}, we see that any chain $\left\{ \mcl{P}_{p, k}, \mcl{P}_{p, k+1}, \dotsc, \mcl{P}_{p, \infty} \right\}$ for some $p$ and $0 \leq k \leq \infty$, is closed, hence any finite union of such chain is also closed.
We then have the following theorem, which says that these are all of the prime ideals and all of the closed subsets.

\begin{theorem}[{Thick Subcategory Theorem, \cite[9.5]{BalSSS}}]
	$
	\spec \SS
	= \left\{\Spfintor\right\}
	\bigcup \left(\bigcup_p \left\{\mcl{P}_{p, 1}, \dotsc, \mcl{P}_{p, \infty}\right\}\right)
	$,
	and the closed subsets in the topology are finite unions of chains
	$\left\{ \mcl{P}_{p, k}, \mcl{P}_{p, k+1}, \dotsc, \mcl{P}_{p, \infty} \right\}$
	for some $0 \leq k \leq \infty$ (i.e. they may include $\Spfintor$).
\end{theorem}

\begin{remark}
	The following diagram shows the structure of $\spec \SS$.
	Each $\mcl{P}_{p,n}$, and $\Spfintor$, is a point.
	A line represents that the closure of the point at the bottom contains the point at the top.
	$$
		\begin{tikzcd}[row sep=small, column sep=scriptsize]	
		\mcl{P}_{2, \infty} & \mcl{P}_{3, \infty} & \cdots & \mcl{P}_{p, \infty} & \cdots\\
		\vdots \arrow[dash]{u}{} & \vdots \arrow[dash]{u}{} &  & \vdots \arrow[dash]{u}{} & \\
		\mcl{P}_{2, n} \arrow[dash]{u}{} & \mcl{P}_{3, n} \arrow[dash]{u}{} & \cdots & \mcl{P}_{p, n} \arrow[dash]{u}{} & \cdots\\
		\vdots \arrow[dash]{u}{} & \vdots \arrow[dash]{u}{} &  & \vdots \arrow[dash]{u}{} & \\
		\mcl{P}_{2, 2} \arrow[dash]{u}{} & \mcl{P}_{3, 2} \arrow[dash]{u}{} & \cdots & \mcl{P}_{p, 2} \arrow[dash]{u}{} & \cdots\\
		\mcl{P}_{2, 1} \arrow[dash]{u}{} & \mcl{P}_{3, 1} \arrow[dash]{u}{} & \cdots & \mcl{P}_{p, n} \arrow[dash]{u}{} & \cdots\\
		& & \Spfintor \arrow[dash]{ull}{} \arrow[dash]{ul}{} \arrow[dash]{ur}{}
	\end{tikzcd}
	$$
\end{remark}

\begin{remark}
	Thick subcategories are interesting for another reason, unrelated to the Balmer spectrum point of view, namely they give a very powerful proof method.
	Say we have a property that is satisfied by $0$, and is closed under cofibers and retracts.
	It follows that the collection of objects that satisfy it is thick.
	Then, for example, by the thick subcategory theorem \ref{thick-subcategory-thm}, it is enough to find one object in $\mcl{C}_n \setminus \mcl{C}_{n+1}$ that satisfies the property, to show that all objects in $\mcl{C}_n$ satisfy it.
\end{remark}



\subsection{The Stacky Point of View and the Relationship Between Morava K-Theory and Morava E-Theory}

First we will describe, without being precise, another point of view on what chromatic homotopy theory is about.

There is a stack of formal groups with strict isomorphisms, denoted by $\Mfgs$.
It can be described as the stack that sends a ring $R$ to the groupoid of formal group laws, with strict isomorphisms between them.
Quillen theorem \ref{quillen-theorem} tells us that $\MU_*$ is the Lazard ring, that is the universal ring that carries the universal formal group law.
This theorem has a second part, which says that $\left(\MU \otimes \MU\right)_*$ is the universal ring that carries two formal group laws and a strict isomorphism between them.
Therefore, $\Mfgs$ is represented by $\left(\MU_*, \left(\MU \otimes \MU\right)_*\right)$.

The geometric points of the stack $\Mfgs$ are precisely the same as $\spec \SS$, that is because for an algebraically closed field of characteristic $0$ there is a unique (up to isomorphism) formal group law which is of height $0$ namely the additive formal group law, and for characteristic $p$ there is a unique (up to isomorphism) formal group law of each height $1 \leq n \leq \infty$.

For a spectrum $X$, $\MU_*\left(X\right)$ is a $\left(\MU_*, \left(\MU \otimes \MU\right)_*\right)$-comodule, which is the same as a sheaf over $\Mfgs$.
From this point of view, chromatic homotopy theory lets us study a spectrum by decomposing it over the stack $\Mfgs$.

We can restrict ourselves to the stack only over $p$-local rings, $\Mfgsp$, which is then represented by $\left(\left(\MU_{\left(p\right)}\right)_*, \left(\MU_{\left(p\right)} \otimes \MU_{\left(p\right)}\right)_*\right)$.
Similarly to $\MU$, $\BP$ is universal ring with the universal $p$-typical formal group law, and $\left(\BP \otimes \BP\right)_*$ is the universal ring with two $p$-typical formal group laws and a strict isomorphism between them.
Since every formal group law is isomorphic to a unique $p$-typical one, we know that the stack $\Mfgsp$ is also represented by $\left(\BP_*, \left(\BP \otimes \BP\right)_*\right)$.

It is now reasonable that $\K[n]$, obtained from $\BP$ by killing the $v_m$'s for $m \neq n$ and inverting $v_n$, sees the $n$-th level, and that $\E{n}$ obtained in the same way but only killing $v_m$ for $m > n$, sees the levels $\leq n$.
Following this point of view, the following theorem can be proven:

\begin{theorem}[{\cite[23, proposition 2]{Lur}}]
	$\E{n}$ and $\K[0] \oplus \cdots \oplus \K[n]$ are Bousfield equivalent.
	That is, they have the same acyclics, locals, and their localization functors are the same.
\end{theorem}



\subsection{Landweber Exact Functor Theorem}

As we have seen, a complex orientation on a cohomology theory, which is given by a map $\MU \to E$, has an associated formal group law, which is given by the map $L = \MU_* \to E_*$.
Note that this formal group law is of degree $-2$, by virtue of the grading on $L = \MU_*$.
One can ask whether the converse is true, namely given a graded ring $R$ and a formal group law $F$ of degree $-2$ given by $L \to R$, is there a complex oriented cohomology theory whose coefficients are $R$ and the associated formal group law is $F$.

A strategy is to define $\left(E_{R,F}\right)_*\left(X\right) = \MU_*\left(X\right) \otimes_{\MU_*} R$.
Unfortunately, this is not always a homology theory.
However there is a condition that one can check, which guarantees that it is.

\begin{definition}
	$L \to R$ is called \emph{Landweber flat} if for every prime $p$, the image of the sequence $p = v_0, v_1, v_2, \dotsc$ in $R$, is regular.
	That is, for every prime $p$ and $n \geq 0$, $v_n$ is not a zero divisor in $R/\left(v_0, v_1, \dotsc, v_{n-1}\right)$.
\end{definition}

\begin{remark}
	Recall that for the formal group law over $\BP$ we have $\left[p\right]\left(x\right) = \sum^F v_n x^{p^n}$.
	Seemingly $v_n$ is a coefficient in a complicated sum involving $F$ itself, however, modulo $\left(v_0, v_1, \dotsc, v_{n-1}\right)$ (which is all we need for Landweber flatness), in fact $v_n$ is simply the coefficient of $x^{p^n}$ in the $p$-series expanded to a usual power series.
	To see this, first of all note that
	$
	\sum^F v_k x^{p^k}
	= F\left(\sum_{k \leq n}^F v_k x^{p^k}, \sum_{k > n}^F v_k x^{p^k}\right)
	$.
	The second term may contribute only powers higher than $p^n$.
	Moreover, the first term modulo $\left(v_0, v_1, \dotsc, v_{n-1}\right)$ is simply $v_n x^{p^n}$, and the conclusion follows.
\end{remark}

\begin{remark}\label{landweber-flat-helper}
	If $p$ is invertible in $R$, then $p = v_0$ is invertible, and $R/p$ is already $0$, so we don't need to check $v_1, v_2, \dotsc$.
\end{remark}

\begin{theorem}[{Landweber Exact Functor Theorem (LEFT), \cite[15, 16]{Lur}}]\label{LEFT}
	If $L \to R$ is Landweber flat, then $E_{R,F}$ defined above is a homology theory.
	Moreover, there are no phantom maps between such spectra, so $E_{R,F}$ is represented by a spectrum.
	This spectrum is complex oriented, has coefficients $R$ and associated formal group law $F$.
\end{theorem}

\begin{example}
	Johnson-Wilson spectrum $\E{n}$ is Landweber flat, since by \cref{k-e-p-series}, the $p$-series has coefficients $p = v_0, v_1, \dotsc, v_n$.
	$p$ is not a zero divisor in $\E{n}_* = \ZZ_{\left(p\right)}\left[v_1, \dotsc v_{n-1}, v_n^{\pm 1}\right]$.
	Then $v_i$ is not a zero divisor in $\E{n}_*/\left(p, v_1, \dotsc, v_{i-1}\right) \cong \Fp\left[v_i, \dotsc v_{n-1}, v_n^{\pm 1}\right]$.
	After $v_n$ the ring becomes $0$ and we are done.
	For other primes, by \cref{landweber-flat-helper} we are done.
\end{example}

\begin{example}
	Morava K-theory $\K[n]$ for $n > 0$ is not Landweber flat since $p = v_0$ is $0$ in $\K[n]_* = \Fp\left[v_n^{\pm 1}\right]$.
\end{example}

\begin{example}
	$\HH{\ZZ}$ is not Landweber flat since although $p = v_0$ is invertible, as we have seen in \cref{HR-2} the $p$-series is $px$, so $v_1$ is $0$ in $\ZZ/p = \Fp$.
\end{example}

We can also ask the following question: given complex oriented cohomology theory $\MU \to E$, such that $L \to E_*$ is Landweber flat, is $E_{R,F}$ equivalent to $E$?
The answer is yes, at least in some cases.

\begin{theorem}\label{LEFT-even}
	Let $E$ be as above, which is also evenly graded (i.e. $E_*$ is an evenly graded ring), then there is an equivalence $E_{R,F} \to E$.
\end{theorem}

\begin{proof}
	This is a slight variation on \cite[18, proposition 11]{Lur}.
	First note that for every spectrum $X$ we have $\MU \otimes X \to E \otimes X$, which induces $\MU_*\left(X\right) \to E_*\left(X\right)$, a map of $\MU_*$-modules.
	Moreover, since $E_* \to E_*\left(X\right)$ is a map of $E_*$-module, the map $\MU_* \to E_*$ makes it a map of $\MU_*$-modules.
	Together this gives a map $\left(E_{R,F}\right)_*\left(X\right) = \MU_*\left(X\right) \otimes_{\MU_*} E_* \to E_*\left(X\right)$.
	This map is a map of homology theories.
	By \cite[17, theorem 6]{Lur}, this map lifts to a map of spectra $E_{R,F} \to E$.
	Since by construction when $X = \SS$ the map above is $E_* \to E_*$ which is an isomorphism, we see that the map $E_{R,F} \to E$ is an equivalence.
\end{proof}

\begin{example}[K-Theory Saga: Landweber Flatness]\label{k-thy-comp-left}
	We return to complex K-theory, from \cref{k-thy-oriented} and \cref{k-thy-fgl}.
	We can take the completion at the element $p \in \K_*$, which gives the spectrum $\K_p^\wedge$.
	This spectrum has coefficients
	$
	\left(\K_p^\wedge\right)_*
	= \left(\K_*\right)_p^\wedge
	= \left(\ZZ\left[\beta^{\pm 1}\right]\right)_p^\wedge
	= \ZZ_p\left[\beta^{\pm 1}\right]
	$.
	The formal group law, as we have seen, is given by $F_{\K_p^\wedge}\left(y,z\right) = y + z + \beta y z$.
	We claim that $F_{\K_p^\wedge}/\K_p^\wedge$ is Landweber flat.
	Clearly $p = v_0$ is not a zero divisor in $\ZZ_p\left[\beta^{\pm 1}\right]$.
	As we have seen in \cref{k-thy-modp-height}, mod-$p$ the $p$-series is $\beta^{p-1} x^p$, so that $v_1 = \beta^{p-1}$ which is not a zero divisor $\Fp\left[\beta^{\pm 1}\right]$.
	Modulo $v_1$ the ring is already $0$, and we are done.
	For other primes, by \cref{landweber-flat-helper} we are done.
	Therefore, by \cref{LEFT-even} we get that $\K_p^\wedge \cong E_{\K_p^\wedge, F_{\K_p^\wedge}}$.
\end{example}



\subsection{Lubin-Tate Deformation Theory}\label{LT-def}

The Johnson-Wilson spectrum $\E{n}$, a variant of Morava E-theory we have considered until now was constructed from $\BP$.
As we noted, it is Landweber flat, which indicates that there is another approach to constructing it.
Indeed there is a way to construct a related spectrum, which will be called the Lubin-Tate spectrum, also a variant of Morava E-theory.

To that end, we first define the category $\mrm{CompRing}$ as the category of complete local rings.
The objects are complete local rings $\left(R, \mfk{m}\right)$, we also denote by $\pi: R \to R/\mfk{m}$ the projection.
Morphisms $\varphi: \left(R, \mfk{m}\right) \to \left(S, \mfk{n}\right)$ are local homomorphisms, i.e. a homomorphism $\varphi: R \to S$ s.t. $\varphi\left(\mfk{m}\right) \subseteq \mfk{n}$.
In particular it induces a homomorphism $\varphi/\mfk{m}: R/\mfk{m} \to S/\mfk{n}$, which satisfies $\varphi/\mfk{m} \circ \pi_R = \pi_S \circ \varphi$.

We fix $k$ be a perfect field of characteristic $p$ (i.e. the Frobenius is an isomorphism), and $\Gamma$ a formal group law over $k$ of height $n < \infty$.
Lubin and Tate \cite{LT} considered a moduli problem associated to $\Gamma/k$, described by a functor $\Def: \mrm{CompRing} \to \mrm{Grpds}$.

\begin{definition}
	Let $\left(R, \mfk{m}\right)$ be a complete local ring and denote by $\pi: R \to R/\mfk{m}$ the quotient.
	A \emph{deformation} of $\Gamma/k$ to $\left(R, \mfk{m}\right)$, is $\left(G, i\right)$, where $G$ is a formal group law over $R$, $i: k \to R/\mfk{m}$ is a homomorphism of fields, such that $i^* \Gamma = \pi^* G$.
	A \emph{$\star$-isomorphism} between two deformations to $\left(R, \mfk{m}\right)$, $f: \left(G_1, i_1\right) \to \left(G_2, i_2\right)$, is defined only when $i_1 = i_2$, and consists of an isomorphism $f: G_1 \to G_2$, such that $\pi^* f: i^*\Gamma = \pi^* G_1 \to \pi^* G_2 \to i^*\Gamma$ is the identity, i.e. $f\left(x\right) = x \mod \mfk{m}$.
	These assemble to a groupoid $\Def[R, \mfk{m}]$, whose objects are deformations to $\left(R, \mfk{m}\right)$, and morphisms are $\star$-isomorphisms.
\end{definition}

\begin{remark}
	$\Def[R, \mfk{m}]$ can be seen as the pullback of the groupoids $\FGL{R}$ and $\coprod_{i: k \to R/\mfk{m}} \left\{\Gamma\right\}$ over $\FGL{R/\mfk{m}}$ (where the maps are $G \mapsto q^* G$ and $i \mapsto i^* \Gamma$ respectively).
\end{remark}

\begin{proposition}[/definition]
	The construction $\Def[R, \mfk{m}]$ is functorial.
\end{proposition}

\begin{proof}
	Let $\varphi: \left(R, \mfk{m}\right) \to \left(S, \mfk{n}\right)$ be a local homomorphism.
	
	For a deformation $\left(G, i\right)$ to $\left(R, \mfk{m}\right)$, we define
	$
	\Def[\varphi]\left(G, i\right)
	= \left(\varphi^* G, \varphi/\mfk{m} \circ i\right)
	$.
	Note that $\varphi^* G$ is a formal group law over $S$, and $\varphi/\mfk{m} \circ i: k \to R/\mfk{m} \to S/\mfk{n}$ is a homomorphism.
	Moreover,
	$
	\left(\varphi/\mfk{m} \circ i\right)^* \Gamma
	= \left(\varphi/\mfk{m}\right)^* i^* \Gamma
	= \left(\varphi/\mfk{m}\right)^* \pi_R^* G
	= \left(\varphi/\mfk{m} \circ \pi_R\right)^* G
	= \left(\pi_S \circ \varphi\right)^* G
	= \pi_S^* \varphi^* G
	$,
	which shows that $\Def[\varphi]\left(G, i\right)$ is a deformation to $\left(S, \mfk{n}\right)$.
	
	For a $\star$-isomorphism $f: \left(G_1, i_1\right) \to \left(G_2, i_2\right)$, which is the data of an isomorphism $f: G_1 \to G_2$ such that $\pi_R^* f = \id[i^*\Gamma]$ is the identity, we need to define a $\star$-isomorphism $\Def[\varphi]\left(G, i_1\right) \to \Def[\varphi]\left(G, i_2\right)$.
	Take it to be the isomorphism $\varphi^* f: \varphi^* G_1 \to \varphi^* G_2$, which satisfies
	$
	\pi_S^* \varphi^* f
	= \left(\varphi/\mfk{m}\right)^* \pi_R^* f
	= \left(\varphi/\mfk{m}\right)^* \id[i^*\Gamma]
	= \id[\left(\varphi/\mfk{m}\right)^* i^*\Gamma]
	= \id[\left(\varphi/\mfk{m} \circ i \right)^*\Gamma]
	$.
	The identity $\id[G]: \left(G, i\right) \to \left(G, i\right)$ is clearly sent to $\id[\varphi^* G]$, and compositions are sent to compositions.
	
	This shows that $\Def[\varphi]: \Def[R, \mfk{m}] \to \Def[S, \mfk{n}]$ is indeed a functor.
	Moreover, it is clear that $\Def[{\id[R]}]$ is the identity and compositions are sent to compositions, which shows that $\Def: \mrm{CompRing} \to \mrm{Grpds}$ is indeed a functor.
\end{proof}

\begin{remark}
	We recall quickly that the Witt vectors $\Wk$ is a ring of characteristic $0$, with maximal ideal $\left(p\right)$, and residue field $\Wk/p \cong k$.
	For example, $\WW{\Fp} = \ZZ_p$.
\end{remark}

\begin{theorem}[{\cite[{4.4, 5.10}]{Rez}, originally due to \cite{LT}}]
	The functor $\Def$ lands in discrete groupoids (i.e. $\Def[R, \mfk{m}]$ has $0$ or $1$ morphisms between objects).
	Furthermore the functor $\Def$ is co-represented, that is there exists a \emph{universal deformation}, and the complete local ring can be chosen (non-canonically) to be $\Wkui$.
\end{theorem}

Let us unravel what that means.
First note that the quotient of $\Wkui$ by the maximal ideal $\left(p, u_1, \dotsc, u_{n-1}\right)$ is $k$.
The universal deformation can be chosen such that the formal group law over it $\Gamma_U$ over $\Wkui$ satisfies $\pi^* \Gamma_U$ is $\Gamma$.
The universality means that for $\left(R, \mfk{m}\right)$, the assignment
$$
\hom_{\mrm{CompRing}}\left(\Wkui, R\right) \to \Def[R, \mfk{m}], \quad
\varphi \mapsto \varphi^* \Gamma_U
$$
is an equivalence.

Now, we can form the graded ring $\Wkuiu$ where $\left|u\right| = 2$.
We can define the formal group law $\left(u \Gamma_U\right) \left(x, y\right) = u^{-1} \Gamma_U \left(u y, u z\right)$, which is of degree $-2$, thus we get a map $L \to \Wkuiu$.

\begin{proposition}[{\cite[{6.9}]{Rez}}]
	$L \to \Wkuiu$ is Landweber flat.
\end{proposition}

Using LEFT \ref{LEFT}, we immediately get:

\begin{corollary}\label{LT-spectrum}
	There is a complex oriented cohomology theory $\E{k, \Gamma} = E_{\Wkuiu, u \Gamma_U}$, called \emph{Lubin-Tate spectrum}, with coefficients $\E{k, \Gamma}_* = \Wkuiu$ and associated formal group law $u \Gamma_U$.
\end{corollary}

\begin{example}[K-Theory Saga: Lubin-Tate]\label{k-thy-comp-defo}
	We continue the complex K-theory saga from \cref{k-thy-comp-left}.
	Take the field $k = \Fp$ and the formal group law $\Gamma\left(y, z\right) = y + z + y z$, of height $n = 1$.
	By the above construction, the ring of the universal deformation is $\WW{\Fp} = \ZZ_p$.
	The universal formal group law of the universal deformation can be taken to be $\Gamma_U \left(y, z\right) = y + z + y z$ (this follows from the proof at \cite[5.10]{Rez}, since here $n=1$ so there are no $u_i$'s).
	We look at the ring $\ZZ_p\left[u^{\pm1}\right]$, and at the formal group law over it
	$
	\left(u\Gamma_U\right) \left(y, z\right)
	= u^{-1} \left(u y + u z + u^2 y z\right)
	= y + z + u y z
	$.
	It is then clear that the isomorphism $\ZZ_p\left[u^{\pm1}\right] \to \ZZ_p\left[\beta^{\pm1}\right]$, sends $u\Gamma_U$ to $F_{\K_p^\wedge}$.
	It follows by \cref{LEFT-even} that
	$
	\K_p^\wedge
	\cong E_{\K_p^\wedge, F_{\K_p^\wedge}}
	\cong \E{\Fp, \Gamma}
	$,
	i.e. $p$-complete K-theory $\K_p^\wedge$ is a Lubin-Tate spectrum at height $1$.
\end{example}

This concludes the construction of the Lubin-Tate variant of Morava E-theory.
We can compare the Lubin-Tate spectrum with the previous one, the Johnson-Wilson spectrum.

\begin{proposition}[{\cite[23, example 1]{Lur}}]
	The Lubin-Tate spectrum $\E{k, \Gamma}$ and Johnson-Wilson spectrum $\E{n}$ are Bousfield equivalent.
\end{proposition}
