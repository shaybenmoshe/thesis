\section{Introduction}

Chromatic homotopy theory is a theory which describes the global structure of the category of spectra.
A key player in this theory is Morava E-theory, which is an higher height analogue of complex K-theory.
In \cite{AS}, Atiyah and Segal show that the complex K-theory of classifying spaces of groups, are deeply connected to their representations, therefore it can be studied using character theory.
In \cite{HKR}, Hopkins, Kuhn and Ravenel develop a generalized character theory, which can be used to study the Morava E-theory of classifying spaces of finite groups.
Over the last years, there have been numerous applications of elliptic curves to chromatic homotopy theory.
We give another such application, in the form of concrete computations of HKR generalized character theory at height $2$ using elliptic curves.

\cref{sec:chromatic} sets up the theory of chromatic homotopy theory.
We recall many of the basic results of the topic, omitting most of the proofs, with the goal of introducing Morava K-theory and different flavors of Morava E-theory.

\cref{sec:AS} recalls the Atiyah-Segal theorem \ref{AS-private}, and explain its connection to character theory.
This section, although not logically necessary for what follows, it will serve as motivation and a basic example.

\cref{sec:HKR} gives an account of HKR generalized character theory.
The main result of the theory for us, which appears in the original paper as \cite[Theorem C]{HKR}, is recalled in \cref{theorem-c}.
We give most of the details of the proof, emphasizing and elaborating on some parts, in hope to clarify the original account, and to relate it to the other parts of this paper.

\cref{sec:ell} focuses on HKR at height $2$.
The main contribution of this paper is the explanation of the usage of elliptic curves to carry out explicit computations of HKR.
We conclude with the development of some code that implements parts of this strategy.



\subsection*{Acknowledgments}

First and foremost, I wish to thank my advisor, Tomer Schlank, for the huge amount of patience and time he invested in guiding me through this project.
He has been extremely helpful, both by sharing his deep insights and conceptual understanding, as well as figuring out many details with me along the way.
I wish to thank Tomer's other students, and especially Lior Yanovski and Shaul Barkan, for giving me new perspectives and learning exciting mathematics together with me.
They've been great friends, and made this project much more enjoyable.
I also thank Gal Porat, for numerous conversations and explanations, mostly related to elliptic curves and algebraic geometry.