\documentclass[11pt]{article}
\usepackage{fontspec}
\usepackage{amsmath, amsthm, amssymb}
\usepackage{eucal}
%\usepackage[top=2cm,bottom=2cm,left=2.5cm,right=2cm]{geometry}
%\usepackage[papersize={11.4cm,8.5cm},top=0.3cm,bottom=0.3cm,left=0.3cm,right=0.3cm]{geometry}
\usepackage[bookmarksnumbered=true,colorlinks=true,allcolors=blue]{hyperref}
\usepackage[capitalise]{cleveref}
\usepackage{polyglossia}
\usepackage{tikz-cd}
\usepackage{stmaryrd}
\usepackage{listings}
\setdefaultlanguage{english}

%\usepackage[backend=biber, babel=hyphen, style=authortitle]{biblatex}

\setlength{\parindent}{0em}
\setlength{\parskip}{0.6em}

\lstset{frame=single, basicstyle=\ttfamily, tabsize=4, breaklines=true, postbreak=\mbox{\textcolor{red}{$\hookrightarrow$}\space}}

\numberwithin{equation}{subsection}

\newtheorem{theorem}[equation]{Theorem}
\newtheorem{lemma}[equation]{Lemma}
\newtheorem{proposition}[equation]{Proposition}
\newtheorem{corollary}[equation]{Corollary}

\theoremstyle{definition}
\newtheorem{definition}[equation]{Definition}

\theoremstyle{remark}
\newtheorem{remark}[equation]{Remark}
\newtheorem{example}[equation]{Example}
\newtheorem{question}[equation]{Question}

\newcommand{\ncmd}{\newcommand}

\ncmd{\mbb}[1]{\mathbb{#1}}
\ncmd{\mrm}[1]{\mathrm{#1}}
\ncmd{\mcl}[1]{\mathcal{#1}}
\ncmd{\mfk}[1]{\mathfrak{#1}}

%\ncmd{\todo}[1]{}
\ncmd{\todo}[1]{\textbf{TODO #1}}

\ncmd{\prnt}[1]{\ifstrempty{#1}{}{\left(#1\right)}}
\ncmd{\formal}[1]{\left[\mkern-1mu\left[#1\right]\mkern-1mu\right]}

\ncmd{\id}[1][]{\mrm{id}\ifstrempty{#1}{}{_{#1}}}

\ncmd{\Ch}[1]{\mrm{Ch}\left(#1\right)}
\ncmd{\Chperf}[1]{\mrm{Ch}_\mrm{perf}\left(#1\right)}
\ncmd{\Mod}[1]{\mrm{Mod}_{#1}}
\ncmd{\Sp}{\mrm{Sp}}
\ncmd{\Spaces}{\mcl{S}}
\ncmd{\GrAb}{\rm{GrAb}}
\ncmd{\Spfin}{\Sp^\mrm{fin}}
\ncmd{\Spfintor}{\Spfin_{\mrm{tor}}}
\ncmd{\Grp}{\mrm{Grp}}
\ncmd{\Ab}{\mrm{Ab}}
\ncmd{\TopGrp}{\mrm{TopGrp}}
\ncmd{\Set}{\mrm{Set}}
\ncmd{\GSet}{G\mrm{Set}}
\ncmd{\Ring}{\mrm{Ring}}
\ncmd{\op}{\mrm{op}}

\ncmd{\Mfgs}{\mcl{M}_{\mrm{fg}}^s}
\ncmd{\Mfgsp}{\mcl{M}_{\mrm{fg},p}^s}

\ncmd{\ZZ}{\mbb{Z}}
\ncmd{\QQ}{\mbb{Q}}
\ncmd{\UU}[1][]{\mrm{U}\prnt{#1}}
\ncmd{\Fp}{\mbb{F}_p}
\ncmd{\WW}[1]{W#1}
\ncmd{\Wk}{\WW{k}}
\ncmd{\Wkui}[1][n-1]{\Wk\left[\left[u_1, \dotsc, u_{#1}\right]\right]}
\ncmd{\Wkuiu}[1][n-1]{\Wkui[#1]\left[u^{\pm 1}\right]}

\ncmd{\Qab}{\mbb{Q}^\mrm{ab}}

\ncmd{\RG}[1][G]{\mrm{R}\left(#1\right)}
\ncmd{\RgG}[1][G]{\mrm{R}^*\left(#1\right)}
\ncmd{\RGI}[1][G]{{\RG[#1]}_I^\wedge}

\ncmd{\Gnp}{G_{n,p}}
\ncmd{\cl}[2][]{\mrm{Cl}\ifstrempty{#1}{}{_{#1}}\left(#2\right)}
\ncmd{\chinpG}{\chi_{n,p}^G}
\ncmd{\Fix}[2][n,p]{\mrm{Fix}_{#1}\left(#2\right)}

\ncmd{\CP}[1]{{\mbb{C}\mrm{P}}^{#1}}
\ncmd{\BB}[1][]{{\mrm{B}#1}}
\ncmd{\BG}{\BB[G]}
\ncmd{\EE}[1][]{{\mrm{E}#1}}
\ncmd{\EG}{\EE[G]}
\ncmd{\fG}{\mbb{G}}

\renewcommand{\SS}{\mbb{S}}
\ncmd{\HH}[1]{\mrm{H}#1}
\ncmd{\MU}[1][]{\mrm{MU}\prnt{#1}}
\ncmd{\BP}{\mrm{BP}}
\ncmd{\BU}[1][]{\BB[\UU]\prnt{#1}}

\ncmd{\K}[1][]{\mrm{K}\prnt{#1}}
\ncmd{\E}[1]{\mrm{E}\left(#1\right)}
\ncmd{\KG}{\K_G}

\ncmd{\Def}[1][]{\mrm{Def}\prnt{#1}}
\ncmd{\FGL}[1]{\mrm{FGL}\left(#1\right)}

\DeclareMathOperator{\spec}{Spec}
\DeclareMathOperator{\spf}{Spf}
\DeclareMathOperator{\supp}{Supp}
\DeclareMathOperator{\colim}{colim}
\DeclareMathOperator{\cofib}{cofib}
\DeclareMathOperator{\tr}{tr}
\DeclareMathOperator{\Gal}{Gal}
\DeclareMathOperator{\aut}{Aut}
\DeclareMathOperator{\pro}{Pro}
\DeclareMathOperator{\stab}{St}
\DeclareMathOperator{\GL}{GL}
\DeclareMathOperator{\im}{im}
\DeclareMathOperator{\OO}{\mcl{O}}
\DeclareMathOperator{\GCD}{GCD}

\title{Towards Computation of HKR Generalized Character Theory at Height 2 Using Elliptic Curves}
\author{Shay Ben Moshe}
%\date{}

\setcounter{tocdepth}{2}

\begin{document}
	\maketitle
	
	\tableofcontents
	
	\section{Introduction}

Chromatic homotopy theory describes the global structure of the category of spectra.
A key player in this theory is Morava E-theory, which is a higher height analogue of ($p$-complete) complex K-theory.
In \cite{AS}, Atiyah and Segal show that the complex K-theory of classifying spaces of groups, is deeply connected to their representations. Therefore it can be studied using character theory.
In \cite{HKR}, Hopkins, Kuhn and Ravenel develop a generalized character theory, which can be used to study the Morava E-theory of classifying spaces of finite groups.
Over the last years, there have been numerous applications of elliptic curves to chromatic homotopy theory.
We give another such application, in the form of concrete computations of HKR generalized character theory at height $2$.



\subsection*{Organization}

\cref{sec:chromatic} sets up the theory of chromatic homotopy theory.
We recall many of the basic results in the field, omitting most of the proofs, with the goal of introducing Morava K-theory and different flavors of Morava E-theory.

\cref{sec:AS} recalls the Atiyah-Segal theorem (\cref{AS-private}), and explains its connection to character theory.
This section, although not logically necessary for what follows, will serve as a motivation and as a basic example.

\cref{sec:HKR} gives an account of HKR generalized character theory.
The main result for us, which appears in the original paper as \cite[Theorem C]{HKR}, is recalled in \cref{theorem-c}.
We give most of the details of the proof, emphasizing and elaborating on some parts.
We hope that this will clarify the original account, and relate it to the other parts of this work.

\cref{sec:ell} focuses on height $2$.
The main contribution of this work is the explanation of the usage of elliptic curves to carry out explicit computations of HKR.
We conclude with the development of some computer code, in Macaulay2, that implements parts of this strategy.\footnote{We conjecture that Macaulay3 is necessary for computations at height $3$.}



\subsection*{Acknowledgments}

First and foremost, I wish to thank my advisor, Tomer Schlank, for the huge amount of patience and time he invested in guiding me through this project.
He has been extremely helpful, both by sharing his deep insights and conceptual understanding, as well as figuring out many details with me along the way.
I wish to thank Tomer's other students, and especially Lior Yanovski and Shaul Barkan, for giving me new perspectives and learning exciting mathematics together with me.
They've been great friends, and made this project much more enjoyable.
I also thank Gal Porat, for numerous conversations and explanations, mostly related to elliptic curves and algebraic geometry.
	\section{Overview of Chromatic Homotopy Theory}

Our goal is to give a quick overview of chromatic homotopy theory.
There are various point of views and approaches to the topic, and we shall highlight some of them.
One of our main goals is to introduce Morava K-theory $\K[n]$ and Morava E-theory $\E{n}$, and other variants of Morava E-theory $\E{k,\Gamma}$, and their connection to formal group laws.
Our motivation will be the Balmer spectrum of the sphere spectrum.
We will follow the construction of Morava K-theory and Morava E-Theory, and other related spectra, from the point of view of formal group laws, and use them to describe the Balmer spectrum of the sphere spectrum.



\subsection{The Balmer Spectrum}

We will start with an algebraic motivation.
Let $R$ be a noetherian ring.
Consider the symmetric monoidal stable $\infty$-category $\Ch{R}$ of chain complexes on $R$.
It is then natural to ask how much information about $R$ is encoded in the category $\Ch{R}$.
We will try to recover $\spec R$, as a topological space, from $\Ch{R}$.

\begin{remark}
	Balmer's work \cite[6.3]{BalSpec} actually recovers the structure sheaf as well, but we will not consider the structure sheaf.
\end{remark}

\begin{definition}
	A \emph{perfect complex} is a complex that is quasi-isomorphic to a bounded complex of finitely-generated projective modules.
	These objects are the compact objects in $\Ch{R}$, thus they can be defined categorically.
	Their full subcategory is denoted by $\Chperf{R}$.
\end{definition}

\begin{definition}
	Let $\mcl{C}$ be some symmetric monoidal stable $\infty$-category.
	A full subcategory $\mcl{T}$ is \emph{thick} if:
	\begin{itemize}
		\item $0 \in \mcl{T}$,
		\item it is closed under cofibers,
		\item it is closed under retracts.
	\end{itemize}
\end{definition}

\begin{example}\label{chperf-tk}
	Consider the case $\mcl{C} = \Chperf{R}$ (e.g. over $\ZZ$, chain complexes quasi-isomorphic to bounded chain complexes of finitely-generated free abelian groups).
	Let $K \in \Ch{R}$, and define
	$\mcl{T}_K = \left\{ A \in \Chperf{R} \mid A \otimes K \cong 0 \right\}$.
	We claim that $\mcl{T}_K$ is thick.
	Clearly $0 \in \mcl{T}_K$.
	Let $A \to B$ be a morphism between two complexes in $\mcl{T}$. Since tensor is left, tensoring the cofiber with $K$ is given by
	$\cofib\left(A \to B\right) \otimes K \cong \cofib\left(A \otimes K \to B \otimes K\right) \cong \cofib\left(0 \to 0\right) \cong 0$, therefore the cofiber is indeed in $\mcl{T}_K$.
	Lastly, if $A \to B \to A$ is the identity and $B \otimes K \cong 0$, we get that $\id[A \otimes K]$ factors through $0$, which implies that $A \otimes K$ is $0$, so that $A \in \mcl{T}_K$.
\end{example}

\begin{definition}
	A thick subcategory $\mcl{T}$ is an \emph{ideal} if $A \in \mcl{T}, B \in \mcl{C} \implies A \otimes B \in \mcl{T}$.
	Furthermore, it is a \emph{prime ideal} if it is a proper subcategory, and $A \otimes B \in \mcl{T} \implies A \in \mcl{T} \textrm{ or } B \in \mcl{T}$.
	The \emph{spectrum} of the category is defined similarly to the classical spectrum of a ring:
	As a set, $\spec \mcl{C} = \left\{ \mcl{P}\textrm{ prime ideal} \right\}$.
	For any family of objects $S \subseteq \mcl{C}$ we define $V\left(S\right) = \left\{ \mcl{P} \in \spec \mcl{C} \mid S \cap \mcl{P} = \emptyset \right\}$.
	We topologize $\spec \mcl{C}$ with the Zariski topology by declaring those to be the closed subsets.
	We also denote $\supp\left(A\right) = V\left(\left\{ A \right\}\right)$.
\end{definition}

\begin{example}
	We continue the example of $\mcl{T}_K$.
	Clearly if $A \otimes K \cong 0$ then also $A \otimes B \otimes K \cong 0$, so it is an ideal.
	Let $\mfk{p}$ be a prime ideal in $R$ in the usual sense, and take $K = R_\mfk{p}$ (concentrated at degree $0$),
	then $A \otimes K = A_\mfk{p}$ (level-wise localization).
	We will omit the proof that $\mcl{T}_K$ is a prime, but we shall prove something weaker, namely only the case where $A,B$ are bounded complexes of finitely generate projective modules (and not merely quasi-isomorphic to such complexes).
	Assume then that
	$
	0
	= \left(A \otimes B\right)_\mfk{p}
	= A_\mfk{p} \otimes_{R_\mfk{p}} B_\mfk{p}
	$.
	Assume by negation that $A_\mfk{p}, B_\mfk{p} \neq 0$,
	i.e. $\left(A_n\right)_\mfk{p}, \left(B_m\right)_\mfk{p} \neq 0$ but $\left(A_n\right)_\mfk{p} \otimes_{R_\mfk{p}} \left(B_m\right)_\mfk{p} = 0$ for some $n,m$.
	Since the localization of a projective module is again a projective module, and a projective over a local ring is free, and clearly if the tensor of two free modules vanish then one of them vanishes, it follows that $\left(A_n\right)_\mfk{p} = 0$ or $\left(B_m\right)_\mfk{p} = 0$, which is a contradiction.
	Therefore $\mcl{T}_\mfk{p}$ is a prime ideal.
\end{example}

\begin{theorem}[{\cite[5.6]{BalSpec}}]
	The map $\spec R \to \spec\left(\Chperf{R}\right)$,
	given by $\mfk{p} \mapsto \mcl{T}_\mfk{p} = \left\{ A \mid A_\mfk{p} = 0 \right\}$
	is a homeomorphism.
\end{theorem}

\begin{proposition}[Prime ideals pullback]\label{primes-pullback}
	Let $F: \mcl{C} \to \mcl{D}$ be an exact symmetric monoidal functor, between two symmetric monoidal stable $\infty$-categories.
	Let $\mcl{P} \in \spec \mcl{D}$ be a prime ideal, then $F^*\mcl{P} = F^{-1}\left(\mcl{P}\right) = \left\{ A \in \mcl{C} \mid F\left(A\right) \in \mcl{P} \right\}$ is a prime ideal.
	Moreover, the function we obtain, $F^*: \spec \mcl{D} \to \spec \mcl{C}$, is continuous.
\end{proposition}

\begin{proof}
	We first prove that for $\mcl{P} \in \spec \mcl{D}$, $F^*\mcl{P} \in \spec \mcl{C}$.
	
	Clearly $F\left(0\right) = 0 \in \mcl{P}$ since $F$ is exact, so $0 \in F^*\mcl{P}$.
	Since $F$ preserves cofibers, for $A,B \in F^*\mcl{P}$, i.e. $F\left(A\right), F\left(B\right) \in \mcl{P}$, and a map $A \to B$ we get
	$
	F\left(\cofib\left(A \to B\right)\right)
	= \cofib\left(F\left(A\right) \to F\left(B\right)\right)
	\in \mcl{P}
	$.
	Let $A \to B \to A$ be a retract, that is the composition is the identity, s.t. $B \in F^*\mcl{P}$. We know that $F\left(A\right) \to F\left(B\right) \to F\left(A\right)$ is also a retract by functoriality, thus $F\left(A\right) \in \mcl{P}$, that is $A \in F^*\mcl{P}$.
	We conclude that $F^*\mcl{P}$ is indeed a thick subcategory.
	
	Let $A \in F^*\mcl{P}, B \in \mcl{C}$, since $F$ is monoidal, $F\left(A \otimes B\right) = F\left(A\right) \otimes F\left(B\right) \in \mcl{P}$, so $A \otimes B \in F^*\mcl{P}$, that is $F^*\mcl{P}$ is an ideal.
	
	We claim that $F^*\mcl{P}$ is a proper subcategory, because an ideal is proper if and only if it doesn't contain $1$, and since $F$ is symmetric monoidal it sends $1$ to $1$.
	
	Lastly, assume that $A \otimes B \in F^*\mcl{P}$, again since $F$ is monoidal, $F\left(A \otimes B\right) = F\left(A\right) \otimes F\left(B\right) \in \mcl{P}$, so $A \in F^*\mcl{P}$ or $B \in F^*\mcl{P}$, that is $F^*\mcl{P}$ is a prime idea.
	
	Now we show that $F^*: \spec \mcl{D} \to \spec \mcl{C}$ is continuous.
	So let $V\left(S\right) \subseteq \spec \mcl{C}$ be a closed subset.
	We have:
	\begin{align*}
		\left(F^*\right)^{-1}\left(V\left(S\right)\right)
		&= \left\{ \mcl{P} \in \mcl{D} \mid F^*\mcl{P} \in V\left(S\right) \right\}\\
		&= \left\{ \mcl{P} \in \mcl{D} \mid F^{-1}\left(\mcl{P}\right) \cap S = \emptyset \right\}\\
		&= \left\{ \mcl{P} \in \mcl{D} \mid \mcl{P} \cap F\left(S\right) = \emptyset \right\}\\
		&= V\left(F\left(S\right)\right)
	\end{align*}
	So $\left(F^*\right)^{-1}\left(V\left(S\right)\right)$ is indeed also closed, which shows that $F^*$ is continuous.
\end{proof}

Now, recall that $\Ch{R} \cong \Mod{\HH{R}}$, therefore we can reinterpret the above theorem as $\spec R \cong \spec\left( \Mod{\HH{R}}^\mrm{comp} \right)$ (where the $\mrm{comp}$ denotes the compact objects in the category).
We shall turn this theorem into a definition:

\begin{definition}
	Let $R$ be an $\mbb{E}_\infty$-ring.
	We define the \emph{spectrum} of $R$ to be
	$\spec R = \spec\left( \Mod{R}^\mrm{comp} \right)$.
\end{definition}

A natural question to ask then is what is the topological space $\spec \SS$.
Recall that $\Mod{\SS} = \Sp$, the category of spectra, and that the compact objects in spectra are the finite spectra $\Spfin$.
So, unwinding the definitions, the question can rephrased as finding the prime ideals in $\Spfin$, and their topology.
Chromatic homotopy theory provides an answer to this question.



\subsection{\texorpdfstring{$\MU$}{MU} and Complex Orientations}

Throughout this section, let $E$ be a multiplicative cohomology theory (that is, equipped with a map $E \otimes E \to E$ and $1 \in E_0$, which is associative and unital after taking homotopy groups).

Consider the map $S^2 \to \BU[1]$ classifying the universal complex line bundle.
Concretely, under the identifications $S^2 \cong \CP{1}$ and $\BU[1] \cong \CP{\infty}$, this map can be realized as the inclusion $\CP{1} \subseteq \CP{\infty}$.
This map induces a map
$$
	\tilde{E}^2\left(\BU[1]\right)
	\to \tilde{E}^2\left(S^2\right)
	\cong \tilde{E}^0\left(S^0\right)
	\cong E^0\left(*\right)
	= E_0.
$$
Since $E$ is unital, there is a canonical generator $1 \in E_0$.

\begin{definition}
	$E$ is called \emph{complex oriented} if the map $\tilde{E}^2\left(\BU[1]\right) \to E_0$ is surjective, equivalently, if $1$ is in the image of that map.
	A choice of a lift $x \in \tilde{E}^2\left(\BU[1]\right)$ of $1 \in E_0$ is called a \emph{complex orientation}.
	(Note that $\left|x\right| = -2$ as it is in cohomological degree $2$.)
\end{definition}

\begin{example}\label{HR-1}
	Let $R$ be some ring, and consider $\HH{R}$.
	It is known that
	$\HH{R}^*\left(\CP{n}\right) \cong R \left[x\right] / \left(x^{n+1}\right)$
	and
	$\HH{R}^*\left(\CP{\infty}\right) \cong R \formal{x}$,
	where $\left|x\right| = -2$,
	and the maps induced by the inclusions of projective spaces maps $x$ to $x$.
	Therefore we see that $x \in \HH{R}^2\left(\BU[1]\right)$ is mapped to $x \in \HH{R}^2\left(S^2\right) = R\left\{x\right\}$, which is mapped to $1 \in \HH{R}_0 = R$.
	Hence, $x$ is a complex orientation.
\end{example}

\begin{example}[$\K$-Theory Saga: Complex Orientation]\label{k-thy-oriented}
	Let $\K$ be complex K-theory, then we know that $\K_* = \ZZ\left[\beta^{\pm 1}\right]$ where $\beta$ is the Bott element, with $\left|\beta\right| = 2$.
	It is also known (by the Atiyah-Hirzebruch spectral sequence) that
	$\K^*\left(\CP{n}\right) \cong \K_* \left[t\right] / \left(t^{n+1}\right)$
	and
	$\K^*\left(\CP{\infty}\right) \cong \K_* \formal{t}$
	(here $\left|t\right| = 0$),
	where the maps induced by the inclusions of projective spaces maps $t$ to $t$.
	We deduce that $\beta^{-1} t \in \K^2\left(\BU[1]\right)$ is mapped to $\beta^{-1} t \in \K^2\left(S^2\right) = \ZZ\left\{\beta^{-1} t\right\}$, which is indeed the generator.
	Therefore $x = \beta^{-1} t$ is complex orientation for $\K$.
\end{example}

\begin{example}
	Recall that $\MU$ is constructed as the colimit $\MU = \colim \MU[n]$.
	Also, $\MU[1] \cong \Sigma^{\infty-2} \BU[1]$.
	Therefore we get a canonical map $\Sigma^{\infty-2} \BU[1] \to \MU$, which gives a cohomology class $x_{\MU} \in \MU^2\left(\BU[1]\right)$.
\end{example}

\begin{proposition}[{\cite[4.1.3]{Rav86}}]
	$x_{\MU}$ is a complex orientation for $\MU$.
\end{proposition}

\begin{theorem}[{\cite[4.1.13]{Rav86}}]
	$\MU$ is the universal complex oriented cohomology theory, in the following sense:
	For any multiplicative cohomology theory $E$, there is a bijection between (homotopy classes of) multiplicative maps $\MU \to E$ and complex orientations on $E$.
	The bijection is given in one direction by pulling back $x_{\MU}$ along a multiplicative map.
\end{theorem}

Assume that $E$ is complex oriented with a complex orientation $x$.

\begin{proposition}[{\cite[4.1.4]{Rav86}}]
	As $E_*$-algebras,
	$E^*\left(\BU[1]\right) \cong E^*\formal{x}$
	and
	$E^*\left(\BU[1] \times \BU[1]\right) \cong E^*\formal{y,z}$.
\end{proposition}
\todo{maybe indicate the use of AHSS, maybe explain that every even is complex orientable}

There is a multiplication map for the group $\UU[1]$, i.e. $\UU[1] \times \UU[1] \to \UU[1]$.
We can take the $\BB$ of this map, and since it commutes with products we get a map $\BU[1] \times \BU[1] \to \BU[1]$, which is the universal map that classifies the tensor product of vector bundles.
Therefore we get a map $E^*\left(\BU[1]\right) \to E^*\left(\BU[1] \times \BU[1]\right)$, which is completely determined by the image of $x \in E^*\formal{x}$ in $E^*\formal{y,z}$ as above.
We conclude that a choice of a complex orientation on $E$ gives rise to an element $F_E\left(y,z\right) \in E^*\formal{y,z}$.

\begin{proposition}[{\cite[4.1.4]{Rav86}}]
	$F_E$ is a formal group law on $E_*$.
\end{proposition}

\begin{definition}
	The \emph{height} of $E$ is simply the height of $F_E$.
\end{definition}

\begin{example}\label{HR-2}
	We continue with $\HH{R}$ from \ref{HR-1}.
	It is known that the tensor of complex line bundles induces the map
	$$
	R\formal{x}
	= \HH{R}^*\left(\BU[1]\right)
	\to \HH{R}^*\left(\BU[1] \times \BU[1]\right)
	= R\formal{y,z},
	$$
	given by $x \mapsto y + z$.
	This is the additive formal group law.
	It is immediate that $\left[p\right]\left(x\right) = p x$.
	So for $R = \QQ$ we get that the height of $\HH{\QQ}$ is 0, while for $R = \Fp$ we have $p x = 0$ so the height of $\HH{\Fp}$ is $\infty$.
\end{example}

\begin{example}[$\K$-Theory Saga: Formal Group Law]\label{k-thy-fgl}
	We return to the example of complex K-theory \ref{k-thy-oriented}.
	It is known that the tensor of complex line bundles induces the map
	$$
	\K_*\formal{t}
	= \K^*\left(\BU[1]\right)
	\to \K^*\left(\BU[1] \times \BU[1]\right)
	= \K_*\formal{u,v},
	$$
	given by $t \mapsto u + v + u v$.
	Note that to comply with the definition of the formal group law, we should use the isomorphism
	$\K^*\left(\BU[1]\right) \cong \K_* \formal{x}$,
	i.e. the element $x = \beta^{-1	} t$.
	By multiplying by $\beta^{-1}$ (recall that the map is of $\K_*$-modules) we get that
	$$
	x
	= \beta^{-1} t \mapsto \beta^{-1} u + \beta^{-1} v + \beta^{-1} u v
	= y + z + \beta y z
	= F_{\K}\left(y,z\right).
	$$
	By induction we prove that the $n$-series is $\left[n\right]\left(x\right) = \beta^{-1} \left(1 + \beta x\right)^n - \beta^{-1}$.
	This is clear for $n = 1$, and we have:
	\begin{align*}
		\left[n+1\right]\left(x\right)
		&= x + \left[n\right]\left(x\right) + \beta x \left[n\right]\left(x\right)\\
		&= x + \beta^{-1} \left(1 + \beta x\right)^n - \beta^{-1} + x \left(1 + \beta x\right)^n - x\\
		&= \beta^{-1} \left(1 + \beta x\right) \left(1 + \beta x\right)^n - \beta^{-1}\\
		&= \beta^{-1} \left(1 + \beta x\right)^{n+1} - \beta^{-1}
	\end{align*}
\end{example}

\begin{example}[$\K$-Theory Saga: mod-$p$]\label{k-thy-modp-height}
	By taking the cofiber of the multiplication-by-$p$ map, we get a spectrum $\K/p$, mod-$p$ K-theory, with coefficients $\left(\K/p\right)_* = \mathbb{F}_p \left[\beta^{\pm 1}\right]$.
	It is evident that $F_{\K/p}\left(y,z\right) = y + z + \beta y z$ as well.
	From the result above, it follows that
	$$
	\left[p\right]\left(x\right)
	= \beta^{-1} \left(1 + \beta x\right)^p - \beta^{-1}
	= \beta^{-1} \left(1^p + \beta^p x^p\right) - \beta^{-1}
	= \beta^{p-1} x^p,
	$$
	which shows that the height is exactly $1$.
\end{example}

A formal group law on $E_*$ is the same data as a map from the Lazard ring $L$, so the complex orientation gives a map $L \to E_*$.
In particular, since $\MU$ is complex oriented, there is a canonical map $L \to \MU_*$.

\begin{theorem}[{Quillen, \cite[4.1.6]{Rav86}}]\label{quillen-theorem}
	The canonical map $L \to \MU_*$ is an isomorphism.
\end{theorem}



\subsection{\texorpdfstring{$\BP$}{BP}, Morava K-Theory and Morava E-Theory}

A good principle in homotopy theory (and in many other areas in math) is to study it one prime at a time.
This is possible in homotopy theory due to the arithmetic square.
So, let us fix a prime $p$.
We can form $\MU_{\left(p\right)}$, the $p$-localization of $\MU$.

\begin{theorem}[{\cite[II 15]{Ada}}]
	There exists a unique map of ring spectra $\varepsilon: \MU_{\left(p\right)} \to \MU_{\left(p\right)}$ (depending on the prime $p$) satisfying:
	\begin{itemize}
		\item $\varepsilon$ is an idempotent, i.e. $\varepsilon^2 = \varepsilon$,
		\item $\varepsilon_*$ sends $\left[\CP{n}\right] \in \pi_*\left(\MU_{\left(p\right)}\right)$ to itself if $n = p^r-1$ and to $0$ otherwise.
	\end{itemize}
\end{theorem}

The map $\varepsilon: \MU_{\left(p\right)} \to \MU_{\left(p\right)}$ gives a cohomology operation, for every $X$ we have $\varepsilon^*: \MU_{\left(p\right)}^*\left(X\right) \to \MU_{\left(p\right)}^*\left(X\right)$.
Denote by $\BP_{\left(p\right)}^*\left(X\right)$ the image of $\varepsilon^*$.

\begin{theorem}[{\cite[II 16]{Ada}, \cite[4.1.12]{Rav86}}]
	$\BP$ is a cohomology theory, represented by an associative commutative ring spectrum $\BP$ (depending on the prime $p$), which is a retract of $\MU_{\left(p\right)}$.
	The homotopy groups of $\BP$ are $\BP_* = \ZZ_{\left(p\right)}\left[v_1, v_2, \dotsc\right]$ where $\left|v_n\right| = 2\left(p^n-1\right)$.
\end{theorem}

For convenience we denote $v_0 = p$ (and indeed $\left|v_0\right| = 2\left(p^0-1\right) = 0$).
Since $\BP$ is a retract of $\MU$, it comes with a map $\MU \to \BP$, that is, a complex orientation.

\begin{proposition}[{\cite[4.1.12 combined with A2.1.25 and A2.2.4]{Rav86}}]\label{bp-p-series}
	The $p$-series of the formal group law associated to $\BP$ is
	$\left[p\right]\left(x\right) = \sum v_n x^{p^n}$.
\end{proposition}

\begin{remark}[{\cite[B.5]{Rav92}}]
	The formal group law on $\BP$ has a similar interpretation to that of $\MU$, namely it is the universal $p$-typical formal group law.
	Moreover, the idempotent $\varepsilon: \MU_{\left(p\right)} \to \MU_{\left(p\right)}$ induces an idempotent on homotopy groups, which can be described as the map that takes a formal group law to the canonical $p$-typical formal group law isomorphic to it.
\end{remark}

Once we have $\BP$, we can turn to the definition of Morava K-theory and Morava E-theory.

\begin{definition}
	Let $0 < n < \infty$.
	\emph{Morava K-theory} at height $n$ and prime $p$, denoted by $\K[p,n]$ or $\K[n]$ when the prime is clear from the context, is the spectrum obtained by killing $p=v_0, \dotsc, v_{n-1}, v_{n+1}, \dotsc$ in $\BP$ and inverting $v_n$.
	Therefore $\K[n]_* = \Fp\left[v_n^{\pm 1}\right]$.
	We also define $\K[0] = \HH{\QQ}$ and $\K[\infty] = \HH{\Fp}$.
	Similarly, \emph{Morava E-theory} at height $n$ and prime $p$, denoted by $\E{p,n}$ or $\E{n}$, is the spectrum obtained by killing $v_{n+1}, v_{n+2}, \dotsc$ in $\BP$ and inverting $v_n$.
	Therefore $\E{n}_* = \ZZ_{\left(p\right)}\left[v_1, \dotsc v_{n-1}, v_n^{\pm 1}\right]$.
\end{definition}

Since Morava K-theory and E-theory are obtained from $\BP$ by cofibers and filtered colimits, they are equipped with a map from $\BP$, hence also with a complex orientation.
Then, from \ref{bp-p-series}, we get:

\begin{corollary}\label{k-e-p-series}
	The $p$-series associated to the formal group laws of $\K[n]$ and $\E{n}$ are $v_n x^{p^n}$ and $v_0 x + \dotsc v_n x^{p^n}$ respectively.
	Therefore the height of $\K[n]$ is exactly $n$.
	(Note that by \ref{HR-2}, this is also true for $\K[0]$ and $\K[\infty]$.)
\end{corollary}

We want to describe some properties of Morava K-theory.
To do so we first need some definitions.

\begin{definition}
	Let $R$ be an evenly graded ring.
	$R$ is called a \emph{graded field} if it satisfies one of the equivalent conditions:
	\begin{itemize}
		\item every non-zero homogenus element is invertible,
		\item it is a field $F$ concentrated at degree 0, or $F\left[\beta^{\pm1}\right]$ for $\beta$ of positive even degree.
	\end{itemize}
	An $\mbb{A}_\infty$-ring $E$ is a \emph{field} if $E_*$ is a graded field.
\end{definition}

\begin{example}
	$K\left(n\right)$ is a field for $0 \leq n \leq \infty$.
\end{example}

\begin{proposition}
	A field $E$ has K\"unneth, i.e. $E_*\left(X\otimes Y\right)\cong E_*\left(X\right)\otimes_{E_*}E_*\left(Y\right)$ for any spectra X,Y.
\end{proposition}

\begin{proposition}[{\cite[24]{Lur}}]
	Let $E \neq 0$ be a complex oriented cohomology theory, whose formal group law has height exactly $n$, then $E \otimes \K[n] \neq 0$.
	Let $E$ be a field s.t. $E \otimes \K[n] \neq 0$, then $E$ admits the structure of a $\K[n]$-module.
	(Here $0 \leq n \leq \infty$.)
\end{proposition}

\begin{example}[K-Theory Saga: Morava K-Theory]\label{k-thy-modp-morava}
	As we have seen in \ref{k-thy-modp-height}, mod-$p$ K-theory, $\K/p$, has height exactly $1$ and coefficients $\left(\K/p\right)_* = \mathbb{F}_p \left[\beta^{\pm 1}\right]$.
	It is also known that $\K$ and $\K/p$, are $\mbb{A}_\infty$-ring spectra, from which it follows that $\K/p$ is a field.
	We deduce that $\K/p$ is a $\K[1]$-module.
	Since $\left|\beta\right| = 2$ and $\left|v_1\right| = 2\left(p-1\right)$ it is free of rank $p-1$.
\end{example}

From this we also deduce some form of uniqueness for Morava K-theory:

\begin{corollary}
	Let $E$ be a field with $E_* \cong \Fp\left[v_n^{\pm 1}\right]$, which is also complex oriented of height exactly $n$.
	Then $E \cong \K[n]$ (as spectra).
\end{corollary}



\subsection{\texorpdfstring{$\spec \SS_{\left(p\right)}$}{spec S(p)} and \texorpdfstring{$\spec \SS$}{spec S}}

We are now in a position to describe the topological space $\spec \SS$.
However, it will be easier to state it first for $\spec \SS_{\left(p\right)}$, and then pullback prime ideals.
We know that $\Mod{\SS_{\left(p\right)}} = \Sp_{\left(p\right)}$, and its compact objects are $\Spfin_{\left(p\right)}$, the $p$-localizations of finite spectra.

\begin{proposition}\label{T_E-thick}
	Let $\mcl{T}_E$ be the $E$-acyclics, i.e.
	$$
	\mcl{T}_E
	= \ker E_*
	= \left\{ X \in \Spfin_{\left(p\right)} \mid E_*\left(X\right)=0 \right\}
	= \left\{ X \in \Spfin_{\left(p\right)} \mid X \otimes E=0 \right\}.
	$$
	Then $\mcl{T}_E$ is thick.
\end{proposition}

\begin{proof}
	The proof follows the same lines of \ref{chperf-tk} for the case $\Chperf{R}$.
\end{proof}

\begin{definition}
	We define $\mcl{C}_{p, n} = \mcl{T}_{\K[n]}$, the $\K[n]$-acyclics.
	By the above proposition, it is thick.
	Also, $\mcl{C}_{p, \infty} = \left\{ 0\right\}$, which is trivially thick.
	When the prime is clear from the context, we write $\mcl{C}_n$ in place of $\mcl{C}_{p, n}$.
\end{definition}

\begin{proposition}[{\cite[26]{Lur}}]
	For $X \in \Spfin_{\left(p\right)}$, if $\K[n]_*\left(X\right) = 0$ then $\K[n-1]_*\left(X\right) = 0$.
\end{proposition}

\begin{definition}
	We say that a spectrum $X \in \Spfin_{\left(p\right)}$ is of \emph{type} $n$ (possibly $\infty$) if its first non-zero Morava K-theory homology is $\K[n]$.
\end{definition}

\begin{corollary}
	$\mcl{C}_n$ is the full subcategory of finite $p$-local spectra of type $> n$, that is $\mcl{C}_n = \left\{ X \in \Spfin_{\left(p\right)} \mid \forall m \leq n: \K[m]_*\left(X\right) = 0 \right\}$.
	Therefore, we also conclude that $\mcl{C}_{n+1} \subseteq \mcl{C}_n$.
\end{corollary}

\begin{proposition}
	The inclusions $\mcl{C}_{n+1} \subset \mcl{C}_n$ are proper.
\end{proposition}

\begin{remark}
	The modern proof of this result relies on the periodicity theorem \cite[1.5.4]{Rav92}.
	Using it, we can construct generalized Moore spectra, which give an example of spectra of type $n$ for every $n$.
\end{remark}

\begin{proposition}
	If $X \in \Spfin_{\left(p\right)}$ is not contractible, then it is of finite type.
	Therefore $\bigcap_{n < \infty} \mcl{C}_n = \left\{0\right\} = \mcl{C}_\infty$.
\end{proposition}

\begin{proof}
	Let $X$ be non-contractible.
	Then $\HH{\ZZ}_*\left(X\right) \neq 0$.
	Let $m$ be the first non-zero degree.
	Using the universal coefficient theorem and the fact that the spectrum is $p$-local we get that $\left(\HH{\Fp}\right)_m\left(X\right) \neq 0$, thus $\left(\HH{\Fp}\right)_*\left(X\right) \neq 0$.
	Since $X$ is finite, $\left(\HH{\Fp}\right)_*\left(X\right)$ is bounded.
	The Atiyah-Hirzebruch spectral sequence for $X$ with cohomology $\K[n]$ has $E^2$-page given by
	$
	E_{t,s}^2
	= 
	H_t\left(X; \K[n]_s\right)
	$.
	Since $\K[n]_s = \Fp$ for $s = 0 \mod 2\left(p^n-1\right)$ and $0$ otherwise,
	we see that the rows $s = 0 \mod 2\left(p^n-1\right)$ are $\left(\HH{\Fp}\right)_*\left(X\right)$, and the others are $0$.
	Therefore if we take $n$ such that the period $2\left(p^n-1\right)$ is larger then the bound on $\left(\HH{\Fp}\right)_*\left(X\right)$, then all differentials have either source or target $0$.
	Thus, the spectral sequence collapses at the $E^2$-page, and since $\left(\HH{\Fp}\right)_*\left(X\right) \neq 0$, we get that $\K[n]\left(X\right) \neq 0$, i.e. $X$ has type $\leq n$.
\end{proof}

\begin{proposition}
	$\mcl{C}_n$ is a prime ideal.
\end{proposition}

\begin{proof}
	Recall from \ref{T_E-thick} that we already know that it is thick.
	For $X,Y \in \Spfin_{\left(p\right)}$, by K\"unneth we have 
	$$
	\K[n]_*\left(X \otimes  Y\right)
	= \K[n]_*\left(X\right) \otimes \K[n]_*\left(Y\right).
	$$
	Assume that $X \in \mcl{C}_n$, that is $\K[n]_*\left(X\right) = 0$.
	It follows that $\K[n]_*\left(X \otimes  Y\right) = 0$, i.e. $X \otimes Y \in \mcl{C}_n$, so $\mcl{C}_n$ is an ideal.
	Assume that $X \otimes Y\in \mcl{C}_n$, that is $\K[n]_*\left(X \otimes  Y\right) = 0$, therefore one of the terms in the RHS of the equation must vanish (since they are graded vector spaces), so $\mcl{C}_n$ is a prime ideal.
\end{proof}

\begin{theorem}[{Thick Subcategory Theorem \cite[theorem 7]{HS}}]\label{thick-subcategory-thm}
	If $\mcl{T}$ is a proper thick subcategory of $\Spfin_{\left(p\right)}$, then $\mcl{T} = \mcl{C}_n$ for some $0 \leq n \leq \infty$.
\end{theorem}

\begin{remark}
	The proof relies on a major theorem called the Nilpotence Theorem.
\end{remark}

\begin{corollary}
	$\spec \SS_{\left(p\right)} = \left\{ \mcl{C}_0, \mcl{C}_1, \dotsc, \mcl{C}_\infty \right\}$,
	and the closed subsets in the topology are chains
	$\left\{ \mcl{C}_k, \mcl{C}_{k+1}, \dotsc, \mcl{C}_\infty \right\}$
	for some $0 \leq k \leq \infty$.
\end{corollary}

\begin{proof}
	Follows immediately from the previous results.
\end{proof}

We now want to describe $\spec \SS$.
Note that the $p$-localization functor $L_{\left(p\right)}$ is a Bousfield localization.
As such, it is left (its right adjoint is the inclusion), and in particular preserves cofibers.
It is also reduced, i.e. sends $0$ to $0$.
Now, $L_{\left(p\right)}$ is smashing, that is $L_{\left(p\right)} X = X \otimes \SS_{\left(p\right)}$, so it is also symmetric monoidal.
As we have seen in \ref{primes-pullback}, under these conditions we can pullback primes.
Since $L_{\left(p\right)}$ is smashing and $\K\left(p,n\right)$ is $p$-local for every $0 \leq n \leq \infty$, we have that $\K[n]_*\left(X_{\left(p\right)}\right) = \K[n]_*\left(X\right)$.
Therefore
$$
\mcl{P}_{p, n}
= L_{\left(p\right)}^* \mcl{C}_{p, n}
= \left\{
	X \in \Spfin
	\mid
	\K[p,n]_*\left(X\right) = 0
\right\}
$$
and
$$
\mcl{P}_{p, \infty}
= L_{\left(p\right)}^* \mcl{C}_{p, \infty}
= \left\{
	X \in \Spfin
	\mid
	X_{\left(p\right)} = 0
\right\}
$$
are prime ideals.
Moreover, by definition $\K[p,0] = \HH{\QQ}$, which implies that
$
\mcl{P}_{p, 0}
= \left\{
	X \in \Spfin
	\mid
	{\HH{\QQ}}_*\left(X\right) = 0
\right\}
$.
We see that $\mcl{P}_{p, 0}$ is independent of $p$, and we denote it by $\Spfintor$.
Again, by \ref{primes-pullback}, we see that any chain $\left\{ \mcl{P}_{p, k}, \mcl{P}_{p, k+1}, \dotsc, \mcl{P}_{p, \infty} \right\}$ for some $p$ and $0 \leq k \leq \infty$, is closed, hence any finite union of such chain is also closed.
We then have the following theorem, which says that these are all of the prime ideals and all of the closed subsets.

\begin{theorem}[{Thick Subcategory Theorem, \cite[9.5]{BalSSS}}]
	$
	\spec \SS
	= \left\{\Spfintor\right\}
	\bigcup \left(\bigcup_p \left\{\mcl{P}_{p, 1}, \dotsc, \mcl{P}_{p, \infty}\right\}\right)
	$,
	and the closed subsets in the topology are finite unions of chains
	$\left\{ \mcl{P}_{p, k}, \mcl{P}_{p, k+1}, \dotsc, \mcl{P}_{p, \infty} \right\}$
	for some $0 \leq k \leq \infty$ (i.e. they may include $\Spfintor$).
\end{theorem}

\begin{remark}
	The following diagram shows the structure of $\spec \SS$.
	Each $\mcl{P}_{p,n}$, and $\Spfintor$, is a point.
	A line represents that the closure of the point at the bottom contains the point at the top.
	$$
		\begin{tikzcd}[row sep=small, column sep=scriptsize]	
		\mcl{P}_{2, \infty} & \mcl{P}_{3, \infty} & \cdots & \mcl{P}_{p, \infty} & \cdots\\
		\vdots \arrow[dash]{u}{} & \vdots \arrow[dash]{u}{} &  & \vdots \arrow[dash]{u}{} & \\
		\mcl{P}_{2, n} \arrow[dash]{u}{} & \mcl{P}_{3, n} \arrow[dash]{u}{} & \cdots & \mcl{P}_{p, n} \arrow[dash]{u}{} & \cdots\\
		\vdots \arrow[dash]{u}{} & \vdots \arrow[dash]{u}{} &  & \vdots \arrow[dash]{u}{} & \\
		\mcl{P}_{2, 2} \arrow[dash]{u}{} & \mcl{P}_{3, 2} \arrow[dash]{u}{} & \cdots & \mcl{P}_{p, 2} \arrow[dash]{u}{} & \cdots\\
		\mcl{P}_{2, 1} \arrow[dash]{u}{} & \mcl{P}_{3, 1} \arrow[dash]{u}{} & \cdots & \mcl{P}_{p, n} \arrow[dash]{u}{} & \cdots\\
		& & \Spfintor \arrow[dash]{ull}{} \arrow[dash]{ul}{} \arrow[dash]{ur}{}
	\end{tikzcd}
	$$
\end{remark}

\begin{remark}
	Thick subcategories are interesting for another reason, unrelated to the Balmer spectrum point of view, namely they give a very powerful proof method.
	Say we have a property that is satisfied by $0$, and is closed under cofibers and retracts.
	It follows that the collection of objects that satisfy it is thick.
	Then, for example, by the thick subcategory theorem \ref{thick-subcategory-thm}, it is enough to find one object in $\mcl{C}_n \setminus \mcl{C}_{n+1}$ that satisfies the property, to show that all objects in $\mcl{C}_n$ satisfy it.
\end{remark}



\subsection{The Stacky Point of View and the Relationship Between Morava K-Theory and Morava E-Theory}

First we will describe, without being precise, another point of view on what chromatic homotopy theory is about.

There is a stack of formal groups with strict isomorphisms, denoted by $\Mfgs$.
It can be described as the stack that sends a ring $R$ to the groupoid of formal group laws, with strict isomorphisms between them.
Quillen theorem \ref{quillen-theorem} tells us that $\MU_*$ is the Lazard ring, that is the universal ring that carries the universal formal group law.
This theorem has a second part, which says that $\left(\MU \otimes \MU\right)_*$ is the universal ring that carries two formal group laws and a strict isomorphism between them.
Therefore, $\Mfgs$ is represented by $\left(\MU_*, \left(\MU \otimes \MU\right)_*\right)$.

The geometric points of the stack $\Mfgs$ are precisely the same as $\spec \SS$, that is because for an algebraically closed field of characteristic $0$ there is a unique (up to isomorphism) formal group law which is of height $0$ namely the additive formal group law, and for characteristic $p$ there is a unique (up to isomorphism) formal group law of each height $1 \leq n \leq \infty$.

For a spectrum $X$, $\MU_*\left(X\right)$ is a $\left(\MU_*, \left(\MU \otimes \MU\right)_*\right)$-comodule, which is the same as a sheaf over $\Mfgs$.
From this point of view, chromatic homotopy theory lets us study a spectrum by decomposing it over the stack $\Mfgs$.

We can restrict ourselves to the stack only over $p$-local rings, $\Mfgsp$, which is then represented by $\left(\left(\MU_{\left(p\right)}\right)_*, \left(\MU_{\left(p\right)} \otimes \MU_{\left(p\right)}\right)_*\right)$.
Similarly to $\MU$, $\BP$ is universal ring with the universal $p$-typical formal group law, and $\left(\BP \otimes \BP\right)_*$ is the universal ring with two $p$-typical formal group laws and a strict isomorphism between them.
Since every formal group law is isomorphic to a unique $p$-typical one, we know that the stack $\Mfgsp$ is also represented by $\left(\BP_*, \left(\BP \otimes \BP\right)_*\right)$.

It is now reasonable that $\K[n]$, obtained from $\BP$ by killing the $v_m$'s for $m \neq n$ and inverting $v_n$, sees the $n$-th level, and that $\E{n}$ obtained in the same way but only killing $v_m$ for $m > n$, sees the levels $\leq n$.
Following this point of view, the following theorem can be proven:

\begin{theorem}[{\cite[23, proposition 2]{Lur}}]
	$\E{n}$ and $\K[0] \oplus \cdots \oplus \K[n]$ are Bousfield equivalent.
	That is, they have the same acyclics, locals, and their localization functors are the same.
\end{theorem}

\todo{chromatic square and chromatic tower, maybe another subsection?}



\subsection{Landweber Exact Functor Theorem}

As we have seen, a complex orientation on a cohomology theory, which is given by a map $\MU \to E$, has an associated formal group law, which is given by the map $L = \MU_* \to E_*$.
Note that this formal group law is of degree $-2$, by virtue of the grading on $L = \MU_*$.
One can ask whether the converse is true, namely given a graded ring $R$ and a formal group law $F$ of degree $-2$ given by $L \to R$, is there a complex oriented cohomology theory whose coefficients are $R$ and the associated formal group law is $F$.

A strategy is to define $\left(E_{R,F}\right)_*\left(X\right) = \MU_*\left(X\right) \otimes_{\MU_*} R$.
Unfortunately, this is not always a homology theory.
However there is a condition that one can check, which guarantees that it is.

\begin{definition}
	$L \to R$ is called \emph{Landweber flat} if for every prime $p$, the image of the sequence $p = v_0, v_1, v_2, \dotsc$ in $R$, which are the coefficients of the $p$-series, is regular.
	That is, for each $p$ and $n \geq 0$, $v_n$ is not a zero divisor in $R/\left(v_0, v_1, \dotsc, v_{n-1}\right)$.
\end{definition}

\begin{remark}\label{landweber-flat-helper}
	If $p$ is invertible in $R$, then $p$ is invertible, and $R/p$ is already $0$, so we don't need to check $v_1, v_2, \dotsc$.
\end{remark}

\begin{theorem}[{Landweber Exact Functor Theorem (LEFT), \cite[15, 16]{Lur}}]\label{LEFT}
	If $L \to R$ is Landweber flat, then $E_{R,F}$ defined above is a homology theory.
	Moreover, there are no phantom maps between such spectra, so $E_{R,F}$ is represented by a spectrum.
	This spectrum is complex oriented, has coefficients $R$ and associated formal group law $F$.
\end{theorem}

\begin{example}
	Morava E-theory is Landweber flat, since by \ref{k-e-p-series}, the $p$-series has coefficients $p = v_0, v_1, \dotsc, v_n$.
	$p$ is not a zero divisor in $\E{n}_* = \ZZ_{\left(p\right)}\left[v_1, \dotsc v_{n-1}, v_n^{\pm 1}\right]$.
	Then $v_i$ is not a zero divisor in $\E{n}_*/\left(p, v_1, \dotsc, v_{i-1}\right) \cong \Fp\left[v_i, \dotsc v_{n-1}, v_n^{\pm 1}\right]$.
	After $v_n$ the ring becomes $0$ and we are done.
	For other primes, by \ref{landweber-flat-helper} we are done.
\end{example}

\begin{example}
	Morava K-theory $\K[n]$ for $n > 0$ is not Landweber flat since $p = v_0$ is $0$ in $\K[n]_* = \Fp\left[v_n^{\pm 1}\right]$.
\end{example}

\begin{example}
	$\HH{\ZZ}$ is not Landweber flat since although $p = v_0$ is invertible, as we have seen in \ref{HR-2} the $p$-series is $px$, so $v_1$ is $0$ in $\ZZ/p = \Fp$.
\end{example}

We can also ask the following question: given complex oriented cohomology theory $\MU \to E$, such that $L \to E_*$ is Landweber flat, is $E_{R,F}$ equivalent to $E$?
The answer is yes, at least in some cases.

\begin{theorem}\label{LEFT-even}
	Let $E$ be as above, which is also evenly graded (i.e. $E_*$ is an evenly graded ring), then there is an equivalence $E_{R,F} \to E$.
\end{theorem}

\begin{proof}
	This is a slight variation on \cite[18, proposition 11]{Lur}.
	First note that for every spectrum $X$ we have $\MU \otimes X \to E \otimes X$, which induces $\MU_*\left(X\right) \to E_*\left(X\right)$, a map of $\MU_*$-modules.
	Moreover, since $E_* \to E_*\left(X\right)$ is a map of $E_*$-module, the map $\MU_* \to E_*$ makes it a map of $\MU_*$-modules.
	Together this gives a map $\left(E_{R,F}\right)_*\left(X\right) = \MU_*\left(X\right) \otimes_{\MU_*} E_* \to E_*\left(X\right)$.
	This map is a map of homology theories. \todo{should I explain why?}
	By \cite[17, theorem 6]{Lur} \todo{does it follow?}, this map lifts to a map of spectra $E_{R,F} \to E$.
	Since by construction when $X = \SS$ the map above is $E_* \to E_*$ which is an isomorphism, we see that the map $E_{R,F} \to E$ is an equivalence.
\end{proof}

\begin{example}[K-Theory Saga: Landweber Flatness]\label{k-thy-comp-left}
	We return to complex K-theory, from \ref{k-thy-oriented} and \ref{k-thy-fgl}.
	We can take the completion at the element $p \in \K_*$, which gives the spectrum $\K_p^\wedge$.
	This spectrum has coefficients
	$
	\left(\K_p^\wedge\right)_*
	= \left(\K_*\right)_p^\wedge
	= \left(\ZZ\left[\beta^{\pm 1}\right]\right)_p^\wedge
	= \ZZ_p\left[\beta^{\pm 1}\right]
	$.
	The formal group law, as we have seen, is given by $F_{\K_p^\wedge}\left(y,z\right) = y + z + \beta y z$.
	We claim that $F_{\K_p^\wedge}/\K_p^\wedge$ is Landweber flat.
	Clearly $p = v_0$ is not a zero divisor in $\ZZ_p\left[\beta^{\pm 1}\right]$.
	As we have seen in \ref{k-thy-modp-height}, mod-$p$ the $p$-series is $\beta^{p-1} x^p$, so that $v_1 = \beta^{p-1}$ which is not a zero divisor $\Fp\left[\beta^{\pm 1}\right]$.
	Modulo $v_1$ the ring is already $0$, and we are done.
	For other primes, by \ref{landweber-flat-helper} we are done.
	Therefore, by \ref{LEFT-even} we get that $\K_p^\wedge \cong E_{\K_p^\wedge, F_{\K_p^\wedge}}$.
\end{example}



\subsection{Lubin-Tate Deformation Theory}\label{LT-def}

The Morava E-theory we have considered until now $\E{n}$, also called Johnson-Wilson spectrum was constructed from $\BP$.
As we noted, it is Landweber flat, which indicates that there is another approach to constructing it.
Indeed there is a way to construct a related spectrum, which will be called the Lubin-Tate spectrum.

To that end, we first define the category $\mrm{CompRing}$ as the category of complete local rings.
The objects are complete local rings $\left(R, \mfk{m}\right)$, we also denote by $\pi: R \to R/\mfk{m}$ the projection.
Morphisms $\varphi: \left(R, \mfk{m}\right) \to \left(S, \mfk{n}\right)$ are local homomorphisms, i.e. a homomorphism $\varphi: R \to S$ s.t. $\varphi\left(\mfk{m}\right) \subseteq \mfk{n}$.
In particular it induces a homomorphism $\varphi/\mfk{m}: R/\mfk{m} \to S/\mfk{n}$, which satisfies $\varphi/\mfk{m} \circ \pi_R = \pi_S \circ \varphi$.

We fix $k$ be a perfect field of characteristic $p$ (i.e. the Frobenius is an isomorphism), and $\Gamma$ a formal group law over $k$ of height $n < \infty$.
Lubin and Tate \cite{LT} considered a moduli problem associated to $\Gamma/k$, described by a functor $\Def: \mrm{CompRing} \to \mrm{Grpds}$.

\begin{definition}
	Let $\left(R, \mfk{m}\right)$ be a complete local ring and denote by $\pi: R \to R/\mfk{m}$ the quotient.
	A \emph{deformation} of $\Gamma/k$ to $\left(R, \mfk{m}\right)$, is $\left(G, i\right)$, where $G$ is a formal group law over $R$, $i: k \to R/\mfk{m}$ is a homomorphism of fields, such that $i^* \Gamma = \pi^* G$.
	A \emph{$\star$-isomorphism} between two deformations to $\left(R, \mfk{m}\right)$, $f: \left(G_1, i_1\right) \to \left(G_2, i_2\right)$, is defined only when $i_1 = i_2$, and consists of an isomorphism $f: G_1 \to G_2$, such that $\pi^* f: i^*\Gamma = \pi^* G_1 \to \pi^* G_2 \to i^*\Gamma$ is the identity, i.e. $f\left(x\right) = x \mod \mfk{m}$.
	These assemble to a groupoid $\Def[R, \mfk{m}]$, whose objects are deformations to $\left(R, \mfk{m}\right)$, and morphisms are $\star$-isomorphisms.
\end{definition}

\begin{remark}
	$\Def[R, \mfk{m}]$ can be seen as the pullback of the groupoids $\FGL{R}$ and $\coprod_{i: k \to R/\mfk{m}} \left\{\Gamma\right\}$ over $\FGL{R/\mfk{m}}$ (where the maps are $G \mapsto q^* G$ and $i \mapsto i^* \Gamma$ respectively).
\end{remark}

\begin{proposition}[/definition]
	The construction $\Def[R, \mfk{m}]$ is functorial.
\end{proposition}

\begin{proof}
	Let $\varphi: \left(R, \mfk{m}\right) \to \left(S, \mfk{n}\right)$ be a local homomorphism.
	
	For a deformation $\left(G, i\right)$ to $\left(R, \mfk{m}\right)$, we define
	$
	\Def[\varphi]\left(G, i\right)
	= \left(\varphi^* G, \varphi/\mfk{m} \circ i\right)
	$.
	Note that $\varphi^* G$ is a formal group law over $S$, and $\varphi/\mfk{m} \circ i: k \to R/\mfk{m} \to S/\mfk{n}$ is a homomorphism.
	Moreover,
	$
	\left(\varphi/\mfk{m} \circ i\right)^* \Gamma
	= \left(\varphi/\mfk{m}\right)^* i^* \Gamma
	= \left(\varphi/\mfk{m}\right)^* \pi_R^* G
	= \left(\varphi/\mfk{m} \circ \pi_R\right)^* G
	= \left(\pi_S \circ \varphi\right)^* G
	= \pi_S^* \varphi^* G
	$,
	which shows that $\Def[\varphi]\left(G, i\right)$ is a deformation to $\left(S, \mfk{n}\right)$.
	
	For a $\star$-isomorphism $f: \left(G_1, i_1\right) \to \left(G_2, i_2\right)$, which is the data of an isomorphism $f: G_1 \to G_2$ such that $\pi_R^* f = \id[i^*\Gamma]$ is the identity, we need to define a $\star$-isomorphism $\Def[\varphi]\left(G, i_1\right) \to \Def[\varphi]\left(G, i_2\right)$.
	Take it to be the isomorphism $\varphi^* f: \varphi^* G_1 \to \varphi^* G_2$, which satisfies
	$
	\pi_S^* \varphi^* f
	= \left(\varphi/\mfk{m}\right)^* \pi_R^* f
	= \left(\varphi/\mfk{m}\right)^* \id[i^*\Gamma]
	= \id[\left(\varphi/\mfk{m}\right)^* i^*\Gamma]
	= \id[\left(\varphi/\mfk{m} \circ i \right)^*\Gamma]
	$.
	The identity $\id[G]: \left(G, i\right) \to \left(G, i\right)$ is clearly sent to $\id[\varphi^* G]$, and compositions are sent to compositions.
	
	This shows that $\Def[\varphi]: \Def[R, \mfk{m}] \to \Def[S, \mfk{n}]$ is indeed a functor.
	Moreover, it is clear that $\Def[{\id[R]}]$ is the identity and compositions are sent to compositions, which shows that $\Def: \mrm{CompRing} \to \mrm{Grpds}$ is indeed a functor.
\end{proof}

\begin{remark}
	We recall quickly that the Witt vectors $\Wk$ is a ring of characteristic $0$, with maximal ideal $\left(p\right)$, and residue field $\Wk/p \cong k$.
	For example, $\WW{\Fp} = \ZZ_p$.
\end{remark}

\begin{theorem}[{\cite[{4.4, 5.10}]{Rez}, originally due to \cite{LT}}]
	The functor $\Def$ lands in discrete groupoids (i.e. $\Def[R, \mfk{m}]$ has $0$ or $1$ morphisms between objects).
	Furthermore the functor $\Def$ is co-represented, that is there exists a \emph{universal deformation}, and the complete local ring can be chosen (non-canonically) to be $\Wkui$.
\end{theorem}

Let us unravel what that means.
First note that the quotient of $\Wkui$ by the maximal ideal $\left(p, u_1, \dotsc, u_{n-1}\right)$ is $k$.
The universal deformation can be chosen such that the formal group law over it $\Gamma_U$ over $\Wkui$ satisfies $\pi^* \Gamma_U$ is $\Gamma$.
The universality means that for $\left(R, \mfk{m}\right)$, the assignment
$$
\hom_{\mrm{CompRing}}\left(\Wkui, R\right) \to \Def[R, \mfk{m}], \quad
\varphi \mapsto \varphi^* \Gamma_U
$$
is an equivalence.

Now, we can form the graded ring $\Wkuiu$ where $\left|u\right| = 2$.
We can define the formal group law $\left(u \Gamma_U\right) \left(x, y\right) = u^{-1} \Gamma_U \left(u y, u z\right)$, which is of degree $-2$, thus we get a map $L \to \Wkuiu$.

\begin{proposition}[{\cite[{6.9}]{Rez}}]
	$L \to \Wkuiu$ is Landweber flat.
\end{proposition}

Using LEFT \ref{LEFT}, we immediately get:

\begin{corollary}\label{LT-spectrum}
	There is a complex oriented cohomology theory $\E{k, \Gamma} = E_{\Wkuiu, u \Gamma_U}$, called \emph{Lubin-Tate spectrum}, with coefficients $\E{k, \Gamma}_* = \Wkuiu$ and associated formal group law $u \Gamma_U$.
\end{corollary}

\begin{example}[K-Theory Saga: Lubin-Tate]\label{k-thy-comp-defo}
	We continue the complex K-theory saga from \ref{k-thy-comp-left}.
	Take the field $k = \Fp$ and the formal group law $\Gamma\left(y, z\right) = y + z + y z$, of height $n = 1$.
	By the above construction, the ring of the universal deformation is $\WW{\Fp} = \ZZ_p$.
	The universal formal group law of the universal deformation can be taken to be $\Gamma_U \left(y, z\right) = y + z + y z$ (this follows from the proof at \cite[5.10]{Rez}, since here $n=1$ so there are no $u_i$'s).
	We look at the ring $\ZZ_p\left[u^{\pm1}\right]$, and at the formal group law over it
	$
	\left(u\Gamma_U\right) \left(y, z\right)
	= u^{-1} \left(u y + u z + u^2 y z\right)
	= y + z + u y z
	$.
	It is then clear that the isomorphism $\ZZ_p\left[u^{\pm1}\right] \to \ZZ_p\left[\beta^{\pm1}\right]$, sends $u\Gamma_U$ to $F_{\K_p^\wedge}$.
	It follows by \ref{LEFT-even} that
	$
	\K_p^\wedge
	\cong E_{\K_p^\wedge, F_{\K_p^\wedge}}
	\cong \E{\Fp, \Gamma}
	$,
	i.e. $p$-complete K-theory $\K_p^\wedge$ is a Lubin-Tate spectrum at height $1$.
\end{example}

This concludes the construction of the Lubin-Tate variant of Morava E-theory.
We can compare the Lubin-Tate spectrum with the previous one, the Johnson-Wilson spectrum.

\begin{proposition}[{\cite[23, example 1]{Lur}}]
	The Lubin-Tate spectrum $\E{k, \Gamma}$ and Johnson-Wilson spectrum $\E{n}$ are Bousfield equivalent.
\end{proposition}

	\section{Atiyah-Segal}

We now leave the realm of chromatic homotopy theory.
One aspect of algebraic topology is to try to capture properties of spaces using algebraic invariants.
One of the most fruitful such invariants is complex K-theory, denoted $\K$, and one of the most important spaces in homotopy theory is $\BG$, so it is natural to ask for a description of $\K\left(\BG\right)$ (by Bott periodicity, we will consider only $\K = \K^0$).
Atiyah and Segal \cite{AS} gave a description of this, and more, in the case that $G$ is a compact Lie group, in terms of representations.

From now we fix a compact Lie group $G$.
Also, a representation means a finite dimensional unitary representation.
We should also note that beyond this part, we will be mostly interested in finite groups.



\subsection{The Atiyah-Segal Theorem}

We denote by $\RG$ the \emph{representation ring} of $G$, that is the collection of virtual representations of $G$ (which can be written as a formal difference $V - U$) up to isomorphism, where the addition is given by direct sum and the product is given by tensor product.
This is an augmented ring $\varepsilon: \RG \to \ZZ$ by the virtual dimension (i.e. $\varepsilon\left(V-U\right) = \dim\left(V-U\right) = \dim V - \dim U$).
The \emph{augmentation ideal} is $I = \ker \varepsilon = \left\{ V - U \in \RG \mid 0 = \dim(V-U) \right\}$.

Atiyah and Segal showed that one can describe $\K\left(\BG\right)$ in these terms, namely, it is the completion of $\RG$ at the ideal $I$:

\begin{theorem}[{\cite{AS}}]\label{AS-private}
	$\K\left(\BG\right) \cong \RGI$.
\end{theorem}

We will not prove the theorem, but we will indicate some of the key ingredients.

First of all, to show that objects are isomorphic, we need a map.
Before giving the map actually used in the proof, we describe an easier way to see where this map comes from.
Recall that $\K\left(X\right) \cong \left[X, \BU \times \ZZ\right]$.
The data of a representation of $G$ is the same thing as a homomorphism $G \to \UU[n]$.
Since $\BB$ is a functor, we get a map $\BG \to \BU[n]$, and by composing with the injection $\BU[n] \cong \BU[n] \times \left\{n\right\} \to \BU \times \ZZ$, we get a map $\BG \to \BU \times \ZZ$, that is, an element of $\K\left(\BG\right)$.
Therefore we get a map $\RG \to \K\left(\BG\right)$.
The theorem shows that it is a ring homomorphism which exhibits $\K\left(\BG\right)$ as the completion of $\RG$ at $I$.

There is an alternative description of this map.
In \cite{Seg}, Segal described a variant of$\K$ theory, called equivariant K-theory $\KG$.
This variant assigns to a $G$-space the ring of virtual $G$-bundles, that is, bundles equipped with an action of $G$ which is compatible with the action on the base $G$-space.
Note that $\K_G$ is no longer homotopy invariant, since it also takes into account the action of $G$.
First we note the following:

\begin{proposition}
	$\KG\left(*\right) = \RG$ (where $*$ is equipped with a trivial $G$-action).
\end{proposition}

\begin{proof}
	This follows from the definitions, since a vector bundle over a point is just a vector space, and it is equipped with a $G$-action over the point, which is just a $G$ representation.
\end{proof}

For any $G$-space $X$, the projection map $\mrm{pr}: X \to X/G$ allows us to pullback vector bundles on $X/G$ to $G$-bundles on $X$.
In other words, it induces a map $\mrm{pr}^*: \K\left(X/G\right) \to \KG\left(X\right)$.

\begin{proposition}[{\cite[2.1]{Seg}}]
	Suppose the action of $G$ on $X$ is \emph{free}. Then $\mrm{pr}^*$ admits an inverse, given by taking a bundle $E \to X$ to $E/G \to X/G$.
	In particular, $\K\left(X/G\right) \cong \KG\left(X\right)$.
\end{proposition}

Now, we have a map of $G$-spaces given by $\EG \to *$.
By the above we get:
$$
\RG
\cong \KG\left(*\right)
\to \KG\left(\EG\right)
\cong \K\left(\EG/G\right)
= \K\left(\BG\right)
$$
It can be shown that this is the same map $\RG \to \K\left(\BG\right)$ described before, which exhibits $\K\left(\BG\right)$ as the $I$-completion of $\RG$.
Atiyah and Segal use this map and variants to prove the theorem.

Here is a sketch of the proof given by Atiyah and Segal.
First of all, we note that the theorem is proven for the entire $\K^*$ rather than just for $\K = \K^0$.
Also note that $\RgG = \KG^*\left(*\right)$ is a $2$-periodic version of the representation ring (because $\KG^*$ also satisfies Bott periodicity).
We have the corresponding $2$-periodic version of the augmentation ideal, which we denote by $I^*$.
They use the Milnor join construction $\EG_n = \underbrace{G * \cdots * G}_{n \text{ times}}$ and $\BG_n = \EG_n/G$, which has the property that $\colim \EG_n \to \colim \BG_n$ is a model for $\EG \to \BG$.
Then, for any compact $G$-space $X$ there is a similar map to the map above: using $X \times \EG_n \to X$ we get a map $\KG^*\left(X\right) \to \KG^*\left(X \times \EG_n\right)$.
All of these are $\RgG$-modules, and Atiyah and Segal show that this map factors through the quotient by $\left(I^*\right)^n$, to give a map $\KG^*\left(X\right)/\left(I^*\right)^n \to \KG^*\left(X \times \EG_n\right)$.
The two sides assemble into pro-rings, and the maps assemble to a map between the pro-rings:
$$
\left\{\KG^*\left(X\right)/\left(I^*\right)^n\right\}_n
\to \left\{\KG^*\left(X \times \EG_n\right)\right\}_n
$$
What they actually prove is the strong form:

\begin{theorem}[{\cite{AS}}]\label{AS-full}
	If $\KG^*\left(X\right)$ is finite over $\RgG$, then the above map of pro-rings is an isomorphism.
\end{theorem}

Their proof has another interesting aspect.
Although it is a statement about the $\K_G$ of some class of $G$-spaces, for one specific group $G$, their proof involves the $\K_G$'s of several groups.
In particular, to prove the result for example for a finite group, their proof involves more general compact Lie groups.
The proof consists of four steps.
In every step we show the theorem holds for a more general type of group:
\begin{itemize}
	\item $G = \UU[1]$ (circle group),
	\item $G = \UU[1]^n$ (torus group),
	\item $G = \UU[n]$,
	\item $G$ a general compact Lie group (this step is proven by $G$ embedding in $\UU[n]$).
\end{itemize}

We note that the first formulation of the Atiyah-Segal theorem \ref{AS-private}, is indeed a private case of the second formulation \ref{AS-full}.
Take the case $X = *$.
By definition, $\KG^*\left(*\right)$ is finite over $\RgG = \KG^*\left(*\right)$, so \ref{AS-full} holds and we have an isomorphism of pro-rings $\left\{\KG^*\left(*\right)/\left(I^*\right)^n\right\}_n \to \left\{\KG^*\left(\EG_n\right)\right\}_n$.
In particular, after computing the limits $\lim: \pro \Ring \to \Ring$, we get an isomorphism of rings $\KG^*\left(*\right)_{I^*}^\wedge \xrightarrow{\sim} \KG^*\left(\EG\right)$.
Taking only the $0$-th cohomology gives the desired isomorphism:
$$
\RG_I^\wedge
\cong \KG\left(*\right)_I^\wedge
\xrightarrow{\sim} \KG\left(\EG\right)
\cong \K\left(\EG/G\right)
= \K\left(\BG\right)
$$




\subsection{Examples}

We compute a few examples in detail, to make the isomorphism more vivid.

\subsubsection{\texorpdfstring{$\UU[1]$}{U(1)}, the Circle Group}

Take $G = \UU[1]$, the circle group.
It is known that the irreducible representations are of dimension $1$ and labeled by an integer $m \in \ZZ$, i.e. they are homomorphisms $\rho_m = \UU[1] \to \UU[1]$ given by $\rho_m\left(e^{i \theta}\right) = e^{m i \theta}$.
In particular, $\rho_0 = 1$ is the trivial representation.
It is then clear that for $m \geq 0$, $\rho_1^{\otimes_m} = \rho_m$ and $\rho_{-1}^{\otimes_{-m}} = \rho_{-m}$.
Therefore the representation ring generated under (virtual) direct sums and tensor products by $\rho_1$ and $\rho_{-1}$.
Moreover, $\rho_1 \otimes \rho_{-1} = 1$.
Therefore we conclude that $\RG[\UU[1]] = \ZZ\left[\rho, \rho^{-1}\right]$.

The augmentation map is the homomorphism $\varepsilon: \RG[\UU[1]] \to \ZZ$ which sends $1,\rho$ and $\rho^{-1}$ to $1$.
Recall that the augmentation ideal is $I = \ker \varepsilon$.
We set $t = \rho-1$, which clearly belongs to $I$.
We can also write then $\RG[\UU[1]] = \ZZ\left[t, \left(1+t\right)^{-1}\right]$.
Note that $\varepsilon$ factors to a map $\RG[\UU[1]]/\left(t\right) \to \ZZ$, which is already an isomorphism, so by the first isomorphism theorem indeed $I = \left(t\right)$.

We compute the completion $\RGI[{\UU[1]}]$.
Note that in $\ZZ\left[t\right]/t^n$, $1+t$ is already invertible.
The reason is that the formal power series for the inverse is finite since large enough powers of $t$ are zero, $\frac{1}{1-\left(-t\right)} = \sum_{m=0}^{n-1} \left(-t\right)^m$ is an inverse.
Therefore we see that $\RG[\UU[1]]/I^n \cong \ZZ\left[t\right]/t^n$, and clearly the maps the in the limit diagram send $t$ to $t$.
We get that $\RGI[{\UU[1]}] = \lim \ZZ\left[t\right]/t^n \cong \ZZ\formal{t}$.

In \cite[proposition 2.24]{VB}, it is shown that $\K\left(\CP{n}\right) \cong \ZZ\left[L\right] / \left(L-1\right)^{n+1}$, where $L$ is the canonical line bundle on $\CP{n}$.
In \ref{k-thy-oriented} we denoted $t = L-1$ (warning: there we looked at $\K^*$, now we focus on $\K$), which allows us to rewrite this as $\K\left(\CP{n}\right) \cong \ZZ\left[t\right] / t^{n+1}$.
As we noted, the limit is $\K\left(\CP{\infty}\right) \cong \ZZ\formal{t}$.
We thus see that $\K\left(\BU[1]\right) \cong \ZZ\formal{t}$ where $t = L - 1$ is the canonical line bundle minus $1$.

The identity map $\rho: \UU[1] \to \UU[1]$, is mapped to the identity $\BU[1] \to \BU[1]$ (by functoriality of $\BB$), which tautologically corresponds the universal line bundle $L$ on $\BU[1]$.
We therefore see that the Atiyah-Segal map $\RG[\UU[1]] \to \K\left(\BU[1]\right)$, sends $\rho$ to $L$, and therefore $t = \rho - 1$ to $t = L - 1$.
Which indeed shows that the map admits $\K\left(\BU[1]\right) \cong \ZZ\formal{t}$ as the $I = \left(t\right)$-completion of $\RG[\UU[1]]$.


\subsubsection{\texorpdfstring{$\ZZ/2$}{Z/2}, Cyclic Group of Order \texorpdfstring{$2$}{2}}

Take $G = \ZZ/2$.
Here we have only two irreducible representations, the trivial, and $\rho\left(0\right) = 1, \rho\left(1\right) = -1$.
Also, it is clear that $\rho \otimes \rho$ is the trivial representation.
Therefore, $\RG[\ZZ/2] = \ZZ[\rho]/\left(\rho^2-1\right)$.
Similarly to before, the augmentation $\varepsilon: \RG[\ZZ/2] \to \ZZ$ sends $1,\rho$ to $1$, so clearly $\left(\rho-1\right) \subseteq I$, and for the same reasoning as in the previous example this is actually an equality.
We change coordinates to $t = \rho-1$, and we have $\RG[\ZZ/2] = \ZZ[t]/\left(t^2+2t\right)$, and $I = \left(t\right)$.

We move to computing the completion $\RG[\ZZ/2]_I^\wedge$.
Modulo $t^2+2t$, i.e. $t^2 = -2t$, we have that $t^n = \left(-2\right)^{n-1} t$.
Thus $I^n = \left(\left(-2\right)^{n-1} t\right) = \left(2^{n-1} t\right)$, so $\RG[\ZZ/2]/I^n = \ZZ[t]/\left(t^2+2t, 2^{n-1} t\right)$.
We first compute the limit of $\RG[\ZZ/2]/I^n$ in abelian groups.
Since the forgetful functor from rings to abelian groups is right adjoint, it commutes with limits, so this will give us the abelian group structure.
As an abelian group, $\RG[\ZZ/2]/I^n$ is isomorphic to $\ZZ \oplus \ZZ/2^{n-1}\left\{t\right\}$.
It is then clear that as an abelian group, $\lim \RG[\ZZ/2]/I^n$ is isomorphic to $\ZZ \oplus \ZZ_2\left\{t\right\}$.

We now define a multiplication on $\ZZ \oplus \ZZ_2\left\{t\right\}$ abelian group, given by $\left(a+bt\right) * \left(c+dt\right) = ac + (ad+bc-2bd)t$.
It can be checked that it is associative and commutative.
We have homomorphisms of groups $\ZZ \oplus \ZZ_2\left\{t\right\} \to \ZZ[t]/\left(t^2+2t, 2^{n-1} t\right)$, admitting it as the limit in groups, which are given by sending $a+bt$ to $a+\left(b \mod 2^{n-1}\right) t$.
By construction this homomorphism is actually a homomorphism of rings (the $-2bdt$ term is explained by the relation $t^2+2t$).
Therefore, by the universal property of the limit, we get a map $\ZZ \oplus \ZZ_2\left\{t\right\} \to \lim \RG[\ZZ/2]/I^n$ in rings.
After taking the forgetful we know that it becomes an isomorphism, but the forgetful reflects isomorphisms, so this is also an isomorphism in rings.

Using the Atiyah-Segal theorem we conclude that
$$
\K\left({\mbb{R}\mrm{P}}^\infty\right)
= \K\left(\BB{\ZZ/2}\right)
\cong \ZZ \oplus \ZZ_2\left\{t\right\},
$$
with multiplication given by $\left(a+bt\right) * \left(c+dt\right) = ac + (ad+bc-2bd)t$.




\subsection{Some Character Theory}

We restrict ourselves to the case of finite groups $G$.
We recall that representations of groups can be studied by their characters.
Specifically the character map $\chi: \RG \to \ZZ\left[\chi_{\rho_i}\right]$, defined by $\chi_\rho = \tr \rho$, is an isomorphism, where the ring on the right is the ring of functions generated by the irreducible characters (the multiplication of two characters is a character so it is indeed closed under multiplication).

We also recall that characters are class functions, that is, they are constant on conjugacy classes.
Let $L$ be some field containing all the values of all characters.
Then a natural place to study characters is in the ring of class functions with values in $L$, denote by $\cl{G;L}$.
Let us phrase this in a way that will be useful in the next section.
$G$ is equipped with a $G$-action by conjugation, $\gamma.g = \gamma g \gamma^{-1}$.
Equip $L$ with the trivial $G$-action.
Then $\cl{G;L} = \hom_{\GSet}\left(G, L\right)$.

We can of course extend the range of the character map to get an injection $\chi: \RG \to \cl{G;L}$.
The first classical theorem regarding the relationship between characters and class functions is:

\begin{theorem}\label{char-1}
	After tensoring with $L$, the character map $\chi \otimes L: \RG \otimes L \xrightarrow{\sim} \cl{G; L}$ becomes an isomorphism.
\end{theorem}

\begin{proof}
	Similarly to the proof in \cite[9.1]{Ser} for $L = \mbb{C}$, we can view $\cl{G;L}$ as a vector space over $L$, and the characters are linearly independent, so by counting them we see that the image of $\chi \otimes L$ has the dimension of the whole vector space and we are done.
\end{proof}

By definition the value of a character is the trace of a linear transformation $\chi_\rho\left(g\right) = \tr \rho\left(g\right) = \sum \lambda_i$ where $\lambda_i$ are the eigenvalues (which exist since the representation is unitary).
Since $g^{\left|G\right|} = e$, we get $\rho\left(g^{\left|G\right|}\right) = \rho\left(e\right) = \id$, but then we get that the eigenvalues of $\rho\left(g^{\left|G\right|}\right)$ are on the one hand $\lambda_i^{\left|G\right|}$ and on the other hand they are all $1$, which means that all the eigenvalues are roots of unity.
Therefore $L = \QQ^\mrm{ab} = \QQ\left(\zeta_\infty\right)$ is always a valid choice for $L$ (regardless of $G$).
To be concrete, we will take this choice.

The Galois group of $\QQ^\mrm{ab}$ is $\Gal\left(\Qab / \QQ\right) \cong {\hat\ZZ}^\times$.
For every $m \in {\hat\ZZ}^\times$ we also denote by $m \in \Gal\left(\Qab / \QQ\right)$ the corresponding element, which can be described as the homomorphism which raises a root of unity to the power of $m$.
Similarly it acts on $G$ by taking $g$ to $g^m$.
Then, for every such $m$ and $g$ we have that
$
\chi_\rho\left(g^m\right)
= \tr \rho\left(g\right)
= \sum \lambda_i^m
= m. \left(\sum \lambda_i\right)
= m. \left(\chi_\rho\left(g\right)\right)
$.
We replace $g$ with $g^{m^{-1}}$ ($m$ is invertible), and rewrite this as $\chi_\rho\left(g\right) = m. \left(\chi_\rho\left(g^{m^{-1}}\right)\right)$.
Similarly to this equality, we can define an action of $\Gal\left(\Qab / \QQ\right)$ on $\cl{G; \Qab}$, by taking a class function $f$ to $m.f$ defined by $\left(m.f\right)\left(g\right) = m. \left(f\left(g^{m^{-1}}\right)\right)$.

Let us rewrite this action in another way, which will be helpful in the next section.
We note that $G \cong \hom_{\TopGrp}\left(\hat\ZZ, G\right)$ (continuous homomorphisms).
We get an action of $\aut\left(\hat\ZZ\right) \cong {\hat\ZZ}^\times$ on $G$ by pre-composition.
Concretely, $m \in {\hat\ZZ}^\times$ acts by sending $g \in G$ to $g^m$.
Since ${\hat\ZZ}^\times$ acts on $\Qab$, we get an action on $\cl{G; \Qab} = \hom_{\GSet}\left(G,  \Qab\right)$ by acting with $m^{-1}$ in the source and with $m$ in the target.
It is evident that this is the same action from the previous paragraph.

As we just saw, the characters are in the fixed points $\cl{G; \Qab}^{\Gal\left(\Qab / \QQ\right)}$.
Also, since the rationals are fixed by the action of the Galois group, rational linear combinations of characters are in the fixed points.
We therefore conclude that the character map after tensoring with $\QQ$ lands in the fixed points, i.e. $\chi \otimes \QQ: \RG \otimes \QQ \to \cl{G; \Qab}^{\Gal\left(\Qab / \QQ\right)}$.
Moreover, the second classical theorem is:

\begin{theorem}[{\cite[Theorem 25]{Ser}}]\label{char-2}
	The map $\chi \otimes \QQ: \RG \otimes \QQ \xrightarrow{\sim} \cl{G; \Qab}^{\Gal\left(\Qab / \QQ\right)}$ is an isomorphism.
\end{theorem}

To conclude, \ref{char-1} tells us that $\RG \otimes \Qab \cong \cl{G; \Qab}$, and \ref{char-2} tells us that $\RG \otimes \QQ \cong \cl{G; \Qab}^{\Gal\left(\Qab / \QQ\right)}$.

	\section{HKR Character Theory}

As we have seen in the previous section, Atiyah and Segal gave a description of $\K\left(\BG\right)$ in terms of the representation ring.
We have also seen in the section on chromatic homotopy theory, that complex K-theory is related to Morava K-theory at height 1 by \ref{k-thy-modp-morava}, and to Morava E-theory at height 1 by \ref{k-thy-comp-defo}.
Representations can be studied using their characters, and one may wonder if a similar construction can be used to studied higher analogues of complex K-theory evaluated at $\BG$.

Hopkins, Kuhn and Ravenel showed in \cite{HKR} that it is indeed possible.
Their paper contains a lot of results, but we will concentrate on theorem C.
Fix some finite group $G$.
Similarly to the proof Atiyah-Segal theorem, the actual proof of theorem C involves a general construction, even to prove the specific case we are interested, but it will be easier to state it first for the specific case.
Let $E = \E{k, \Gamma}$ be the Lubin-Tate spectrum from \ref{lt-spectrum}, for some field $k$ of characteristic $p$, and $\Gamma$ a formal group law over $k$ of height $n$. \todo{they write it specifically for $k = \mbb{F}_{p^n}$, but we don't really need that, right?}
There is some ring $L = L\left(E^*\right)$ (which depends on the spectrum $E$).
It is then possible define some generalized characters $\cl[n,p]{G; L}$, which are completely algebraic and combinatorial (besides the definition of the ring $L$).
Lastly, there is a character map $\chinpG: E^*\left(\BG\right) \to \cl[n,p]{G; L}$.
This character map has similar formal properties to the ordinary character map, namely, similarly to \ref{char-1}, after tensoring with $L$, the character map
$$
\chinpG \otimes L:
E^*\left(\BG\right) \otimes L
\to \cl[n,p]{G; L}
$$
becomes an isomorphism.
Similarly to \ref{char-2}, there is an action of $\aut\left(\ZZ_p^n\right) \cong \left(\ZZ_p^\times\right)^n$ on $\cl[n,p]{G; L}$, and it turns out that the character map lands in the fixed points.
Moreover, we can merely rationalize, which is given by inverting $p$, the source, rather tensoring with $L$.
It turns out that after rationalization and restricting the codomain to the fixed points, the map becomes an isomorphism, that is
$$
p^{-1} \chinpG:
p^{-1} E^*\left(\BG\right)
\to \cl[n,p]{G; L}^{\aut\left(\ZZ_p^n\right)}
$$
is an isomorphism.

\todo{write something about the structure of this section}



\subsection{Towards a Definition of the Character Map}

Following \cite{HKR}, we denote by $\Lambda_r = \left(\ZZ/{p^r}\right)^n$ and $\Lambda = \ZZ_p^n$.

An element $g \in G$ is called \emph{$p$-power-torsion} if $g^{p^a} = e$ for some $a$.
Note that a conjugation of a $p$-power-torsion element is again $p$-power-torsion.
We also denote $\exppG \in \mbb{N}$ to be the minimal number s.t. every every $p$-power-torsion element $g$ satisfies $g^{p^r} = e$.

Consider the set $\Gnp$ of $n$-tuples $\left(g_1, \dotsc, g_n\right)$ of commuting $p$-power-torsion elements.
This set has an action of $G$ by conjugating all the elements in a tuple by the same element, i.e. $\gamma \in G$ acts by $\gamma. \left(g_1, \dotsc, g_n\right) = \left(\gamma^{-1} g_1 \gamma, \dotsc, \gamma^{-1} g_n \gamma\right)$.
Concretely, for $r \geq \exp_p\left(G\right)$, we have $\Gnp = \hom_{\Grp}\left(\Lambda_r, G\right)$, with the $G$-action by conjugation at the values.
In a similar fashion, $\Gnp = \hom_{\TopGrp}\left(\Lambda, G\right)$ (the homomorphisms are required to be continuous).

For any ring $R$, we define the class functions $\cl[n,p]{G; R} = \hom_{\Set}\left(\Gnp, R\right)^G$, that is functions (recall that $\Gnp$ is just a $G$-set) to $R$ which are invariant under the $G$ action.
Note that this is a purely algebraic/combinatorial construction, just a copy of $R$ for every orbit of $\Gnp/G$, that is $\cl[n,p]{G; R} \cong \bigoplus_{\left[\alpha\right] \in \Gnp/G} R$.

We would like to construct a character map $E^*\left(\BG\right) \to \cl[n,p]{G; R}$, for some $R$.
We will try to unravel what this means, and find appropriate $R$'s at the same time.
By the above, this is a homomorphism $E^*\left(\BG\right) \to \bigoplus_{\left[\alpha\right] \in \Gnp/G} R$.
That is, for every $\left[\alpha\right] \in \Gnp/G$ we need to provide a homomorphism $E^*\left(\BG\right) \to R$.
Choose a representative $\alpha \in \Gnp = \hom_{\Grp}\left(\Lambda_r, G\right)$ (for $r \geq \exppG$).
Since $\BB$ is a functor we get $\BB[\alpha]: \BB[\Lambda_r] \to \BG$, and then we can take $E^*$-cohomology to get a homomorphism $\BB[\alpha]^*: E^*\left(\BG\right) \to E^*\left(\BB[\Lambda_r]\right)$.
If we had a homomorphism $E^*\left(\BB[\Lambda_r]\right) \to R$, such that its composition with $\BB[\alpha]^*$ is independent of the choice of $r$ and $\alpha$, we would indeed get a well-defined character map.



\subsection{The Rings \texorpdfstring{$L_r\left(E^*\right)$}{Lr(E*)} and \texorpdfstring{$L\left(E^*\right)$}{L(E*)}}

As we have seen, to construct the character map we needed a ring $R$ together with homomorphisms $E^*\left(\BB[\Lambda_r]\right) \to R$.
We first define a ring $L_r = L_r\left(E^*\right)$.

Recall that $E^*\left(\BU[1]\right) = E^*\left[\left[x\right]\right]$, where $x$ is the complex orientation.
For any homomorphism $\alpha: \Lambda_r \to \UU[1]$, we can take $\BB$ and $E^*$ to get $\BB[\alpha]^*: E^*\left(\UU[1]\right) \to E^*\left(\BB[\Lambda_r]\right)$.
Let $S_r = \left\{ \BB[\alpha]^*\left(x\right) \mid \alpha: \Lambda_r \to \UU[1], \alpha \neq 1 \right\} \subseteq E^*\left(\BB[\Lambda_r]\right)$.
We define $L_r = S_r^{-1} E^*\left(\BB[\Lambda_r]\right)$.
Also, $E^*\left(\BB[\Lambda_r]\right)$ clearly has an $\aut\left(\Lambda_r\right)$ action. \todo{pre composition? post composition?}
This action lifts to an action on $L_r$. \todo{why?}

The projections $\Lambda_{r+1} \to \Lambda_r$ induce maps $E^*\left(\BB[\Lambda_r]\right) \to E^*\left(\BB[\Lambda_{r+1}]\right)$.
By post-composing with the localization of the target, this map induces a map $E^*\left(\BB[\Lambda_r]\right) \to L_{r+1}$.
It turns out that this map lifts to a map from the localization of the source $L_r \to L_{r+1}$. \todo{why?}
Moreover, the projection $\aut\left(\Lambda_{r+1}\right) \to \aut\left(\Lambda_r\right)$ gives a $\aut\left(\Lambda_{r+1}\right)$ action on $L_r$, and the map $L_r \to L_{r+1}$ is equivariant with respect to that action. \todo{why?}
The colimit of this tower is the desired ring, that is $L = L\left(E^*\right) = \colim L_r$.
Since it has a compatible $\aut\left(\Lambda_r\right)$ action for all $r$, we get an $\aut\left(\Lambda\right)$ action compatible with all of the $\aut\left(\Lambda_r\right)$ actions.

\todo{we still need to show that this is compatible with the choice of $r$ and $\alpha$ before}

	\section{Elliptic Curves}

At this point, one may wonder how we can find interesting pairs $\left(k, \Gamma\right)$, of a perfect field and a formal group law over it, to obtain Lubin-Tate spectra.
Two simple examples which we have already seen are the additive formal group law over $\Fp$, of height $\infty$, which gives rise to $\HH{\Fp}$, and the multiplicative formal group law over $\Fp$, of height $1$, which gives rise to $\K_p^\wedge$.
Elliptic curves are another source for formal group laws.



\subsection{Formal Group Laws From Elliptic Curves}

Let $C$ be an elliptic curve over a ring $R$, with $O$ the point at infinity.
In \cite[IV]{Sil}, there is a construction of a formal group law $\Gamma_C$ over $R$, obtained by considering the infinitesimal neighborhood of $O$ in $C$.
We will also denote by $\fG_C$ the corresponding formal group.

Now, assume that $R = k$ is a finite field of characteristic $p$.
We denote by $C\left[p^r\right]$ the $p^r$-torsion, i.e. the (scheme-theoretic) kernel of the multiplication-by-$p^r$ map.
By \cite[IV.7.5]{Sil} and \cite[V.3.1]{Sil}, we have:

\begin{proposition}\label{height-torsion}
	The height of $\Gamma_C$ is either $1$ or $2$.
	Moreover, the height is $2$ if and only if the only point of $C\left[p^r\right]$ is $O$ for all $r \geq 1$.
\end{proposition}

In fact, there are many more equivalent conditions to the above, which can be found in the above reference, and this is turned into a definition.

\begin{definition}
	$C$ is called \emph{supersingular} if $\Gamma_C$ is of height $2$.
\end{definition}

From now on we assume that our elliptic curve is supersingular.

Similarly to the Lubin-Tate deformation theory of formal group laws described in \ref{LT-def}, there is a deformation theory for elliptic curves.
Recall that $\Gamma_C$ over $k$ has a universal deformation.
The Serre-Tate theorem \cite[2.9.1]{KM} then implies the following:

\begin{theorem}\label{serre-tate}
	There exists a deformation $C_U$ over $\Wk\formal{u_1}$ of $C$, whose formal group law $\Gamma_{C_U}$ is a universal deformation of $\Gamma_C$.
\end{theorem}

In this case we get a Lubin-Tate spectrum $E = \E{k, \Gamma_C}$.
We recall from \ref{LT-spectrum} that the coefficients can be taken to be $E_* = \Wk\formal{u_1}\left[u^{\pm 1}\right]$ where $\left|u\right| = 2$, and the formal group law is $u\left(\Gamma_C\right)_U$, which by the above can be described by $u \Gamma_{C_U}$.



\subsection{HKR From Elliptic Curves}

Recall that the our main goal in HKR theory was to compute $p^{-1} E^*\left(\BG\right)$ for some Lubin-Tate spectrum $E$.
We shall focus only the $0$-th level, as this is $2$-periodic in the usual way.
The main result \ref{theorem-c-pt} for us was (stated here only for the $0$-th level, and with $L_r$ for $r \geq r_0$)
$$
p^{-1} E^0\left(\BG\right)
\cong \prod_{\left[\alpha\right] \in \Gnp/{\left(G \times \aut \left(\Lambda_r\right)\right)}}
\left(L_r^0\right)^{\stab_{\aut \left(\Lambda_r\right)}\left(\alpha\right)}.
$$
That is, in order to compute $p^{-1} E^0\left(\BG\right)$, we need to compute $L_r^0$, as defined in \ref{Lr}, and various fixed-points sub-rings thereof.

Recall that we fixed some supersingular elliptic curve $C$ over a finite field $k$ of characteristic $p$, so that $\Gamma_C$ is of height $n=2$, and we consider $E = \E{k, \Gamma_C}$.
As we saw in \ref{serre-tate}, we can use the deformation $C_U$ to describe the universal deformation of the formal group.
In \ref{alg-geo-Lr}, we have seen that $L_r^0$ can be described as follows.
First, $\spec E^0\left(\BB[\Lambda_r]\right) = \left(\fG_{C_U}\left[p^r\right]\right)^2$.
Second, $S_r^0 = \left\{ [k_1]\left(t\right) +_{\Gamma_{C_U}} [k_2]\left(s\right) \mid \left(k_1, k_2\right) \neq 0 \mod p^r\right\}$.
And $L_r^0 = \left(S_r^0\right)^{-1} E^0\left(\BB[\Lambda_r]\right)$.

Now, since $\fG_C$ is the formal neighborhood of $O$ in $C$, we have a map $\fG_C \to C$.
Since the multiplication on $\fG_C$ comes from the multiplication on $C$, we have the commutative square:
$$
\begin{tikzcd}
	\fG_C \arrow{r}{} \arrow{d}{\left[p^r\right]} & C \arrow{d}{\left[p^r\right]} \\
	\fG_C \arrow{r}{} & C
\end{tikzcd}
$$
Taking the kernels of both vertical maps, we get a map $\fG_C\left[p^r\right] \to C\left[p^r\right]$.
Since $C$ is supersingular, by \ref{height-torsion}, the only point of $C\left[p^r\right]$ is $O$, i.e. it is a nilpotent thickening of the point $O$, which means that the map $\fG_C\left[p^r\right] \to C\left[p^r\right]$ is an isomorphism of schemes, since $\fG_C$ is the formal neighborhood of $O$.
Since the group structure on $\fG_C$ is inherited from that of $C$, this is also an isomorphism of group schemes.

In the same way as above, we have a map between the $p^r$-torsion of the deformations, $\fG_{C_U}\left[p^r\right] \to C_U\left[p^r\right]$.
Reducing modulo the maximal ideal, i.e. the map $\Wk\formal{u_1} \to k$, gives the map above $\fG_C\left[p^r\right] \to C\left[p^r\right]$, which is an isomorphism.
By Nakayama's lemma we see that the map $\fG_{C_U}\left[p^r\right] \to C_U\left[p^r\right]$ is also an isomorphism of schemes, and again this is actually an isomorphism of group schemes.

This means that in our computations of $L_r^0$, we can use the elliptic curve $C_U$ rather then its formal group law, as the $p^r$-torsion groups are isomorphic.
This has the advantage that the operations on the elliptic curve are given by polynomials, rather then formal power series.
More explicitly, we have that $\spec E^0\left(\BB[\Lambda_r]\right) \cong \left(C_U\left[p^r\right]\right)^2$, i.e. the scheme-theoretic kernel of the multiplication-by-$p^r$ map on the elliptic curve, squared.
We then need to localize away from the zeros of the functions $[k_1]\left(-\right) +_{C_U} [k_2]\left(-\right)$, for $\left(k_1, k_2\right) \neq 0 \mod p^r$.
As we have seen in \ref{alg-geo-Lr}, we can actually consider only $k_i$'s which are a multiple of $p^{r-1}$, which means that we need to consider only $p^2-1$ pairs.



\subsection{Specific Elliptic Curve}

We now restrict ourselves to a special case.
Take $p = 2$ and $k = \mbb{F}_4$.
We take $C$ to be the elliptic curve given by the Weierstrass equation $y^2 + y = x^3$.
It is supersingular as follows from \cite[exercise V.5.7 combined with proposition A.1.1.c]{Sil}.
Another way to see that is by a direct computation of the terms of the formal group law, which shows that the coefficient of $x$ in the $2$-series is $2=0$, see \cite[6.1.4]{Bea}.

Furthermore, denote by $C_U$ the elliptic curve given by $y^2 + u_1 xy + y = x^3$ over $\Wk\formal{u_1} = \ZZ_2\left[\zeta_3\right]\formal{u_1}$.
The maximal ideal is $\left(1-\zeta_3, u_1\right)$ with residue field $\mbb{F}_4$, and modulo this ideal $C_U$ reduces to $C$.
Furthermore, in \cite[3.5]{LT}, it is proven that the formal group law of $C_U$ is indeed a universal deformation of that of $C$.
Specifically, there the ring $\ZZ_2\left[\zeta_3\right]$ is denoted by $R$, and $u_1$ by $t$.
It is claimed that the formal group law (up to order 2) is given by $x+y+u_1 xy$.
Then, by \cite[1.1]{LT}, it is the universal deformation, because $C_2 = \frac{1}{2}\left(\left(x+y\right)^2-x^2-y^2\right)=xy$.

Our next goal is to compute the ring $L_r^0$ corresponding to $E = \E{\mbb{F}_4, \Gamma_C}$.
To that end, we first need to compute $E^0\left(\BB[\Lambda_r]\right)$, which as we saw is given by $\OO\left(\left(C_U\left[2^r\right]\right)^2\right) = \left(\OO\left(C_U\left[2^r\right]\right)\right)^{\otimes 2}$.
We then need to localize away from the zeros of $[k_1]\left(-\right) +_{C_U} [k_2]\left(-\right)$.
We have only $2^2-1=3$ pairs, which are $\left(2^{r-1},0\right), \left(0,2^{r-1}\right), \left(2^{r-1},2^{r-1}\right)$.
Note that the first two are symmetric, and can be computed even before taking the tensor product.
That is, $L_r^0$ is given by computing $\OO\left(C_U\left[2^r\right]\right)$ (to get the $2^r$-torsion), localizing away from $[2^{r-1}]\left(-\right)$ (to get the points of order exactly $2^r$), tensoring with itself (to get pairs of such points), and localizing away from $[2^{r-1}]\left(-\right) +_{C_U} [2^{r-1}]\left(-\right)$ (to get such pairs that span).

Furthermore, we recall that by \ref{Lr-fixed-points}, $2$ is invertible in $L_r^0$, so the whole computation can be carried with $C_U$ base changed to $\QQ_2\left[\zeta_3\right]\formal{u_1}$, and we will obtain the same result.
Moreover, we note that the elliptic curve, and all the operations described above, are defined already over $R = \QQ\left[u_1\right]$, so we can carry the whole computation over $R$, and tensor in the end with $\QQ_2\left[\zeta_3\right]\formal{u_1}$ to get $L_r^0$.



\subsection{Concrete Computations}

We describe a way to do the calculation described above, namely over $R = \QQ\left[u_1\right]$.
We will display this along the Macaulay 2 code, that carries out the computation.
The code will be at times more general then this specific example, but not completely general.

The basic operation on the elliptic curve are developed in projective coordinates.
Therefore, we homogenize the elliptic curve above to $X^3 = Y^2 Z + u_1 X Y Z + Y Z^2$.
When we need to compute the steps described above, we will work with the affine patch where $Y = 1$, which contains the origin $O = \left[0;1;0\right]$.
We note that in this patch we remove exactly one point from the curve, for if $Y = 0$, then $X^3 = 0 + 0 + 0 = 0$, i.e. $X = 0$, so we remove only the point $\left[0;0;1\right]$.
This will be useful to give a computation of the addition map, and more specifically for the multiplication-by-$2^r$ map.

The code is given below, and what follows is an explanation of the code.

\subsubsection{General Remarks}\label{degree-considerations}

We will use in the code matrices (rather then other data types which store a list of values), as Macaulay 2 has the best support for matrices.

We note that we know a priori the degrees of all the constructions over $R$.
\begin{itemize}
	\item The points of $C_U\left[p^r\right]$ are isomorphic to $\Lambda_r = \left(\ZZ/p^r\right)^2$, so this is of degree $p^{2r}$.
	\item The points of order exactly $p^r$ are then of degree $p^{2r}-p^{2\left(r-1\right)}$.
	\item
		The points of $L_r^0$ are of degree $p^{4\left(r-1\right)}\left(p^2-1\right)\left(p^2-p\right)$.
		To see this, we note that they have a transitive free action of $\aut\left(\Lambda_r\right) = \GL\left(2, \ZZ/p^r\right)$ (since we pick two elements that generate $\Lambda_r$).
		Being invertible is equivalent to having an invertible determinant, and an element is invertible in $\ZZ/p^r$ if and only if it is non-zero modulo $p$.
		There are $p^{r-1}$ ways to lift a non-zero number in $\ZZ/p$ to $\ZZ/p^r$, so there are $p^{4\left(r-1\right)}$ lifts to an invertible matrix.
		Lastly, $\GL\left(2, \ZZ/p\right)$ has $\left(p^2-1\right)\left(p^2-p\right)$, since the first column can be any non-zero vector, and the second must be linearly independent of it.
\end{itemize}

Knowing the degrees will be useful later on.
First and foremost, as some of our constructions will involve arbitrary choices, which might lead to a result which is too large, but having the correct degree guarantees that our choice is valid.
Second, this is a useful sanity-check to verify our results. 

\subsubsection{Basic Objects}

We define the basic objects concerning the computation.
Note that \texttt{r = 2} can be replaced in principal by any value.	
Moreover, we can change \texttt{R = QQ[u1]} to \texttt{QQ} or even \texttt{GF(p)} if we want to work over them, and then we also need to remove the \texttt{u1 * X * Y * Z} term from \texttt{F}.

In the code \texttt{RemovedP} is the point described above which is not in the affine patch we will be using ($Y = 1$).

\subsubsection{Util Functions}

The function \texttt{DivideGcd} has a matrix as its input, which will always be a list of polynomials, and outputs the matrix divided by the $\GCD$ of all of the elements.
The function \texttt{comp} receives two matrices of polynomials, and computes the composition by substituting the variables.
It also divides by the $\GCD$, which does not affect the function in projective coordinates, but is essential in some instances to get the correct affine functions.

\subsubsection{Functions on the Elliptic Curve}

The first function calculated is what we call \texttt{Star}.
This operation is used to define the addition (as explained below), and it satisfies $P_1 \star P_2 = - \left(P_1 + P_2\right)$, in other words, $\left(P_1 \star P_2\right) + P_1 + P_2 = O$.
Geometrically, $P_1 \star P_2$ is the third intersection point of the line through $P_1$ and $P_2$ and the elliptic curve.
Equivalently, projectively, the line is given by $t P_1 + s P_2$, and we are looking for the places where $F\left(t P_1 + s P_2\right) = 0$.
The two trivial solutions are where $t = 0$ or $s = 0$.
We think of $F\left(t P_1 + s P_2\right)$ as a polynomial in $t,s$.
Since $F$ is homogeneous of degree $3$, all terms will have total degree $3$.
The cubic terms $t^3$ and $s^3$ are then precisely those that are in $F\left(t P_1\right)$ and $F\left(s P_2\right)$ respectively.
We assumed that $P_1,P_2$ are on the curve, so we can subtract these terms, and look for the non-trivial zero of $F\left(t P_1 + s P_2\right) - F\left(t P_1\right) - F\left(s P_2\right)$.
This is now a homogeneous polynomial of degree $3$ without $t^3$ and $s^3$, that is, it is of the form $t s \left(c_t t + c_s s\right)$.
Therefore the third solution is for $t = -c_s$ and $s = c_t$, i.e. $P_1 \star P_2 = -c_s P_1 + c_t P_2$.

Now, taking $P_1 = P, P_2 = O$ we get $\left(P \star O\right) + P + O = O$, so $P \star O = -P$.
This explains the introduction of \texttt{Neg}.

Now that we have \texttt{Star} and \texttt{Neg} we can define the addition by $-\left(P_1 \star P_2\right)$.
We call this \texttt{AddCalc}, as this is not quiet the addition function we will be using.
This function has a problem, it vanishes on the diagonal $P_1 = P_2$, which is precisely what we need to multiply by $2$.
The reason is that already the function we denoted by \texttt{Star} vanishes on the diagonal, because when $P_1 = P_2$, the line through the points $t P_1 + s P_2$ is singular (it is just the point).
However, this function is defined everywhere else.
Luckily we can use the following trick, $P_1 + P_2 = \left(P_1 - Q\right) + \left(P_2 + Q\right)$.
Specifically, we take $Q$ to be the removed point $\left[0;0;1\right]$.
\todo{why does this actually work then? do we use the fact that we only care about points in the formal neighborhood of $O$?}

At this point we can introduce \texttt{Mul2} simply by $P+P$.
We further define the function \texttt{MulNTwoDiv} which computes $\left[n\right]$ by expanding $n$ in binary form (if it is divisible by $2$, then $\left[n\right] = \left[2\right]\left[\frac{n}{2}\right]$, otherwise $\left[n\right] = \left[n-1\right]+\id$).
This gives us $\left[p^{r-1}\right],\left[p^r\right]$ denoted by \texttt{Mulprm1} and \texttt{Mulpr} respectively.

\subsubsection{Order Exactly $p^r$}

The next step is to compute $\OO\left(C_U\left[p^r\right]\right)$, and its localization away from $[p^{r-1}]\left(-\right)$.
\todo{explain well why we can work in $Y = 1$}
Then $\OO\left(C_U\left[p^r\right]\right)$ is described as the quotient of $R\left[x, z\right]$ (where $R = \QQ\left[u_1\right]$) by some ideal.
First, we want to restrict to the curve, which gives $F\left(x,1,z\right)$, this explains the first element appended to \texttt{quoPr}.
Second, we want only $p^r$-torsion points, that is $\left[p^r\right]\left(x,1,z\right) = O$.
As $O = \left[0;1;0\right]$, this is equivalent to requiring that the first and last coordinates of $\left[p^r\right]\left(x,1,z\right)$ are $0$, this explains the second and third element appended to \texttt{quoPr}.

Next, we actually wish our points to be of order exactly $p^r$ (and not just $p^r$-torsion).
To require this, we want that $\left[p^{r-1}\right]\left(x,1,z\right) \neq O$.
This means that we to ensure that two values (the first and last coordinates) are not $0$ together, so we can't simply invert both of them (as this requires that both are non-zero, we only need one of them to be non-zero).
Note that if they vanish together, then any linear combination them vanishes as well, but the other direction is not immediate.
However, since there are only finitely many points, the number of different values attained is finite, and there are infinitely many different linear combinations over $R$, so there are linear combinations that vanish if and only if the two vanish together.
We can then choose an arbitrary linear combination, and the result might be too large, but if the degree is correct then our choice was valid.
Since we a priori know the degrees, as described in \ref{degree-considerations}, we can verify this.
It turns out that here taking the sum of the first and last coordinates works.
Now, inverting an element $x \in S$ is equivalent to adding a formal element $d$ and requiring to be its inverse, i.e. $x^{-1} S = S\left[d\right]/(1 - dx)$.
This explains the last element appended to \texttt{quoPr}.

We then define \texttt{IPr}, which is the ideal spanned by those elements, i.e. the quotient by it gives the points of order exactly $p^r$.
Furthermore, \texttt{gbIPr} is a Gr\"{o}bner basis for it.
The \texttt{MonomialOrder} used in the definition of \texttt{BasePr} was found experimentally to yield faster computations (the heuristic is that $x$ appears with the highest power, and it determines $z$ so $z$ it will be a polynomial in $x$, and $d$ will be a polynomial in $x,z$).

\subsubsection{Spanning Pairs}

Here we compute $L_r^0$.
First we just take the tensor product of the previous with itself, which is given simply by duplicating all the variables to \texttt{x1,z1,x2,z2,d1,d2}, and all the relations (we use the Gr\"{o}bner basis for faster computation).
Then, we need to localize away from the kernel of $\left[p^{r-1}\right]\left(x_1,1,z_1\right) +_{C_U} \left[p^{r-1}\right]\left(x_2,1,z_2\right) \neq O$.
We use the same trick, it turns out that here the sum is a again a valid linear combination, and we add another variable \texttt{d3}, and require it to be its inverse.
The \texttt{MonomialOrder} used in the definition of \texttt{BaseLr0} was found experimentally to yield faster computations (the heuristic is similar to the previous).

\subsubsection{The Code}

\begin{lstlisting}
-------- Basic Objects --------

p = 2;
r = 2;
R = QQ[u1];

A3 = R[X,Y,Z];
A33 = R[X1,Y1,Z1,X2,Y2,Z2];
A332 = A33[t,s];

F = X^3 - (Y^2 * Z + u1 * X * Y * Z + Y * Z^2);

Mt = matrix{{t}};
Ms = matrix{{s}};

O = matrix{{0, 1, 0}};
P = matrix{{X, Y, Z}};
P1 = matrix{{X1, Y1, Z1}};
P2 = matrix{{X2, Y2, Z2}};
RemovedP = matrix{{0, 0, 1}};



-------- Util Functions --------

DivideGcd = mat -> (
	g := gcd flatten entries mat;
	matrix {(flatten entries mat) // g}
);

comp = (a, b) -> (
	DivideGcd(sub(a, b))
);



-------- Functions on the Elliptic Curve --------

StarCalc = sub(F, Mt * P1 + Ms * P2) - sub(F, Mt * P1) - sub(F, Ms * P2);
StarCt = StarCalc_(t^2 * s);
StarCs = StarCalc_(t * s^2);
Star = -StarCs * P1 + StarCt * P2;

Neg = comp(Star, P|O);

AddCalc = comp(Neg, sub(Star, P1|P2));
MovedP1 = comp(AddCalc, P1|sub(Neg, RemovedP));
MovedP2 = comp(AddCalc, P2|RemovedP);
Add = comp(AddCalc, MovedP1|MovedP2);

Mul2 = comp(Add, P|P);

MulNTwoDiv = n -> (
	if n == 1 then P
	else if n % 2 == 0 then comp(Mul2, MulNTwoDiv(n / 2))
	else comp(Add, P|MulNTwoDiv(n-1))
);

Mulprm1 = MulNTwoDiv(p^(r-1));

Mulpr = MulNTwoDiv(p^r);



-------- Order Exactly p^r --------

BasePr = R[x,z,d, MonomialOrder=>{Weights => {1, 1000 ,1000000}, Lex}];
p0 = matrix{{x, 1, z}};

quoPr = {};

quoPr = append(quoPr, sub(F, p0));

mulpr = sub(Mulpr, p0);
quoPr = append(quoPr, mulpr_(0,0));
quoPr = append(quoPr, mulpr_(0,2));

mulprm1 = sub(Mulprm1, p0);

quoPr = append(quoPr, 1 - d * (mulprm1_(0,0) + mulprm1_(0,2)));

IPr = ideal quoPr;

gbIPr = flatten entries gens gb IPr;



-------- Spanning Pairs --------

BaseLr0 = R[x1,z1,x2,z2,d1,d2,d3, MonomialOrder=>{Weights => {1,1000,1,1000,1000000,1000000,3000000}, Lex}];
p01d = matrix{{x1, z1, d1}};
p02d = matrix{{x2, z2, d2}};
p1 = matrix{{x1, 1, z1}};
p2 = matrix{{x2, 1, z2}};

quoLr0 = {};

quoLr0 = quoLr0 | (flatten entries sub(matrix{gbIPr}, p01d));
quoLr0 = quoLr0 | (flatten entries sub(matrix{gbIPr}, p02d));

Mulprm1 = MulNTwoDiv(p^(r-1));

mulprm11 = sub(Mulprm1, p1);
mulprm12 = sub(Mulprm1, p2);
addMulprm11and2 = sub(Add, mulprm11|mulprm12)
quoLr0 = append(quoLr0, 1 - d3 * (addMulprm11and2_(0,0) + addMulprm11and2_(0,2)));

ILr0 = ideal quoLr0;

gbILr0 = flatten entries gens gb ILr0;
\end{lstlisting}

	
	\bibliography{refs}{}
	\bibliographystyle{alpha}

\end{document}