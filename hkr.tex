\section{HKR Character Theory}

As we have seen in the previous section, Atiyah and Segal gave a description of $\K\left(\BG\right)$ in terms of the representation ring.
We have also seen in the section on chromatic homotopy theory, that complex K-theory is related to Morava K-theory at height 1 by \ref{k-thy-modp-morava}, and to Morava E-theory at height 1 by \ref{k-thy-comp-defo}.
Representations can be studied using their characters, and one may wonder if a similar construction can be used to study higher analogues of complex K-theory evaluated at $\BG$.

Hopkins, Kuhn and Ravenel showed in \cite{HKR} that it is indeed possible.
Their paper contains a lot of results, but we will concentrate on theorem C.
Fix some finite group $G$.
Similarly to the proof Atiyah-Segal theorem, the actual proof of theorem C involves a general construction, even to prove the specific case we are interested, but it will be easier to state it first for the specific case.
Let $E = \E{k, \Gamma}$ be the Lubin-Tate spectrum from \ref{lt-spectrum}, for some field $k$ of characteristic $p$, and $\Gamma$ a formal group law over $k$ of height $n$. \todo{they write it specifically for $k = \mbb{F}_{p^n}$, but we don't really need that, right?}
There is some ring $L = L\left(E^*\right)$ (which depends on the spectrum $E$).
It is then possible define some generalized characters $\cl[n,p]{G; L}$, which are completely algebraic and combinatorial (besides the definition of the ring $L$).
Lastly, there is a character map $\chinpG: E^*\left(\BG\right) \to \cl[n,p]{G; L}$.
This character map has similar formal properties to the ordinary character map, namely, similarly to \ref{char-1}, after tensoring with $L$, the character map
$$
\chinpG \otimes L:
E^*\left(\BG\right) \otimes L
\to \cl[n,p]{G; L}
$$
becomes an isomorphism.
Similarly to \ref{char-2}, there is an action of $\aut\left(\ZZ_p^n\right) \cong \left(\ZZ_p^\times\right)^n$ on $\cl[n,p]{G; L}$, and it turns out that the character map lands in the fixed points.
Moreover, we can merely rationalize, which is given by inverting $p$, the source, rather tensoring with $L$.
It turns out that after rationalization and restricting the codomain to the fixed points, the map becomes an isomorphism, that is
$$
p^{-1} \chinpG:
p^{-1} E^*\left(\BG\right)
\to \cl[n,p]{G; L}^{\aut\left(\ZZ_p^n\right)}
$$
is an isomorphism.

\todo{write something about the structure of this section}



\subsection{Towards a Definition of the Character Map}

Following \cite{HKR}, we denote by $\Lambda_r = \left(\ZZ/{p^r}\right)^n$ and $\Lambda = \ZZ_p^n$.

An element $g \in G$ is called \emph{$p$-power-torsion} if $g^{p^a} = e$ for some $a$.
Note that a conjugation of a $p$-power-torsion element is again $p$-power-torsion.
We also define $r_0 \in \mbb{N}$ to be the minimal $r$ s.t. every $p$-power-torsion element $g$ is $p^{r_0}$-torsion, i.e. satisfies $g^{p^{r_0}} = e$.

Consider the set $\Gnp$ of $n$-tuples $\left(g_1, \dotsc, g_n\right)$ of commuting $p$-power-torsion elements.
This set has an action of $G$ by conjugating all the elements in a tuple by the same element, i.e. $\gamma \in G$ acts by $\gamma. \left(g_1, \dotsc, g_n\right) = \left(\gamma^{-1} g_1 \gamma, \dotsc, \gamma^{-1} g_n \gamma\right)$.
Concretely, for $r \geq r_0$, we have $\Gnp = \hom_{\Grp}\left(\Lambda_r, G\right)$, with the $G$-action by conjugation at the values.
In a similar fashion, $\Gnp = \hom_{\TopGrp}\left(\Lambda, G\right)$ (the homomorphisms are required to be continuous).

Let $R$ be a ring.
Equip it with the trivial $G$-action.
We define the class functions $\cl[n,p]{G; R} = \hom_{\GSet}\left(\Gnp, R\right)$, that is functions to $R$ which are invariant under the $G$ action.
This is a ring, by defining the operations point-wise.
Note that this is a purely algebraic/combinatorial construction, just a copy of $R$ for every orbit of $\Gnp/G$, that is $\cl[n,p]{G; R} \cong \bigoplus_{\left[\alpha\right] \in \Gnp/G} R$.

We would like to construct a character map $E^*\left(\BG\right) \to \cl[n,p]{G; R}$, for some $R$.
We will try to unravel what this means, and find appropriate $R$'s at the same time.
By the above, this is a homomorphism $E^*\left(\BG\right) \to \bigoplus_{\left[\alpha\right] \in \Gnp/G} R$.
That is, for every $\left[\alpha\right] \in \Gnp/G$ we need to provide a homomorphism $E^*\left(\BG\right) \to R$.
Choose a representative $\alpha \in \Gnp = \hom_{\Grp}\left(\Lambda_r, G\right)$ (for $r \geq r_0$).
Since $\BB$ is a functor we get $\BB[\alpha]: \BB[\Lambda_r] \to \BG$, and then we can take $E$-cohomology to get a homomorphism $\BB[\alpha]^*: E^*\left(\BG\right) \to E^*\left(\BB[\Lambda_r]\right)$.
If we had a homomorphism $E^*\left(\BB[\Lambda_r]\right) \to R$, such that its composition with $\BB[\alpha]^*$ is independent of the choice of $r$ and $\alpha$, we would indeed get a well-defined character map.



\subsection{The \texorpdfstring{$E$}{E}-Cohomology of \texorpdfstring{$\BB[A]$}{BA} and Their Maps}

We postpone the discussion of the rings, to give some properties of $E$-cohomology.

First we wish to describe the $E^*\left(\BB[\ZZ/m]\right)$.
Let $\psi_m: \ZZ/m \to \UU[1]$ be the homomorphism $\psi\left(1\right) = e^{2\pi i/m}$
This induces $\BB\psi^*: E^*\left(\BU[1]\right) \to E^*\left(\BB[\ZZ/m]\right)$.
Denote by $x \in E^2\left(\BB[\ZZ/m]\right)$ the cohomology class $\BB\psi^*\left(x\right)$.

\begin{proposition}[{\cite[5.8]{HKR}}]
	The $E^*$-cohomology of $\BB[\ZZ/m]$ is given by $E^*\left(\BB[\ZZ/m]\right) = E^*\formal{x} / \left(\left[m\right]\left(x\right)\right)$.
	Write $m = sp^t$ with $a$ coprime to $p$, then this is a free $E^*$-module of rank $p^{nt}$.
\end{proposition}
\todo{I don't understand the proof, try to understand and write here}

\begin{proposition}
	Let $Y$ be a space s.t. $E^*\left(Y\right)$ is a free $E^*$-module of finite rank.
	Then $Y$ satisfies Kunneth with respect to any $X$, that is, the map $E^*\left(X\right) \otimes_{E_*} E^*\left(Y\right) \to E^*\left(X \times Y\right)$ is an isomorphism.
\end{proposition}

\begin{proof}
	Look at the functors $X \mapsto E^*\left(X\right) \otimes E^*\left(Y\right)$ and $X \mapsto E^*\left(X \times Y\right)$.
	Both of them are manifestly homotopy invariant.
	Since $E^*\left(Y\right)$ is free, it is also flat, and so both functors satisfy Mayer-Vietoris.
	Both functors send arbitrary wedges to arbitrary products, since tensor with a free finite rank modules commutes with arbitrary products.
	We conclude that they are both cohomology theories.
	Moreover, they agree on $X = *$, and therefore are isomorphic.
\end{proof}

Using both propositions we can bootstrap to arbitrary finite abelian groups.

\begin{proposition}[{\cite[5.8]{HKR}}]\label{E-B-abelian}
	Let $A$ be an abelian group, and write $\left|A\right| = sp^t$ for $s$ coprime to $p$.
	Then $E^*\left(\BB[A]\right)$ is a free $E^*$-module of rank $p^t$, and $\BB[A]$ satisfies Kunneth.
	Specifically, for $A = \ZZ/m_1 \oplus \cdots \oplus \ZZ/m_l$:
	$$
	E^*\left(\BB[\ZZ/m_1] \times \cdots \times \BB[\ZZ/m_l]\right)
	\cong E^*\formal{x_1, \dotsc, x_l}/\left(\left[m_1\right]\left(x_1\right), \dotsc, \left[m_l\right]\left(x_l\right)\right).
	$$
\end{proposition}

\begin{proof}
	A finite abelian group is the product of finite cyclic groups.
	Since $\BB$ commutes with products, we can induct on the number of components in the product.
\end{proof}

Recall that the formal group law on $E_*$ was defined by taking the $E^*\left(\BB-\right)$ to the multiplication map $\UU[1] \times \UU[1] \to \UU[1]$.
That is, this map induces the map $x \mapsto F\left(y_1, y_2\right) = y_1 +_F y_2$ on the cohomology.

By pre-composing with the diagonal, and doing this for $k$-copies of $\UU[1]$, we see that the multiplication-by-$k$ map $\UU[1] \overset{k}{\to} \UU[1]$ induces the map $E^*\formal{x} \to E^*\formal{y}$ given by $x \mapsto \left[k\right]\left(y\right)$.
Moreover, it follows that $\left[l\right]\left(x\right) \mapsto \left[kl\right]\left(y\right)$.

Therefore, for a map $\bigoplus_{i=1}^t \UU[1] \to \bigoplus_{j=1}^s \UU[1]$, given on the $i,j$-th coordinate by multiplication-by-$k_{ij}$, induces a map
$E^*\formal{x_1, \dotsc, x_s} \to E^*\formal{y_1, \dotsc, y_t}$
given by $x_j \mapsto \sum_F\left[k_{ij}\right]\left(y_i\right)$.
From this it follows that:
\begin{align*}
	\sum_{j,F}\left[l_j\right]\left(x_j\right)
	&\mapsto \sum_{j,F}\left[l_j\right]\left(\sum_{i,F}\left[k_{ij}\right]\left(y_i\right)\right)\\
	&= \sum_{j,F}\sum_{i,F}\left[k_{ij} l_j\right]\left(y_i\right)\\
	&= \sum_{i,F}\left[\sum_j k_{ij} l_j\right]\left(y_i\right)
\end{align*}

Now, let $k: \bigoplus_{i=1}^t \ZZ/m_i \to \bigoplus_{j=1}^s \ZZ/\mu_j$, given on the $i,j$-th coordinate by multiplication-by-$k_{ij}$.
Recall the maps $\psi_m: \ZZ/m \to \UU[1]$ given by $1 \mapsto e^{2\pi i/m}$.
We look at the maps $\bigoplus_{i=1}^t \psi_{m_i}: \bigoplus_{i=1}^t \ZZ/m_i \to \bigoplus_{i=1}^t \UU[1]$, and similarly $\bigoplus_{j=1}^s \psi_{\mu_j}$.
The composition $\left(\bigoplus_{j=1}^s \psi_{\mu_j}\right) \circ k$ is given on the $i,j$-th coordinate by
$
1
\mapsto k_{ij}
\mapsto e^{2\pi ik_{ij}/m_j}
$.
Define a map $k: \bigoplus_{i=1}^t \UU[1] \to \bigoplus_{j=1}^s \UU[1]$, by letting the $i,j$-th coordinate being the multiplication-by-$k_{ij}$ map.
We then get the commutative diagram:
$$
\begin{tikzcd}
	\bigoplus_{i=1}^t \UU[1] \arrow{r}{k} & \bigoplus_{j=1}^s \UU[1] \\
	\bigoplus_{i=1}^t \ZZ/m_i \arrow[hook]{u}{\bigoplus_{i=1}^t \psi_{\mu_i}} \arrow{r}{k} & \bigoplus_{j=1}^s \ZZ/\mu_j \arrow[hook]{u}{\bigoplus_{j=1}^s \psi_{\mu_j}}
\end{tikzcd}
$$
By taking $E^*\left(\BB-\right)$ we get the commutative diagram:
$$
\begin{tikzcd}
	E^*\formal{y_1, \dotsc, y_t} \arrow{d}{} & E^*\formal{x_1, \dotsc, x_s} \arrow{d}{} \arrow{l}{} \\
	E^*\formal{y_1, \dotsc, y_t}/\left(\left[m_i\right]\left(y_i\right)\right) & E^*\formal{x_1, \dotsc, x_s}/\left(\left[\mu_j\right]\left(x_j\right)\right) \arrow{l}{}
\end{tikzcd}
$$
Where the vertical maps are given by $y_i \mapsto y_i$ and $x_j \mapsto x_j$.
We have computed the upper map before, so we conclude:

\begin{proposition}\label{E-B-map-cyclic}
	Let $k: \bigoplus_{i=1}^t \ZZ/m_i \to \bigoplus_{j=1}^s \ZZ/\mu_j$ be given on the $i,j$-th coordinate by multiplication-by-$k_{ij}$.
	After taking $E^*\left(\BB-\right)$, it induces the map given by $x_j \mapsto \sum_F\left[k_{ij}\right]\left(y_i\right)$.
	Moreover, for integers $l_1, \dotsc, l_s$, it gives $\sum_{j,F}\left[l_j\right]\left(x_j\right) \mapsto \sum_{i,F}\left[\sum_j k_{ij} l_j\right]\left(y_i\right)$.
\end{proposition}

\begin{remark}
	Note that the $k_{ij}$'s are determined only modulo $\ZZ/\mu_j$, and the proposition implies that the value is independent of the their lifts to $\ZZ$.
	\todo{is that true?}
\end{remark}



\subsection{The Rings \texorpdfstring{$L_r\left(E^*\right)$}{Lr(E*)} and \texorpdfstring{$L\left(E^*\right)$}{L(E*)}}

As we have seen, to construct the character map we needed a ring $R$ together with homomorphisms $E^*\left(\BB[\Lambda_r]\right) \to R$.

Recall that $E^*\left(\BU[1]\right) = E^*\formal{x}$, where $x$ is the complex orientation.
For any homomorphism $\alpha: \Lambda_r \to \UU[1]$, we can take $E^*\left(\BB-\right)$ to get $\BB[\alpha]^*: E^*\left(\UU[1]\right) \to E^*\left(\BB[\Lambda_r]\right)$.
Let $S_r = \left\{ \BB[\alpha]^*\left(x\right) \mid \alpha: \Lambda_r \to \UU[1], \alpha \neq 1 \right\} \subseteq E^*\left(\BB[\Lambda_r]\right)$.
We define $L_r = S_r^{-1} E^*\left(\BB[\Lambda_r]\right)$.

Recall that $E^*\left(\BB[\Lambda_r]\right) \cong E^*\formal{x_1, \dotsc, x_n}/\left(\left[p^r\right]\left(x_1\right), \dotsc, \left[p^r\right]\left(x_n\right)\right)$ by \ref{E-B-abelian}.
Let $\Lambda_r \overset{\alpha}{\to} S^1$ be a homomorphism.
Since it lands in the $p^r$-torsion, it factors as $\Lambda_r \overset{k}{\to} \ZZ/{p^r} \overset{\psi_{p^r}}{\to} \UU[1]$, where $k$ is given on the $i$-th coordinate by multiplication-by-$k_i$.
The condition $\alpha \neq 1$ amounts to the condition $\left(k_1, \dotsc, k_n\right) \neq 0$.
By \ref{E-B-map-cyclic}, the induced map is given by $\BB[\alpha]^*\left(x\right) = \sum_{i,F} [k_i]\left(x_i\right)$.
Therefore $S_r = \left\{ \sum_{i,F} [k_i]\left(x_i\right) \mid \left(k_1, \dotsc, k_n\right) \neq 0\right\}$.

The projections $\Lambda_{r+1} \to \Lambda_r$ are given by the multiplication-by-$1$ on each coordinate, so again by \ref{E-B-map-cyclic} they induce the maps $E^*\left(\BB[\Lambda_r]\right) \to E^*\left(\BB[\Lambda_{r+1}]\right)$, given by $x_i \mapsto x_i$.
Moreover $\sum_{i,F} [k_i]\left(x_i\right) \in S^r$ is mapped to $\sum_{i,F} [k_i]\left(x_i\right) \in S^{r+1}$.
Therefore, once we invert $S^{r+1}$ in the target, all $S^r$ are mapped to units, so we get a map $L_r \to L_{r+1}$.
We define $L = L\left(E^*\right) = \colim L_r$.

Also, $E^*\left(\BB[\Lambda_r]\right)$ clearly has an $\aut\left(\Lambda_r\right)$ action.
Let $k: \Lambda_r \to \Lambda_r$ be a homomorphism given by on the $i,j$-th coordinate by multiplication by $k_{ij}$.
Once again by \ref{E-B-map-cyclic}, for integers $l_1, \dotsc, l_n$, the induced map sends $\sum_{i,F} [l_i]\left(x_i\right)$ to $\sum_{i,F}\left[\sum_j k_{ij} l_j\right]\left(x_i\right)$.
Since we consider only $k \in \aut\left(\Lambda_r\right)$, i.e. invertible, the matrix $\left(k_{ij}\right)$ is invertible.
Therefore, if $\left(l_1, \dotsc, l_n\right) \neq 0$, then also $\left(\sum_j k_{1j} l_j, \dotsc, \sum_j k_{1j} l_j\right) \neq 0$, so if the source is in $S^r$, the result is in $S^r$ as well.
This shows that action lifts to an action on $L_r$.

Moreover, the projection $\aut\left(\Lambda_{r+1}\right) \to \aut\left(\Lambda_r\right)$ gives a $\aut\left(\Lambda_{r+1}\right)$ action on $L_r$, and the map $L_r \to L_{r+1}$ is equivariant with respect to that action, by similar considerations. \todo{explain more?}
Since $L$ has a compatible $\aut\left(\Lambda_r\right)$ action for all $r$, we get an $\aut\left(\Lambda\right)$ action compatible with all of the $\aut\left(\Lambda_r\right)$ actions.

