\section{HKR Generalized Character Theory}

As we have seen in the previous section, Atiyah and Segal gave a description of $\K\left(\BG\right)$ in terms of the representation ring.
We have also seen in the section on chromatic homotopy theory, that complex K-theory is related to Morava K-theory at height 1 by \cref{k-thy-modp-morava}, and to Morava E-theory at height 1 by \cref{k-thy-comp-defo}.
Representations can be studied using their characters, and one may wonder if a similar construction can be used to study higher analogues of complex K-theory evaluated at $\BG$.

Hopkins, Kuhn and Ravenel showed in \cite{HKR} that it is indeed possible.
Their paper contains a lot of results, but we will concentrate on Theorem C.
Fix some finite group $G$.
Similarly to the proof of Atiyah-Segal theorem, the actual proof of Theorem C involves a general construction, even to prove the specific case we are interested in, but it will be easier to state it first for the specific case.
Let $E = \E{k, \Gamma}$ be the Lubin-Tate spectrum from \cref{LT-spectrum}, for some perfect field $k$ of characteristic $p$, and $\Gamma$ a formal group law over $k$ of height $n$.
We will also denote by $F$ the formal group law on $E_*$ (which is $u \Gamma_U$, as in \cref{LT-spectrum}).
There is some ring $L = L\left(E^*\right)$ (which depends on the spectrum $E$).
It is then possible define some generalized class functions $\cl[n,p]{G; L}$, which are completely algebraic and combinatorial (besides the definition of the ring $L$).
Lastly, there is a character map $\chinpG: E^*\left(\BG\right) \to \cl[n,p]{G; L}$.
This character map has similar formal properties to the ordinary character map, namely, similarly to \cref{char-1}, after tensoring with $L$, the character map
$$
\chinpG \otimes L:
E^*\left(\BG\right) \otimes L
\xrightarrow{\sim} \cl[n,p]{G; L}
$$
becomes an isomorphism.
Similarly to \cref{char-2}, there is an action of $\aut\left(\ZZ_p^n\right) \cong \left(\ZZ_p^\times\right)^n$ on $\cl[n,p]{G; L}$, and it turns out that the character map lands in the fixed points.
Moreover, we can merely rationalize, which is given by inverting $p$, the source, rather tensoring with $L$.
The target is already rational.
It turns out that after rationalization and restricting the codomain to the fixed points, the map becomes an isomorphism, that is
$$
p^{-1} \chinpG:
p^{-1} E^*\left(\BG\right)
\xrightarrow{\sim} \cl[n,p]{G; L}^{\aut\left(\ZZ_p^n\right)}
$$
is an isomorphism.

We will first define some of the objects above, to make things more precise, and we will see what exactly we need to construct the rest.
Once we understand that, we will give a more general and detailed construction, which will allow us to state formally the main theorem, \cref{theorem-c}, which is a stronger version of the results written above.
After that we will introduce the idea of complex oriented descent, and prove the theorem.



\subsection{Towards a Definition of the Character Map}\label{towards-character-map}

Following \cite{HKR}, we denote by $\Lambda_r = \left(\ZZ/{p^r}\right)^n$ and $\Lambda = \ZZ_p^n$.

An element $g \in G$ is called \emph{$p$-power-torsion} if $g^{p^a} = e$ for some $a$.
Note that a conjugation of a $p$-power-torsion element is again $p$-power-torsion.
We also define $r_0 \in \mbb{N}$ to be the minimal $r$ s.t. every $p$-power-torsion element $g$ is $p^{r_0}$-torsion, i.e. satisfies $g^{p^{r_0}} = e$.

\begin{definition}
	We define $\Gnp$ to be the set of $n$-tuples $\left(g_1, \dotsc, g_n\right)$ of commuting $p$-power-torsion elements.
	It has a $G$-action by conjugation, $\gamma. \left(g_1, \dotsc, g_n\right) = \left(\gamma g_1 \gamma^{-1}, \dotsc, \gamma g_n \gamma^{-1}\right)$.
\end{definition}

Concretely, for $r \geq r_0$, we have $\Gnp = \hom_{\Grp}\left(\Lambda_r, G\right)$, with the $G$-action by conjugation at the target.
In a similar fashion, $\Gnp = \hom_{\TopGrp}\left(\Lambda, G\right)$ (the homomorphisms are required to be continuous).

Let $R$ be a ring.
Equip it with the trivial $G$-action.

\begin{definition}\label{class-function-*}
	The \emph{class functions} are $\cl[n,p]{G; R} = \hom_{\GSet}\left(\Gnp, R\right)$, that is functions from $\Gnp$ to $R$ which are invariant under conjugation.
\end{definition}

This is a ring, by defining the operations point-wise.
Note that this is a purely combinatorial construction, just a copy of $R$ for every orbit in $\Gnp/G$, that is $\cl[n,p]{G; R} \cong \prod_{\left[\alpha\right] \in \Gnp/G} R$.

We would like to construct a character map $E^*\left(\BG\right) \to \cl[n,p]{G; R}$, for some $R$, which depends on $r \geq r_0$.
We will try to unravel what this means, and find appropriate $R$'s at the same time.
By the above, this is a homomorphism $E^*\left(\BG\right) \to \prod_{\left[\alpha\right] \in \Gnp/G} R$.
That is, for every $\left[\alpha\right] \in \Gnp/G$ we need to provide a homomorphism $E^*\left(\BG\right) \to R$.
Choose a representative $\alpha \in \Gnp = \hom_{\Grp}\left(\Lambda_r, G\right)$ (for $r \geq r_0$).
Since $\BB$ is a functor we get $\BB[\alpha]: \BB[\Lambda_r] \to \BG$, and then we can take $E^*$-cohomology to get a homomorphism $\BB[\alpha]^*: E^*\left(\BG\right) \to E^*\left(\BB[\Lambda_r]\right)$.
If we had a homomorphism $E^*\left(\BB[\Lambda_r]\right) \to R$, we would indeed get a character map.



\subsection{The \texorpdfstring{$E^*$}{E*}-Cohomology of \texorpdfstring{$\BB[A]$}{BA} and Their Maps}

We postpone the discussion of the rings, to give some properties of $E^*$-cohomology.

First we describe $E^*\left(\BB[\ZZ/m]\right)$.
Let $\psi_m: \ZZ/m \to \UU[1]$ be the homomorphism determined by $\psi_m\left(1\right) = e^{2\pi i/m}$.
This induces a map $\BB\psi_m^*: E^*\left(\BU[1]\right) \to E^*\left(\BB[\ZZ/m]\right)$.
Denote by $x \in E^2\left(\BB[\ZZ/m]\right)$ the cohomology class $\BB\psi_m^*\left(x\right)$.

\begin{proposition}[{\cite[5.8]{HKR}}]
	The $E^*$-cohomology of $\BB[\ZZ/m]$ is given by $E^*\left(\BB[\ZZ/m]\right) = E^*\formal{x} / \left(\left[m\right]\left(x\right)\right)$.
	Write $m = sp^t$ with $s$ coprime to $p$, then this is a free $E^*$-module of rank $p^{nt}$ (where $n$ was the height).
\end{proposition}

\begin{proposition}
	Let $Y$ be a space s.t. $E^*\left(Y\right)$ is a free $E^*$-module of finite rank.
	Then $Y$ satisfies K\"unneth with respect to any $X$, that is, the map $E^*\left(X\right) \otimes_{E^*} E^*\left(Y\right) \xrightarrow{\sim} E^*\left(X \times Y\right)$ is an isomorphism.
\end{proposition}

\begin{proof}
	Look at the functors $X \mapsto E^*\left(X\right) \otimes_{E^*} E^*\left(Y\right)$ and $X \mapsto E^*\left(X \times Y\right)$.
	Both of them are manifestly homotopy invariant.
	Since $E^*\left(Y\right)$ is free, it is also flat, and so both functors satisfy Mayer-Vietoris.
	Both functors send arbitrary wedges to arbitrary products, since tensor with a free finite rank modules commutes with arbitrary products.
	We conclude that they are both cohomology theories.
	Moreover, they agree on $X = *$, and therefore are isomorphic.
\end{proof}

Using both propositions we can bootstrap to arbitrary finite abelian groups.

\begin{proposition}[{\cite[5.8]{HKR}}]\label{E-B-abelian}
	Let $A$ be an abelian group, and write $\left|A\right| = sp^t$ for $s$ coprime to $p$.
	Then $E^*\left(\BB[A]\right)$ is a free $E^*$-module of rank $p^{nt}$, and $\BB[A]$ satisfies K\"unneth.
	Specifically, for $A = \ZZ/m_1 \oplus \cdots \oplus \ZZ/m_l$:
	$$
	E^*\left(\BB[\ZZ/m_1] \times \cdots \times \BB[\ZZ/m_l]\right)
	\cong E^*\formal{x_1, \dotsc, x_l}/\left(\left[m_1\right]\left(x_1\right), \dotsc, \left[m_l\right]\left(x_l\right)\right).
	$$
\end{proposition}

\begin{proof}
	A finite abelian group is the product of finite cyclic groups.
	Since $\BB$ commutes with products, we can induct on the number of components in the product, and the proof follows by the two previous propositions.
\end{proof}

Recall that the formal group law on $E_*$ was defined by taking the $E^*\left(\BB-\right)$ to the multiplication map $\UU[1] \times \UU[1] \to \UU[1]$.
That is, this map induces the map $x \mapsto F\left(y, z\right) = y +_F z$ on the cohomology.

By pre-composing with the diagonal, and doing this for $k$-copies of $\UU[1]$, we see that the multiplication-by-$k$ map $\UU[1] \overset{k}{\to} \UU[1]$ induces the map $E^*\formal{x} \to E^*\formal{y}$ given by $x \mapsto \left[k\right]\left(y\right)$.

Let $k: \bigoplus_{i=1}^t \UU[1] \to \bigoplus_{j=1}^s \UU[1]$ be the map given on the $i,j$-th coordinate by multiplication-by-$k_{ij}$, it induces a map
$E^*\formal{x_1, \dotsc, x_s} \to E^*\formal{y_1, \dotsc, y_t}$
given by $x_j \mapsto \sum_F\left[k_{ij}\right]\left(y_i\right)$.
From this it follows that:
\begin{align*}
	\sum_{j,F}\left[l_j\right]\left(x_j\right)
	&\mapsto \sum_{j,F}\left[l_j\right]\left(\sum_{i,F}\left[k_{ij}\right]\left(y_i\right)\right)\\
	&= \sum_{j,F}\sum_{i,F}\left[k_{ij} l_j\right]\left(y_i\right)\\
	&= \sum_{i,F}\left[\sum_j k_{ij} l_j\right]\left(y_i\right)
\end{align*}

Let $k: \bigoplus_{i=1}^t \ZZ/m_i \to \bigoplus_{j=1}^s \ZZ/\mu_j$ be given on the $i,j$-th coordinate by multiplication-by-$k_{ij}$ (where $k_{ij}$ is defined only modulo $\mu_j$).
Recall the maps $\psi_m: \ZZ/m \to \UU[1]$ given by $1 \mapsto e^{2\pi i/m}$.
We look at the maps $\bigoplus_{i=1}^t \psi_{m_i}: \bigoplus_{i=1}^t \ZZ/m_i \to \bigoplus_{i=1}^t \UU[1]$, and similarly $\bigoplus_{j=1}^s \psi_{\mu_j}$.
The composition $\left(\bigoplus_{j=1}^s \psi_{\mu_j}\right) \circ k$ is given on the $i,j$-th coordinate by
$
1
\mapsto k_{ij}
\mapsto e^{2\pi ik_{ij}/\mu_j}
$.
Define a map $k: \bigoplus_{i=1}^t \UU[1] \to \bigoplus_{j=1}^s \UU[1]$, by letting the $i,j$-th coordinate being the multiplication-by-$k_{ij}$ map (where we choose some lift of $k_{ij}$ from $\ZZ/\mu_j$ to $\ZZ$).
We then get the commutative diagram:
$$
\begin{tikzcd}
	\bigoplus_{i=1}^t \UU[1] \arrow{r}{k} & \bigoplus_{j=1}^s \UU[1] \\
	\bigoplus_{i=1}^t \ZZ/m_i \arrow[hook]{u}{\bigoplus_{i=1}^t \psi_{m_i}} \arrow{r}{k} & \bigoplus_{j=1}^s \ZZ/\mu_j \arrow[hook]{u}{\bigoplus_{j=1}^s \psi_{\mu_j}}
\end{tikzcd}
$$
By taking $E^*\left(\BB-\right)$ we get the commutative diagram:
$$
\begin{tikzcd}
	E^*\formal{y_1, \dotsc, y_t} \arrow{d}{} & E^*\formal{x_1, \dotsc, x_s} \arrow{d}{} \arrow{l}{} \\
	E^*\formal{y_1, \dotsc, y_t}/\left(\left[m_i\right]\left(y_i\right)\right) & E^*\formal{x_1, \dotsc, x_s}/\left(\left[\mu_j\right]\left(x_j\right)\right) \arrow{l}{}
\end{tikzcd}
$$
Where the vertical maps are given by $y_i \mapsto y_i$ and $x_j \mapsto x_j$.
We have computed the upper map before, and since the vertical maps are surjections, we conclude:

\begin{proposition}\label{E-B-map-cyclic}
	Let $k: \bigoplus_{i=1}^t \ZZ/m_i \to \bigoplus_{j=1}^s \ZZ/\mu_j$ be given on the $i,j$-th coordinate by multiplication-by-$k_{ij}$.
	After taking $E^*\left(\BB-\right)$, it induces the map given by $x_j \mapsto \sum_F\left[k_{ij}\right]\left(y_i\right)$.
	Moreover, for integers $l_1, \dotsc, l_s$, it gives $\sum_{j,F}\left[l_j\right]\left(x_j\right) \mapsto \sum_{i,F}\left[\sum_j k_{ij} l_j\right]\left(y_i\right)$.
\end{proposition}

\begin{remark}
	Note that the $k_{ij}$'s are defined only modulo $\ZZ/\mu_j$, but $\left[k_{ij}\right]\left(y_i\right)$ requires us to choose a lift to $\ZZ$.
	We see that the result is independent of the lift.
\end{remark}



\subsection{The Rings \texorpdfstring{$L_r\left(E^*\right)$}{Lr(E*)} and \texorpdfstring{$L\left(E^*\right)$}{L(E*)}}

As we have seen, to construct the character map we needed a ring $R$ together with homomorphisms $E^*\left(\BB[\Lambda_r]\right) \to R$.
We will construct such a ring, $L_r = L_r\left(E^*\right)$.
Moreover, we will take a colimit to construct a ring $L = L\left(E^*\right)$.

Recall that $E^*\left(\BU[1]\right) = E^*\formal{x}$, where $x$ is the complex orientation.
For any homomorphism $\alpha: \Lambda_r \to \UU[1]$, we can take $E^*\left(\BB-\right)$ to get $\BB[\alpha]^*: E^*\left(\UU[1]\right) \to E^*\left(\BB[\Lambda_r]\right)$.
Let $S_r = \left\{ \BB[\alpha]^*\left(x\right) \mid \alpha: \Lambda_r \to \UU[1], \alpha \neq 1 \right\} \subseteq E^*\left(\BB[\Lambda_r]\right)$.

\begin{definition}\label{Lr}
	We define $L_r = S_r^{-1} E^*\left(\BB[\Lambda_r]\right)$.
	There is indeed a map $E^*\left(\BB[\Lambda_r]\right) \to L_r$, namely the localization map.
\end{definition}

We wish to describe the above construction with coordinates, to make it more explicit.
Recall that $E^*\left(\BB[\Lambda_r]\right) \cong E^*\formal{x_1, \dotsc, x_n}/\left(\left[p^r\right]\left(x_1\right), \dotsc, \left[p^r\right]\left(x_n\right)\right)$ by \cref{E-B-abelian}.
Let $\Lambda_r \overset{\alpha}{\to} \UU[1]$ be a homomorphism.
Since it lands in the $p^r$-torsion, it factors as $\Lambda_r \overset{k}{\to} \ZZ/{p^r} \overset{\psi_{p^r}}{\to} \UU[1]$, where $k$ is given on the $i$-th coordinate by multiplication-by-$k_i$.
The condition $\alpha \neq 1$ amounts to the condition $\left(k_1, \dotsc, k_n\right) \neq 0 \mod p^r$.
By \cref{E-B-map-cyclic}, the induced map is given by $\BB[\alpha]^*\left(x\right) = \sum_F [k_i]\left(x_i\right)$.
Therefore $S_r = \left\{ \sum_F [k_i]\left(x_i\right) \mid \left(k_1, \dotsc, k_n\right) \neq 0 \mod p^r\right\}$.

\begin{proposition}
	The map $E^*\left(\BB[\Lambda_r]\right) \to E^*\left(\BB[\Lambda_{r+1}]\right)$ induced by the projection $\Lambda_{r+1} \to \Lambda_r$, lifts to a map $L_r \to L_{r+1}$.
\end{proposition}

\begin{proof}
	The projection $\Lambda_{r+1} \to \Lambda_r$ is given by the multiplication-by-$1$ on each coordinate, so again by \cref{E-B-map-cyclic} they induce the maps $E^*\left(\BB[\Lambda_r]\right) \to E^*\left(\BB[\Lambda_{r+1}]\right)$, given by $x_i \mapsto x_i$.
	Moreover $\sum_F [k_i]\left(x_i\right) \in S_r$ is mapped to $\sum_F [k_i]\left(x_i\right) \in S_{r+1}$.
	Therefore, once we invert $S_{r+1}$ in the target, clearly $S_r$ are sent to invertibles, so the map lifts to the localization of the source.
\end{proof}

\begin{definition}
	We define $L = L\left(E^*\right) = \colim L_r$.
\end{definition}

By definition, $\aut\left(\Lambda_r\right)$ acts on $\Lambda_r$, so by functoriality we get that it also acts on $E^*\left(\BB[\Lambda_r]\right)$.

\begin{proposition}
	The $\aut\left(\Lambda_r\right)$-action lifts to an action on $L_r$.
\end{proposition}

\begin{proof}
	Let $k: \Lambda_r \to \Lambda_r$ be an automorphism given by on the $i,j$-th coordinate by multiplication by $k_{ij}$.
	Once again, by \cref{E-B-map-cyclic}, for integers $l_1, \dotsc, l_n$, the induced map sends $\sum_{j,F} [l_j]\left(x_j\right)$ to $\sum_{i,F}\left[\sum_j k_{ij} l_j\right]\left(x_i\right)$.
	Since $k$ is an automorphism, the matrix $\left(k_{ij}\right)$ is invertible.
	Therefore, if $\left(l_1, \dotsc, l_n\right) \neq 0$, then also $\left(\sum_j k_{1j} l_j, \dotsc, \sum_j k_{1j} l_j\right) \neq 0$, so if the source is in $S_r$, the result is in $S_r$ as well.
	This shows that action lifts to an action on $L_r$.
\end{proof}

Using the projection $\pi_r: \aut\left(\Lambda\right) \to \aut\left(\Lambda_r\right)$ we endow $L_r$ with an $\aut\left(\Lambda\right)$-action.
By factoring the projection through $\aut\left(\Lambda_{r+1}\right)$, we see that the map $L_r \to L_{r+1}$ is equivariant with respect to that action.
In conclusion:

\begin{proposition}
	The rings $L_r$ have an $\aut\left(\Lambda\right)$-action, and the maps $L_r \to L_{r+1}$ are equivariant with respect to this action.
	Therefore $L$ has an $\aut\left(\Lambda\right)$-action as well.
\end{proposition}

One may wonder if the ring $L_r$ is the zero ring.
An argument in \cite{HKR} shows that this isn't the case, and even more is true.

\begin{proposition}[{\cite[6.5, 6.6, 6.8]{HKR}}]\label{Lr-fixed-points}
	The element $p$ is invertible in $L$, so $L$ is a $p^{-1} E^*$-module.
	Furthermore, $L^{\aut \left(\Lambda\right)} = p^{-1} E^*$, and $L$ is faithfully flat over it.
	Moreover, this holds when $L$ is replaced with $L_r$.
\end{proposition}


\subsubsection{Algebro-Geometric Interpretation}\label{alg-geo-Lr}

First we wish to simplify the situation a little bit.
Recall from \cref{LT-spectrum} that we have a formal group law $F = u \Gamma_U$ over $\E{k, \Gamma}_* = \Wkuiu$.
This came from the computation $\E{k, \Gamma}^*\left(\BU[1]\right) = \E{k, \Gamma}^* \formal{x}$ with $\left|x\right| = -2$.
It will be more convenient to work with $t = u^{-1} x$, which lives in degree $0$, similarly to \cref{k-thy-fgl}.
On these elements, the formal group law acts the same as $\Gamma_U$.
By the invertibility of $u$, we can do the computation of $L_r$ with these elements, and the results will not be affected.
Moreover, since everything is defined already over $E^0 = \Wkui$, we can do all the computations over it, which will give the ring $L_r^0$, the $0$-th degree part of $L_r$, and add $u^{\pm 1}$ at the end to get $L_r$.

The formal group law $\Gamma_U$ over $E^0$, gives a formal group $\fG = \spf E^0\formal{t}$ with multiplication $\fG \times \fG \to \fG$.
By definition, the $p^r$-torsion elements in $\fG$, is the scheme-theoretic kernel of the multiplication-by-$p^r$ map $\left[p^r\right]: \fG \to \fG$, that is $\fG\left[p^r\right] = \spec\left(E^0\formal{t}/\left(\left[p^r\right]\left(t\right)\right)\right) = \spec E^0\left(\BB[\ZZ / p^r]\right)$.
Since $\Gamma_U$ is of height $n$, the leading term of the $p^r$-series, is $t^{p^{rn}}$, which by a variant of Weierstrass preparation (see \cite[5.1]{HKR}) shows that $\fG\left[p^r\right]$ is of rank $p^{rn}$.
We also see that $\spec E^0\left(\BB[\Lambda_r]\right) = \left(\fG\left[p^r\right]\right)^n$.

Inverting $S_r$ is equivalent to inverting their $0$-th graded analogues $S_r^0 = \left\{ \sum_{\Gamma_U} [k_i]\left(t_i\right) \mid \left(k_1, \dotsc, k_n\right) \neq 0 \mod p^r\right\}$.
Algebro-geometrically, this is equivalent to restricting to the open subset where all the functions $\sum_{\Gamma_U} [k_i]\left(t_i\right)$ don't vanish.
That is, $L_r^0$ is the open subset of $n$-tuple of points in $\fG\left[p^r\right]$, i.e. $n$ points in the $p^r$-torsion, which are linearly independent.
In fact, the points of $\fG\left[p^r\right]$, as a group, are isomorphic (non-canonically) to $\left(\ZZ/p^r\right)^n = \Lambda_r$, so this means that the $n$ points are not only linearly independent, but also span, that is they form a basis.

Moreover, if $\sum_{\Gamma_U} [k_i]\left(t_i\right) = 0$ then also $\sum_{\Gamma_U} [p k_i]\left(t_i\right) = \left[p\right]\left(\sum_{\Gamma_U} [k_i]\left(t_i\right)\right) = 0$.
So, if any $p k_i$ is not $0$ modulo $p^r$, inverting $\sum_{\Gamma_U} [p k_i]\left(t_i\right)$ already inverts $\sum_{\Gamma_U} [k_i]\left(t_i\right)$.
Well, $p k_i = 0 \mod p^r$ if and only if $k_i = 0 \mod p^{r-1}$, which shows that we can invert only those where all $k_i$'s are a multiple of $p^{r-1}$.
Since there are $n$ $k_i$'s, each of them can take any of $p$ values (numbers which are a multiple of $p^{r-1}$), and not all $0$, we need to invert only $p^n - 1$ elements.

The description of $L_r^0$ as a basis for $\Lambda_r$ also shows where the $\aut \left(\Lambda_r\right)$ action comes from, it just changes the basis by multiplying by an invertible matrix.

\begin{example}[K-Theory Saga: The Ring $L_r$]\label{k-thy-Lr}
	We continue with complex K-theory.
	Recall from \cref{k-thy-comp-defo} that $\K_p^\wedge \cong \E{\Fp, \Gamma}$, where $\Gamma$ is the multiplicative formal group law, $\Gamma\left(y,z\right) = y + z + yz$.
	That is, $p$-complete K-theory is a Lubin-Tate spectrum at height $n = 1$, so the the construction above applies to it.
	It is worth noting that the computation here should be related to Atiyah-Segal, although there we considered $\K$ itself, and here $\K_p^\wedge$, and as we will see we will indeed get that $L$ is a natural place to study characters in a $p$-complete situation.
	
	In \cref{k-thy-comp-left}, we saw that $\left(\K_p^\wedge\right)_* = \ZZ_p\left[\beta^{\pm 1}\right]$.
	As in \cref{alg-geo-Lr}, it is easier to work with the element $t = \beta^{-1} x$ and the formal group law $u + v + uv$ over $\left(\K_p^\wedge\right)_0 = \ZZ_p$.
	The $n$-series then is $\left[n\right]\left(t\right) = \left(1 + t\right)^n - 1$.
	
	In our case $\Lambda_r = \ZZ/p^r$, and we have
	$
	\left(\K_p^\wedge\right)^0\left(\BB[\ZZ/p^r]\right)
	\cong \ZZ_p\formal{t}/\left(\left[p^r\right]\left(t\right)\right)
	= \ZZ_p\formal{t}/\left(\left(1 + t\right)^{p^r} - 1\right)
	$.
	Again, by a variant of Weierstrass preparation (see \cite[5.1]{HKR}) the inclusion $\ZZ_p\left[t\right] \to \ZZ_p\formal{t}$ induces an isomorphism of the ring above to 	$\ZZ_p\left[t\right]/\left(\left(1 + t\right)^{p^r} - 1\right)$.
	To make the computation easier, we change variable $s = 1+t$, to work with the ring
	$\ZZ_p\left[s\right]/\left(s^{p^r} - 1\right)$.
	
	We note that this is in accordance with the algebro-geometric point of view from \cref{alg-geo-Lr}, as the spectrum of this ring is isomorphic as a group scheme to the group of roots of unity of order $p^r$.
	
	Under this change of variables, $S_r = \left\{s^k - 1 \mid 0 < k < p^r\right\}$.
	By \cref{Lr-fixed-points}, $p$ is invertible in the localization, so might as well invert it before inverting $S_r$.
	We then denote $R = \QQ_p\left[s\right]/\left(s^{p^r} - 1\right)$, and our goal is to compute ${S_r}^{-1} R$.
	Denote by $\Phi_k\left(s\right)$ the $k$-th cyclotomic polynomial, and by $\zeta_k$ a primitive $k$-th root of unity.
	Recall that $s^{p^r} - 1 = \left(s^{p^{r-1}} - 1\right) \Phi_{p^r}\left(s\right)$
	Therefore we have a quotient map $R \to \QQ_p\left[s\right]/\left(\Phi_{p^r}\right) \cong \QQ_p\left(\zeta_{p^r}\right)$, and we claim that it admits the target as the $S_r$-localization of the source.
	
	First, note that $s^k-1 \in S_r$ is sent to $\zeta_{p^r}^k - 1$, and since $0 < k < p^r$, this is not zero.
	Since the codomain is a field, this is invertible, so by the universal property of localization, we get a map ${S_r}^{-1} R \to \QQ_p\left[s\right]/\left(\Phi_{p^r}\right)$.
	
	Second, look at the map $\QQ_p\left[s\right] \to {S_r}^{-1} R$ (the composition of the quotient and localization maps).
	We took the quotient by $s^{p^r} - 1$, and inverted $s^{p^{r-1}} - 1$, so $\Phi_{p^r}\left(s\right)$ is zero in ${S_r}^{-1} R$ as well.
	Therefore, the map actually factors to a map $\QQ_p\left[s\right]/\left(\Phi_{p^r}\right) \to {S_r}^{-1} R$, which is clearly an inverse to the map above.
	
	We conclude that our ring is $L_r^0 = \QQ_p\left(\zeta_{p^r}\right)$.
	This is again in accordance with \cref{alg-geo-Lr}, since the points now are primitive roots of unity, that is each points forms a basis for the group of roots of unity.
	
	The whole graded ring is $L_r = \QQ_p\left(\zeta_{p^r}\right)\left[\beta^{\pm 1}\right]$.
	From this it is also easy to see that $L = \QQ_p\left(\zeta_{p^\infty}\right)\left[\beta^{\pm 1}\right]$.
\end{example}



\subsection{The Generalized Class Functions Ring}

As in the case of Atiyah-Segal, in order to establish Theorem C, that is, the result on the character map described in the introduction to this section, we need to formulate a more general theorem.
The theorem will also depend on a $G$-space $X$, and the proof will use this freedom in a crucial way.
Moreover, the proof will rely on passing to other subgroups as well, although unlike in Atiyah-Segal's proof, it will only use abelian subgroups of $G$.
To this end, we define the generalized objects we need.

Let $X$ be a finite $G$-CW complex.
Recall from \cref{towards-character-map} that for $r \geq r_0$, we have $\Gnp = \hom_{\Grp}\left(\Lambda_r, G\right)$, with the $G$-action by conjugation.
Note that for $\alpha \in \Gnp$, $X^{\im \alpha}$ is simply the fixed points of $X$ at the $n$-tuple $\left(g_1, \dotsc, g_n\right)$ which is represented by $\alpha$, so the following definition is independent of $r \geq r_0$.

\begin{definition}
	The \emph{fixed point space} of $X$ is $\Fix{G,X} = \coprod_{\alpha \in \Gnp} X^{\im \alpha}$.
	This space has a $G \times \aut \left(\Lambda_r\right)$-action, described below.
\end{definition}

$\Fix{G,X}$ admits a $G$-action, where $\gamma \in G$ sends $x \in X^{\im \alpha}$ to $\gamma x \in X^{\im \gamma.\alpha}$.
This is well defined, since if $x$ fixed by $\alpha$, i.e. $g_i x = x$, then $\gamma g_i \gamma^{-1} \gamma x = \gamma x$, so $\gamma x$ is fixed by $\gamma. \alpha$.
Moreover, it admits an $\aut\left(\Lambda_r\right)$-action.
Let $\varphi \in \aut\left(\Lambda_r\right)$, for any $\alpha \in \Gnp$, clearly $\im \alpha = \im \left(\alpha\circ\varphi\right)$, so $x \in X^{\im \alpha}$ is mapped by $\varphi$ to $x \in X^{\im \left(\alpha\circ\varphi\right)}$ (i.e. this action just permutes to coordinates labeled by the $\alpha \in \Gnp$).
The actions commute, since $\gamma.\left(\alpha \circ \varphi\right) = \left(\gamma.\alpha\right) \circ \varphi$, because $\varphi$ acts on the source and $\gamma$ on the target.
Therefore we have a $G \times \aut \left(\Lambda_r\right)$-action.

The action on $\Fix{G,X}$ gives a $G \times \aut \left(\Lambda_r\right)$-action of $E^*$-algebras on $E^*\left(\Fix{G,X}\right)$.
As we saw, $L_r$ admits an $\aut \left(\Lambda_r\right)$-action, define the trivial $G$-action on it, to get a $G \times \aut \left(\Lambda_r\right)$-action.
Take the diagonal $G \times \aut \left(\Lambda_r\right)$-action on $L_r \otimes_{E^*} E^*\left(\Fix{G,X}\right)$.

\begin{definition}
	The \emph{class functions} are:
	$$
	\cl[n,p]{G,X; L_r}
	= \left(L_r \otimes_{E^*} E^*\left(\Fix{G,X}\right)\right)^G
	$$
	This $E^*$-algebra still has an $\aut \left(\Lambda_r\right)$-action.
\end{definition}

Note that for $X = *$, trivial $G$-space, $\Fix{G,*} = \Gnp$ as a $G$-space.
Hence $E^*\left(\Fix{G,*}\right) \cong \hom_{\Set}\left(\Gnp, E^*\right)$.
Taking the $G$ fixed points gives $\hom_{\GSet}\left(\Gnp, E^*\right)$ (note that this $\hom$ is in $\GSet$).
In conclusion, we get $\cl[n,p]{G,*; L_r} =  \hom_{\GSet}\left(\Gnp, L_r\right)$.
This agrees with \cref{class-function-*}.

We give an alternative description of the algebra before taking the $G$ fixed points.
Simply by taking the coproduct out of the cohomology as a product, and out of the tensor product, we get:

\begin{proposition}\label{Lr-Fix}
	$
	L_r \otimes_{E^*} E^*\left(\Fix{G,X}\right)
	\cong \prod_{\alpha \in \Gnp} \left(L_r \otimes_{E^*} E^*\left(X^{\im \alpha}\right)\right)
	$.
\end{proposition}

We wish to emphasize the combinatorial nature of the $G \times \aut \left(\Lambda_r\right)$-action.
To that end, we first formulate a general combinatorial statement:

\begin{proposition}\label{combinatorial-situation}
	Let $H$ be a group.
	Let $I$ be some indexing $H$-set, and let $Y_{\left[i\right]}$ be a collection of $H$-spaces indexed by $I$ that depend only on the orbit.
	Endow $\prod_{i \in I} Y_{\left[i\right]}$ with an $H$-action by
	$
	h.\left(y_i\right)_{i \in I}
	= \left(h.y_{h.i}\right)_{i \in I}
	$.
	Then
	$
	\left(\prod_{i \in I} Y_{\left[i\right]}\right)^H
	\cong \prod_{\left[i\right] \in I/H} Y_{\left[i\right]}^{\stab_H\left(i\right)}
	$.
\end{proposition}

\begin{proof}
	First, the action is indeed well defined since $y_{h.i} \in Y_{\left[i\right]}$, because $i$ and $h.i$ are in the same orbit.
	Requiring that $\left(y_i\right)_{i \in I}$ is a fixed point amounts to $y_i = h.y_{h.i}$ for all $i$ and $h$.
	Note that this condition equates $y_i$ only with values which are in the $H$-orbit of $i$.
	Moreover, the value of $y_i$ determines the whole $H$-orbit $y_{h.i}$ (by $h^{-1} y_i$).
	Therefore a fixed point is determined by one element per orbit, $y_i$, that satisfies $y_i = h. y_i$ when $h.i = i$, i.e. when $h$ is in the stabilizer $\stab_H\left(i\right)$.
	So the condition is simply that $y_i \in Y_{\left[i\right]}^{\stab_H\left(i\right)}$, and the conclusion follows.
\end{proof}

We now want to apply this to our case.

\begin{proposition}\label{combinatorial-cl-fixed}
	We have:
	\begin{multline*}
		\cl[n,p]{G,X; L_r}^{\aut \left(\Lambda_r\right)}\\
		\cong \prod_{\left[\alpha\right] \in \Gnp/{\left(G \times \aut \left(\Lambda_r\right)\right)}}
		\left(
		L_r^{\stab_{\aut \left(\Lambda_r\right)}\left(\alpha\right)}
		\otimes_{E^*} E^*\left(X^{\im \alpha}\right)^{\stab_G\left(\alpha\right)}
		\right)
	\end{multline*}
	In particular, when $X = *$, we have:
	$$
	\cl[n,p]{G; L_r}^{\aut \left(\Lambda_r\right)}
	\cong \prod_{\left[\alpha\right] \in \Gnp/{\left(G \times \aut \left(\Lambda_r\right)\right)}}
	L_r^{\stab_{\aut \left(\Lambda_r\right)}\left(\alpha\right)}
	$$
\end{proposition}

\begin{proof}
	We will take $G \times \aut \left(\Lambda_r\right)$ fixed points of \cref{Lr-Fix} in two steps, first by $G$ then by $\aut \left(\Lambda_r\right)$.
	Recall that $L_r$ is fixed by $G$.
	The action on the cohomology part comes from the action on the space, taking $x \in X^{\im \alpha}$ to $\gamma x \in X^{\im \gamma.\alpha}$.
	Since this is invertible, this gives a homeomorphism between $X^{\im \alpha}$ and $X^{\im \gamma.\alpha}$, which shows that $E^*\left(X^{\im \alpha}\right) \cong E^*\left(X^{\im \gamma.\alpha}\right)$.
	Using this isomorphism, we see that $E^*\left(X^{\im \alpha}\right)$ depends only on the $G$-orbit of $\alpha$, so we can employ \cref{combinatorial-situation} for \cref{Lr-Fix} with $H = G$ and $I = \Gnp$, to get:
	$$
	\cl[n,p]{G,X; L_r}
	\cong \prod_{\left[\alpha\right] \in \Gnp/G} \left(L_r \otimes_{E^*} E^*\left(X^{\im \alpha}\right)^{\stab_G\left(\alpha\right)}\right)
	$$
	We now have the $\aut \left(\Lambda_r\right)$-action.
	Recall that $\aut \left(\Lambda_r\right)$ didn't act on the space part, since it fixes $\im \alpha$.
	We are then in the situation of \cref{combinatorial-situation} again, for $H = \aut \left(\Lambda_r\right)$ and $I = \Gnp/G$, and the general case follows.
	
	When $X = *$, all fixed points $X^{\im \alpha}$ are again trivial, which shows that the $E^*$-cohomology is simply $E^*$, and tensoring with it over $E^*$ does nothing, and the specific case follows.
\end{proof}



\subsection{The General Character Map}

We now construct the character map, that also depends on $r$, which is omitted from the notation:
$$\chinpG: E^*\left(\EG \times_G X\right) \to \cl[n,p]{G,X; L_r}$$
This map is given by a map $E^*\left(\EG \times_G X\right) \to L_r \otimes_{E^*} E^*\left(\Fix{G,X}\right)$ which lands in the $G$ fixed points.
By \cref{Lr-Fix}, this is the data of a map $E^*\left(\EG \times_G X\right) \to L_r \otimes_{E^*} E^*\left(X^{\im \alpha}\right)$ for each $\alpha \in \Gnp$.

Let $\alpha \in \Gnp$, that is $\alpha: \Lambda_r \to G$.
By functoriality of $\EE$, this induces a map $\EE[\Lambda_r] \to \EG$.
Consider the inclusion $X^{\im \alpha} \to X$.
The multiplication of these maps gives $\EE[\Lambda_r] \times X^{\im \alpha} \to \EG \times X$.
Since $X$ and $\EG$ have a $G$-action, the map $\alpha: \Lambda_r \to G$ induces a $\Lambda_r$-action on them.
Moreover, by definition, this $\Lambda_r$-action restricts to a trivial action on $X^{\im \alpha}$.
We equip both sides with the diagonal $\Lambda_r$-action, which makes the map equivariant, and we get a map between the $\Lambda_r$ orbits.
The $\Lambda_r$ orbits of the source are $\BB[\Lambda_r] \times X^{\im \alpha}$.
Since the action on the target was pulled form the diagonal $G$-action, we can further take the $G$ orbits on the target, to get a map $\BB[\Lambda_r] \times X^{\im \alpha} \to \EG \times_G X$.

Taking $E^*$-cohomology we get $E^*\left(\EG \times_G X\right) \to E^*\left(\BB[\Lambda_r] \times X^{\im \alpha}\right)$.
Since $X$ was assumed to be a finite $G$-CW complex, we have K\"unneth for the target by \cref{E-B-abelian}, so the map is equivalently a map $E^*\left(\EG \times_G X\right) \to E^*\left(\BB[\Lambda_r]\right) \otimes_{E^*} E^*\left(X^{\im \alpha}\right)$.
Using the localization map $E^*\left(\BB[\Lambda_r]\right) \to L_r$ we finally get the desired map $E^*\left(\EG \times_G X\right) \to L_r \otimes_{E^*} E^*\left(X^{\im \alpha}\right)$.
This concludes the construction of the data of the character map.

\begin{proposition}[{\cite[6.9]{HKR}}]
	The map $E^*\left(\EG \times_G X\right) \to L_r \otimes_{E^*} E^*\left(\Fix{G,X}\right)$ constructed above lands in the $G \times \aut \left(\Lambda_r\right)$ fixed points.
\end{proposition}

Therefore, by taking the $G$ fixed points on the target, we indeed get the desired character map $\chinpG: E^*\left(\EG \times_G X\right) \to \cl[n,p]{G,X; L_r}$, which lands in the $\aut \left(\Lambda_r\right)$ fixed points.

\begin{proposition}
	The character maps are compatible with the maps $\cl[n,p]{G,X; L_r} \to \cl[n,p]{G,X; L_{r+1}}$, coming from the maps $L_r \to L_{r+1}$.
	Therefore, we have a character map for $L$, that is $\chinpG: E^*\left(\EG \times_G X\right) \to \cl[n,p]{G,X; L}$, which lands in the $\aut \left(\Lambda\right)$ fixed points.
\end{proposition}

\begin{proof}
	We constructed the character map by constructing a map for each $\alpha$.
	It is easy to see that these maps are compatible with the maps $L_r \to L_{r+1}$, coming from the projections.
\end{proof}

Since $p$ is invertible in $L$ by \cref{Lr-fixed-points}, and $\aut \left(\Lambda\right)$ doesn't change $p^{-1}$, it also follows that after inverting $p$, that is, rationalizing, the map $p^{-1} \chinpG: p^{-1} E^*\left(\EG \times_G X\right) \to \cl[n,p]{G,X; L}$ still lands in the $\aut \left(\Lambda\right)$ fixed points (and the same is true for $L_r$ in place of $L$).

We are now in position to state the main theorem.
This should remind you of \cref{char-1} and \cref{char-2}.

\begin{theorem}[{\cite[Theorem C]{HKR}}]\label{theorem-c}
	First, after tensoring with $L$, the character map
	$\chinpG \otimes L: E^*\left(\EG \times_G X\right) \otimes_{E^*} L \xrightarrow{\sim} \cl[n,p]{G,X; L}$
	becomes an isomorphism.
	Second, the map
	$p^{-1}\chinpG: p^{-1} E^*\left(\EG \times_G X\right) \xrightarrow{\sim} \cl[n,p]{G,X; L}^{\aut \left(\Lambda\right)}$
	is an isomorphism.
	Moreover, these statements hold when $L$ is replaced with $L_r$, for $r \geq r_0$.
\end{theorem}

\begin{corollary}\label{theorem-c-pt}
	Using \cref{combinatorial-cl-fixed}, for the case $X = *$ we get isomorphisms
	$$
	E^*\left(\BG\right) \otimes_{E^*} L
	\cong \cl[n,p]{G; L}
	\cong \prod_{\left[\alpha\right] \in \Gnp/G} L,
	$$
	and
	$$
	p^{-1} E^*\left(\BG\right)
	\cong \cl[n,p]{G; L}^{\aut \left(\Lambda\right)}
	\cong \prod_{\left[\alpha\right] \in \Gnp/{\left(G \times \aut \left(\Lambda\right)\right)}}
	L^{\stab_{\aut \left(\Lambda\right)}\left(\alpha\right)},
	$$
	and these statements hold when $L$ is replaced with $L_r$, for $r \geq r_0$.
\end{corollary}

The first part of the theorem will be proven in the remaining of the section.

\begin{proof}[Proof (of the second part)]
	Consider the isomorphism from the first part,
	$E^*\left(\EG \times_G X\right) \otimes_{E^*} L \xrightarrow{\sim} \cl[n,p]{G,X; L}$.
	Endowing the source with $\aut \left(\Lambda\right)$-action by acting only on $L$, makes it equivariant.
	Therefore, there is an isomorphism on the fixed points.
	Using $L^{\aut \left(\Lambda\right)} = p^{-1} E^*$, from \cref{Lr-fixed-points}, the fixed points on the source are:
	\begin{align*}
		\left(E^*\left(\EG \times_G X\right) \otimes_{E^*} L\right)^{\aut \left(\Lambda\right)}
		&= E^*\left(\EG \times_G X\right) \otimes_{E^*} L^{\aut \left(\Lambda\right)}\\
		&= E^*\left(\EG \times_G X\right) \otimes_{E^*} p^{-1} E^*\\
		&= p^{-1} E^*\left(\EG \times_G X\right)
	\end{align*}
	So indeed $p^{-1} E^*\left(\EG \times_G X\right) \xrightarrow{\sim} \cl[n,p]{G,X; L}^{\aut \left(\Lambda\right)}$ is an isomorphism.
	(The exact same proof works when $L$ is replaced with $L_r$, for $r \geq r_0$.)
\end{proof}



\subsection{The Idea of the Proof and Complex Oriented Descent}

Our next goal is to prove the first part of \cref{theorem-c}.
That is, for a finite group $G$ and a finite $G$-CW complex $X$, the character map becomes an isomorphism after tensoring with $L$, i.e.
$\chinpG \otimes L: E^*\left(\EG \times_G X\right) \otimes_{E^*} L \xrightarrow{\sim} \cl[n,p]{G,X; L}$
is an isomorphism, and the same with $L$ replaced with $L_r$, for $r \geq r_0$.
One may wonder why we had to introduce the $G$-space $X$ into the construction, in order to prove the case of interest, $X = *$.
The reason is, that there is a trick, called \emph{complex oriented descent}, that allows us to reduce to the case of $G$-spaces $X$ with \emph{abelian stabilizers}.
Using this and some further ideas we reduce to the case where $G$ is abelian, and $X = *$.
That is, introducing the space $X$ into the construction, allows us to reduce the statement to abelian groups.

To be more explicit, this is the strategy.
We will consider the character map as a natural transformation between functors of pairs $\left(G, X\right)$, and then we will follow these steps:
\begin{itemize}
	\item Use complex oriented descent to reduce to $X$ with abelian stabilizers,
	\item Use Mayer-Vietoris to reduce to spaces $X = D^n \times G/A$ with $A$ abelian,
	\item Use homotopy invariance to reduce to $X = G/A$,
	\item Use induction to reduce from $\left(G, G/A\right)$ to $\left(A, *\right)$,
	\item Prove for $\left(A, *\right)$.
\end{itemize}

This strategy will be formulated as a theorem later, after we introduce complex oriented descent.
To introduce it, we need some definitions.

\begin{definition}
	Let $\xi$ be a $d$-dimensional complex vector bundle over a space $X$.
	The \emph{flag bundle} $F\left(\xi\right) \to X$ is the bundle of complete flags in $\xi$.
\end{definition}

The fiber over a point can be described as an (ordered) $d$-tuple $\left(\ell_1, \dotsc, \ell_d\right)$ of orthogonal lines.
To define it precisely, we can take the $d$-fold power of the projective bundle $P\left(\xi\right)$, and restrict to the sub-bundle of orthogonal lines.
We note that for a trivial bundle $X \times V \to X$, we have $F\left(X \times V\right) \cong X \times F\left(V\right)$, i.e. the flags are computed fiber-wise.

\begin{definition}
	Let $C^*: \Spaces^\op \to \GrAb$ be a contra-variant functor from spaces to graded abelian groups.
	$C^*$ is said to satisfy \emph{complex oriented descent},
	if for every space $X$ and bundle $\xi$ over $X$, $F$ sends the diagram
	$X \leftarrow F\left(\xi\right) \leftleftarrows F\left(\xi\right) \times_X F\left(\xi\right)$,
	to an equalizer diagram
	$C^*\left(X\right) \to C^*\left(F\left(\xi\right)\right) \rightrightarrows C^*\left(F\left(\xi\right) \times_X F\left(\xi\right)\right)$.
\end{definition}

We note that if $\xi$ is a $G$-vector bundle over the $G$-space $X$, then $F\left(\xi\right) \to X$ is also a $G$-bundle, since $G$ acts unitarily.
We also recall that every finite group $G$ has a faithful finite dimensional complex representation.
The reason complex oriented descent is useful is the following.

\begin{proposition}\label{faithful-triv-bundle}
	Let $X$ be a $G$-space, and let $\rho: G \to V$ be a faithful representations.
	Then the $G$-space $F\left(X \times V\right) \cong X \times F\left(V\right)$ has abelian stabilizer.
\end{proposition}

\begin{proof}
	Let $\left(x, \left(\ell_i\right)\right)$ be a point in $X \times F\left(V\right)$.
	We wish to show that its stabilizer is abelian.
	Let $g,h \in G$ be two elements which fix the point, that is, $g.x = x = h.x$ and $\rho_g\left(\ell_i\right) = \ell_i = \rho_h\left(\ell_i\right)$.
	We see that the linear transformations $\rho_g: V \to V$ and $\rho_h: V \to V$ are simultaneously diagonalizable w.r.t to the decomposition $\left(\ell_i\right)$ of $V$.
	Therefore, by a classical result in linear algebra, they commute, $\rho_g \rho_h = \rho_h \rho_g$, i.e. $\rho_{gh} = \rho_{hg}$.
	Since $\rho$ is faithful, we get that $gh = hg$.
\end{proof}

\begin{definition}
	Let $H \leq G$ be a subgroup, and let $Y$ be an $H$-space.
	We define the $G$-space $G \times_H Y$ as follows.
	Define an $H$-action on $G$ by $h.g = gh^{-1}$.
	This gives a diagonal action on $G \times Y$, i.e. $h.\left(g, y\right) = \left(g h^{-1}, h.y\right)$.
	The orbits are $G \times_H Y$.
	This space has a $G$-action by $\gamma. \left(g, y\right) = \left(\gamma g, y\right)$.
	This is well defined, since
	$
	\gamma. \left[g h^{-1}, h.y\right]
	= \left[\gamma g h^{-1}, h.y\right]
	= \left[\gamma g, y\right]
	= \gamma . \left[g, y\right]
	$.
\end{definition}

\begin{definition}
	Let $\mcl{C}$ be the category whose objects are pairs $\left(G, X\right)$ where $G$ is a finite group and $X$ is a finite $G$-CW complex.
	The morphisms in $\mcl{C}$ are generated from the following:
	First, the usual morphisms $\left(G, X\right) \to \left(G, Y\right)$.
	Second, for $H < G$ and $Y$, we add a morphism $\left(H, Y\right) \to \left(G, G \times_H Y\right)$.
\end{definition}

\begin{definition}
	Let $C^*: \mcl{C}^\op \to \GrAb$ be a contra-variant functor from $\mcl{C}$.
	We define the following properties of $C^*$:
	\begin{itemize}
		\item \emph{Homotopy invariance} - for every $G$, the functor $C^*\left(G, -\right)$ is $G$-homotopy invariant.
		\item \emph{Mayer-Vietories} - for every $G$, $C^*\left(G, -\right)$ satisfies Mayer-Vietoris.
		\item \emph{Complex oriented descent} - for every $G$, $C^*\left(G, -\right)$ satisfies complex oriented descent.
		\item \emph{Induction} - for every, $H \leq G$ and $H$-space $Y$, the morphism $\left(H, Y\right) \to \left(G, G \times_H Y\right)$ induces an isomorphism $C^*\left(G, G \times_H Y\right) \xrightarrow{\sim} C^*\left(H, Y\right)$. $G$.
	\end{itemize}
\end{definition}

\begin{theorem}[{\cite[6.10]{HKR}}]\label{cmplx-oriented-natural-transformation}
	Let $C^*, D^*: \mcl{C}^\op \to \GrAb$ be functors satisfying the above properties.
	Let $\tau: C^* \to D^*$ be a natural transformation between them.
	Suppose that $\tau$ commutes with the connecting morphisms of Mayer-Vietoris, and that $\tau\left(A, *\right)$ is an isomorphism for all abelian groups $A$.
	Then $\tau$ is a natural isomorphism.
\end{theorem}

\begin{proof}
	We will follow the steps of the strategy outlined before (although we will describe it in the opposite order).
	
	Let $G$ be a group, and $A \leq G$ an abelian subgroup.
	The morphism $\left(A, *\right) \to \left(G, G \times_A *\right) \cong \left(G, G/A\right)$, by naturality of $\tau$, induces a commutative square:
	$$
	\begin{tikzcd}
		C^*\left(G, G/A\right) \arrow{r}{} \arrow{d}{\tau\left(G, G/A\right)} & C^*\left(A, *\right) \arrow{d}{\tau\left(A, *\right)} \\
		D^*\left(G, G/A\right) \arrow{r}{} & D^*\left(A, *\right)
	\end{tikzcd}
	$$
	By induction, the horizontal morphisms are isomorphisms.
	By assumption, $\tau\left(A, *\right)$ is an isomorphism.
	We conclude that $\tau\left(G, G/A\right)$ is an isomorphism.
	
	Since $C^*, D^*$ are homotopy invariant, for every disk $D^n$ equipped with a trivial action, the map $\left(G, G/A \times D^n\right) \to \left(G, G/A\right)$ induces an isomorphism.
	Similarly to before, by naturality $\tau\left(G, G/A \times D^n\right)$ is an isomorphism.
	
	Now, let $X$ a finite $G$-CW complex, s.t. the stabilizer of every point is abelian.
	All the cells are then of the form $G/A \times D^n$ for some abelian subgroup $A \leq G$ and disk $D^n$.
	By an induction on the number of cells, using Mayer-Vietoris and the fact that $\tau$ commutes with the connecting morphisms, $\left(G, X\right)$ is an isomorphism.
	
	Lastly, let $X$ be an arbitrary finite $G$-CW complex.
	Let $V$ be a faithful $G$ representation, and consider the bundle $X \times V \to V$.
	By naturality, the diagram $X \leftarrow X \times F\left(V\right) \leftleftarrows X \times F\left(V\right) \times F\left(V\right)$ induces a commutative diagram:
	$$
	\begin{tikzcd}
		C^*\left(X\right) \arrow{r}{f} \arrow{d}{\alpha = \tau\left(G, X\right)}
		& C^*\left(X \times F\left(V\right)\right) \arrow{r}{} \arrow[shift left]{r}{} \arrow{d}{\beta = \tau\left(G, X \times F\left(V\right)\right)}
		& C^*\left(X \times F\left(V\right) \times F\left(V\right)\right) \arrow{d}{\tau\left(G, X \times F\left(V\right) \times F\left(V\right)\right)}
		\\
		D^*\left(X\right) \arrow{r}{g}
		& D^*\left(X \times F\left(V\right)\right) \arrow{r}{} \arrow[shift left]{r}{}
		& D^*\left(X \times F\left(V\right) \times F\left(V\right)\right)
	\end{tikzcd}
	$$
	By complex oriented descent, the two rows are equalizer diagrams.
	By \cref{faithful-triv-bundle}, $X \times F\left(V\right)$ has abelian stabilizers, hence we already know that $\beta = \tau\left(G, X \times F\left(V\right)\right)$ is an isomorphism.
	We can then construct the map $\beta^{-1} g: D^*\left(X\right) \to C^*\left(X \times F\left(V\right)\right)$, which by definitions makes the diagram commute.
	By the universal property of the equalizer, we get a map $\alpha': D^*\left(X\right) \to C^*\left(X\right)$ s.t. the diagram is commutative.
	The composition $\alpha' \alpha: C^*\left(X\right) \to C^*\left(X\right)$ makes the diagram commute, and since $C^*\left(X\right)$ is the equalizer, by uniqueness, $\alpha' \alpha = \id_{C^*\left(X\right)}$.
	Similarly $\alpha \alpha': D^*\left(X\right) \to D^*\left(X\right)$ makes the diagram commute, so $\alpha \alpha' = \id_{D^*\left(X\right)}$.
	This shows that $\alpha = \tau\left(G, X\right)$ is invertible, which completes the proof.
\end{proof}



\subsection{Proof of the Main Theorem}

We are going to use the previous results to prove the main theorem, \cref{theorem-c}.
Almost all of the whole proof will work with $L$ replaced by $L_r$, for $r \geq r_0$, without a change, so we state everything for $L$ except for the end where there is a difference.
Recall that we have already proved the second part.
Therefore, what is left to prove is that 
$\chinpG \otimes L: E^*\left(\EG \times_G X\right) \otimes_{E^*} L \xrightarrow{\sim} \cl[n,p]{G,X; L}$
is an isomorphism.
We will do this using \cref{cmplx-oriented-natural-transformation}.

Denote $C^*\left(G, X\right) = E^*\left(\EG \times_G X\right) \otimes_{E^*} L$,
and $D^*\left(G, X\right) = \cl[n,p]{G,X; L} = \left(L_r \otimes_{E^*} E^*\left(\Fix{G,X}\right)\right)^G$.
Their definition on morphisms $\left(G, X\right) \to \left(G, Y\right)$ is clear, simply by functoriality of all constructions when $G$ is fixed.
The definition on morphisms for induction will be given below, together with the proof that they satisfy induction.
We also denote by $\tau\left(G, X\right)$ the character map $\chinpG \otimes L$ for $X$.

\begin{lemma}
	Both functors $C^*$ and $D^*$ are homotopy invariant.
\end{lemma}

\begin{proof}
	This is immediate since $E^*$ is homotopy invariant, the Borel construction $X \mapsto \EG \times_G X$ is $G$-homotopy invariant, and the fixed points of a $G$-CW complex are also $G$-homotopy invariant.
\end{proof}

\begin{lemma}
	Both functors $C^*$ and $D^*$ satisfy Mayer-Vietoris, and $\tau$ commutes with the connecting morphisms.
\end{lemma}

\begin{proof}
	The Borel construction $X \mapsto \EG \times_G X$ is a limit, and so are fixed points, so they commute with pushouts.
	Therefore, the usual pushouts that induce Mayer-Vietoris, give Mayer-Vietoris for our functors.
	Moreover, the definition makes it clear that the character map commutes with the connecting morphisms.
\end{proof}

\begin{lemma}
	Both functors $C^*$ and $D^*$ satisfy complex oriented descent.
\end{lemma}

\begin{proof}
	Hopkins, Kuhn and Ravenel prove in \cite[2.5]{HKR} that any complex oriented cohomology theory (and not only the cohomology theories of interest to us, namely Lubin-Tate) satisfies complex oriented descent, and we will rely on this result.
	
	Let $\xi$ be a $G$-vector bundle over $X$.
	
	$\EG \times_G \xi$ is a $G$-vector bundle on $\EG \times_G X$, and it satisfies $F\left(\EG \times_G \xi\right) \cong \EG \times_G F\left(\xi\right)$.
	Then the fact that $E^*$ satisfies complex oriented descent, and that $L$ is flat (see \cref{Lr-fixed-points}), imply that $C^*$ satisfies complex oriented descent.
	
	Moreover, in \cite[2.6]{HKR}, they prove that for an abelian subgroup, $A \leq G$, the diagram $X^A \leftarrow F\left(\xi\right)^A \leftleftarrows F\left(\xi\right)^A \times_{X^A} F\left(\xi\right)^A$ gives an equalizer diagram in $E^*$-cohomology.
	In the situation of $D^*$, we use the result for $A = \im \alpha$ which is indeed abelian by the fact they are $n$ commuting elements (equivalently, by the fact that it is the image of an abelian group).
	Equalizers, which are limits, commute with limits, and therefore commute with products and taking $G$-fixed points.
	Using this, and the flatness of $L$ again, we deduce, by \cref{Lr-Fix}, that $D^*$ satisfies complex oriented descent as well.
\end{proof}

\begin{lemma}
	Both functors $C^*$ and $D^*$ satisfy induction.
\end{lemma}

\begin{proof}
	Let $H \leq G$ be a subgroup, and $Y$ an $H$-space.
	
	We have $\EG \times_G \left(G \times_H Y\right) \cong \EE[H] \times_H Y$.
	Taking $E^*$-cohomology and tensoring with $L$ gives an isomorphism $C^*\left(G, G \times_H Y\right) \xrightarrow{\sim} C^*\left(H, Y\right)$, which shows the functoriality for this sort of morphisms, and the fact that it is an isomorphism show that $C^*$ satisfies induction.
	
	We now claim that there is a homeomorphism $\varphi: G \times_H \Fix{H, Y} \xrightarrow{\sim} \Fix{G, G \times_H Y}$.
	
	By definition:
	$$
	G \times_H \Fix{H, Y}
	= G \times_H \coprod_{\alpha \in H_{n,p}} Y^{\im \alpha}
	$$
	An element here is the data of $g \in G$, $\alpha \in H_{n,p}$ and $y \in Y^{\im \alpha}$.
	We will denote its $H$-orbit by $\left[g, \alpha, y\right]$.
	For an $h \in H$, the relation we get is $\left[g, \alpha, y\right] = \left[gh^{-1}, h.\alpha, h.y\right]$.
	An element $\gamma \in G$ acts by $\gamma. \left[g, \alpha, y\right] = \left[\gamma g, \alpha, y\right]$.
	
	Similarly, by definition:
	$$
	\Fix{G, G \times_H Y}
	= \coprod_{\alpha \in G_{n,p}} \left(G \times_H Y\right)^{\im \alpha}
	$$
	An element here is the data of $\alpha \in \Gnp$, $\left[g, y\right] \in \left(G \times_H Y\right)^{\im \alpha}$.
	We will denote this by $\left(\alpha, \left[g, y\right]\right)$.
	For an $h \in H$, the relation we get is $\left(\alpha, \left[g, y\right]\right) = \left(\alpha, \left[gh^{-1}, hy\right]\right)$.
	An element $\gamma \in G$ acts by $\gamma. \left(\alpha, \left[g, y\right]\right) = \left(\gamma. \alpha, \left[\gamma g, y\right]\right)$.
	
	Define the map $\varphi: G \times_H \Fix{H, Y} \to \Fix{G, G \times_H Y}$ by
	$\varphi\left(\left[g, \alpha, y\right]\right) = \left(g. \alpha, \left[g, y\right]\right)$.
	
	We need to show that it doesn't depend on the $H$-orbit representative in the source, and indeed,
	\begin{align*}
		\varphi\left(\left[gh^{-1}, h.\alpha, h.y\right]\right)
		&= \left(gh^{-1}. h.\alpha, \left[gh^{-1}, h.y\right]\right)\\
		&= \left(g. \alpha, \left[gh^{-1}, h.y\right]\right)\\
		&= \left(g. \alpha, \left[g, y\right]\right)\\
		&=\varphi\left(\left[g, \alpha, y\right]\right).
	\end{align*}
	
	We need to show that the element defined lands in the target, $\left(g. \alpha, \left[g, y\right]\right) \in \Fix{G, G \times_H Y}$, i.e. $\left[g, y\right] \in \left(G \times_H Y\right)^{g. \im \alpha}$.
	Since $\alpha \in H_{n,p}$, we have $g. \im \alpha \leq gHg^{-1}$, so let $ghg^{-1} \in \im \alpha$, and we verify that $\left[g, y\right]$ is invariant under it (via the $G$-action).
	Recall that $y \in Y^{\im \alpha}$, so we get,
	$
	ghg^{-1}.\left[g, y\right]
	= \left[ghg^{-1} g, y\right]
	= \left[gh, y\right]
	= \left[g, hy\right]
	= \left[g, y\right].
	$
	
	We show that it is $G$-equivariant.
	So let $\gamma \in G$, and indeed,
	\begin{align*}
		\varphi\left(\gamma. \left[g, \alpha, y\right]\right)
		&= \varphi\left(\left[\gamma g, \alpha, y\right]\right)\\
		&= \left(\gamma g. \alpha, \left[\gamma g, y\right]\right)\\
		&= \gamma. \left(g. \alpha, \left[g, y\right]\right)\\
		&= \gamma. \varphi\left(\left[g, \alpha, y\right]\right)
	\end{align*}
	
	We show that it is one-to-one.
	Assume $\varphi\left(\left[g, \alpha, y\right]\right) = \varphi\left(\left[g', \alpha', y'\right]\right)$,
	i.e. $\left(g. \alpha, \left[g, y\right]\right) = \left(g'. \alpha', \left[g', y'\right]\right)$.
	In particular, $\left[g, y\right] = \left[g', y'\right]$.
	It follows that $y' = h.y$ and $g' = gh^{-1}$ for some $h \in H$.
	Then, $g. \alpha = g'. \alpha' = gh^{-1}. \alpha'$, so $\alpha = h^{-1}. \alpha'$, equivalently $\alpha' = h. \alpha$.
	We therefore conclude that they are indeed in the same $H$-orbit,
	$
	\left[g, \alpha, y\right]
	= \left[gh^{-1}, h.\alpha, h.y\right]
	= \left[g', \alpha', y'\right]
	$.
	
	Lastly, we show that it is surjective, which is the only step which is not routine.
	Let $\left(\alpha, \left[g, y\right]\right)$, i.e. $\alpha \in \Gnp$, $\left[g, y\right] \in \left(G \times_H Y\right)^{\im \alpha}$.
	We claim that if such a triple exists, i.e. the fixed points are not empty, then necessarily $g^{-1}. \alpha \in H_{n,p}$.
	Let $\gamma \in \im g^{-1}. \alpha$.
	Since $g \gamma g^{-1}$ is in $\im \alpha$, it fixes $\left[g, y\right]$, that is
	$
	\left[g, y\right]
	= g \gamma g^{-1}. \left[g, y\right]
	= \left[g \gamma g^{-1} g, y\right]
	= \left[g \gamma, y\right]
	$.
	Therefore, for some $h \in H$,
	$\left(gh^{-1}, hy\right)= \left(g \gamma, y\right)$,
	in particular $\gamma = h^{-1}$, so $\gamma \in H$.
	It follows that $g^{-1}. \alpha \in H_{n,p}$.
	We now claim that $y \in Y^{\im g^{-1}. \alpha}$.
	So let $\eta \in \im g^{-1}. \alpha$, which we now know is in $H$ as well.
	Since $\left[g, y\right] \in \left(G \times_H Y\right)^{\im \alpha}$, it is fixed by $g \eta g^{-1}$, i.e. there is some $h \in H$, s.t.
	$
	\left(gh^{-1}, hy\right)
	= g \eta g^{-1}. \left(g, y\right)
	= \left(g \eta g^{-1} g, y\right)
	= \left(g \eta, y\right)
	$,
	and we get that $y = \eta y$.
	We shows that $y \in Y^{\im g^{-1}. \alpha}$, so the element $\left[g, g^{-1} \alpha, y\right]$ is well defined, and mapped to $\left(\alpha, \left[g, y\right]\right)$.
	
	It is clear that the map $\varphi$ is continuous, and so is its inverse $\left[g, g^{-1} \alpha, y\right] \mapsto \left(\alpha, \left[g, y\right]\right)$, which shows that indeed $D^*$ satisfies induction.
\end{proof}

\begin{lemma}
	$\tau\left(A, *\right)$ is an isomorphism for all abelian groups $A$.
\end{lemma}

\begin{proof}
	We need to verify that
	$
	\chi_{n,p}^A:
	E^*\left(\BB[A]\right) \otimes_{E^*} L
	\to \cl[n,p]{A; L}
	$
	is an isomorphism for abelian groups $A$.
	
	By \cref{E-B-abelian}, we have K\"unneth for $E^*\left(\BB[A]\right)$, so it takes direct sums in $A$ to tensor products.
	
	For an abelian group, the $A$-action on $A_{n,p}$ is trivial (since it is by conjugation).
	Therefore,
	$
	\cl[n,p]{A; L}
	= \hom_{A\Set}\left(A_{n,p}, L\right)
	= \hom_{\Set}\left(A_{n,p}, L\right)
	= \hom_{\Set}\left(\hom_{\Ab}\left(\Lambda, A\right), L\right)
	$.
	The $\hom_{\Ab}\left(\Lambda, A\right)$ commutes with direct sums in the second coordinate, and the outer $\hom_{\Set}$ takes them to tensor product.
	
	Both functors take direct sums to tensor products, so by the structure theorem for abelian groups, we reduce to the case $\ZZ/q^k$ for a prime $q$.
	
	First we handle the case $q \neq p$.
	Again by \cref{E-B-abelian}, $E^*\left(\BB[\ZZ/q^k]\right)$ is a free module of rank $1$, so the source of $\chi_{n,p}^{\ZZ/q^k}$ is $L$.
	Moreover, $\hom_{\Ab}\left(\Lambda, \ZZ/q^k\right)$ has only the trivial homomorphism, so $\cl[n,p]{\ZZ/q^k; L} = L$.
	It is easy to see that this is indeed an isomorphism then.
	
	Lastly, we handle the case $q = p$.
	At this point we have to prove it separately for every $L_r$ with $r \geq r_0$, and the result follows for $L$ as well, by taking the colimit on both sides.
	
	In this case the group is $\ZZ/p^k$, so $r_0 = k$.
	Let $r \geq r_0 = k$.
	Now $\left(\ZZ/p^k\right)_{n,p}$, i.e. $n$ commuting $p$-power-torsion elements, is just $n$ elements, so it is canonically isomorphic to $\left(\ZZ/p^k\right)^n = \Lambda_k$.
	The character map $\chi_{n,p}^{\ZZ/p^k}$ is then a morphism $E^*\left(\BB[\ZZ/p^k]\right) \otimes_{E^*} L_r \to \hom_{\Set}\left(\Lambda_k, L_r\right)$.
	
	By \cref{E-B-abelian}, the source of the character map is isomorphic to $E^*\formal{x}/\left(\left[p^k\right]\left(x\right)\right) \otimes_{E^*} L_r$.
	Using the exponential rule, and the adjunction between the forgetful from $L_r$-modules to $E^*$-modules and $- \otimes_{E^*} L_r$, we get that the character map is the same data as a map
	$\phi: \Lambda_k \to \hom_{L_r}\left(L_r\formal{x}/\left(\left[p^k\right]\left(x\right)\right), L_r\right)$.
	
	In \cite{HKR} the proof ends here, since they have an alternative definition of $L_r$ as a ring representing some functor, which makes the above map an isomorphism almost immediately.
	We didn't take this route, so we make the connection here more explicit.
	
	Recall the definition of $L_r$ from \cref{Lr}, and the definition of the character map.
	Chasing the definitions, we see that for $0 \neq \left(l_i\right) \in \Lambda_k$, $\phi\left(\left(l_i\right)\right)$ is the homomorphism which sends $x$ to $\sum_F \left[l_i\right]\left(x_i\right) \in L_r$.
	Since $r \geq r_0 = k$, the last element is in $S_r$, and in particular invertible in $L_r$.
	We now finish the proof by applying \cite[6.2]{HKR} to $\phi$, which shows that in this case indeed the character map is an isomorphism.
	We remark that in their notation, we use the result for $r = k$ and $R = L_r$.
	Moreover, they denote by $\Lambda_r^*$ the Pontryagin dual, which is isomorphic to $\Lambda_r$, and $\phi\left(\alpha\right)$ in their notation really means the value at $x$ (as can be seen in \cite[5.5]{HKR}).
\end{proof}

\begin{proof}[Proof (of the first part of \cref{theorem-c})]
	Follows immediately by combining the previous lemmas and \cref{cmplx-oriented-natural-transformation}.
\end{proof}
