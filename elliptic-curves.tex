\section{Elliptic Curves}

At this point, one may wonder how we can find interesting pairs $\left(k, \Gamma\right)$, of a perfect field and a formal group law over it, to obtain Lubin-Tate spectra.
Two simple examples which have already seen are the additive formal group law over $\Fp$, of height $\infty$, which gives rise to $\HH{\Fp}$, and the multiplicative formal group law over $\Fp$, of height $1$, which gives rise to $\K_p^\wedge$.
Elliptic curves are another source for formal group laws.



\subsection{Formal Group Laws From Elliptic Curves}

Let $C$ be an elliptic curve over a ring $R$, with $O$ the point at infinity.
In \cite[IV]{Sil}, there is a construction of a formal group from $C$, obtained by considering the infinitesimal neighborhood of $O$.
Choosing coordinates gives coordinates to the formal group, i.e. a formal group law denoted by $\Gamma_C$.

Now, assume that $R = k$ is a finite field of characteristic $p$.
We denote by $C\left[p^r\right]$ the $p^r$-torsion, i.e. the kernel of the multiplication-by-$p^r$ map.
By \cite[IV.7.5]{Sil} and \cite[V.3.1]{Sil}, we have:

\begin{proposition}
	The height of $\Gamma_C$ is either $1$ or $2$.
	Moreover, the height is $2$ if and only if $C\left[p^r\right] = \left\{ O \right\}$ for all $r \geq 1$.
\end{proposition}

In fact, there are many more equivalent conditions to the above, and we make this into a definition.

\begin{definition}
	$C$ is called \emph{supersingular} if $\Gamma_C$ is of height $2$.
\end{definition}

Similarly to the Lubin-Tate deformation theory of formal group laws described in \ref{LT-def}, there is a deformation theory for elliptic curves.
Denote by $C_U$ the universal deformation of $C$, which is an elliptic curve over some ring $R_U$.
This has a corresponding formal group law $\Gamma_{C_U}$.
We then have the following theorem, implied by the Serre-Tate theorem \cite[1.2.1]{ST}.

\begin{theorem}[``$\Gamma_{C_U} = \left({\Gamma_C}\right)_U$"]
	The formal group law $\Gamma_{C_U}$ over $R_U$, is a universal deformation of $\Gamma_C$ over $k$, in the sense of \ref{LT-def}.
\end{theorem}

In this case we get a Lubin-Tate spectrum $E = \E{k, \Gamma_C}$ of height $2$.
We recall from \ref{LT-spectrum} that the coefficients can be taken to be $E_* = \Wk\formal{u_1}\left[u^{\pm 1}\right]$ where $\left|u\right| = 2$, and the formal group law is $u\left(\Gamma_C\right)_U$, which by Serre-Tate can be described by $u \Gamma_{C_U}$.
