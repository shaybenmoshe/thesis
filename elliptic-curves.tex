\section{Elliptic Curves}

At this point, one may wonder how we can find interesting pairs $\left(k, \Gamma\right)$, of a perfect field and a formal group law over it, to obtain Lubin-Tate spectra.
Two simple examples which we have already seen are the additive formal group law over $\Fp$, of height $\infty$, which gives rise to $\HH{\Fp}$, and the multiplicative formal group law over $\Fp$, of height $1$, which gives rise to $\K_p^\wedge$.
Elliptic curves are another source for formal group laws.



\subsection{Formal Group Laws From Elliptic Curves}

Let $C$ be an elliptic curve over a ring $R$, with $O$ the point at infinity.
In \cite[IV]{Sil}, there is a construction of a formal group law $\Gamma_C$, which represents a formal group $\fG_C$, obtained by considering the infinitesimal neighborhood of $O$ in $C$.

Now, assume that $R = k$ is a finite field of characteristic $p$.
We denote by $C\left[p^r\right]$ the $p^r$-torsion, i.e. the kernel of the multiplication-by-$p^r$ map.
By \cite[IV.7.5]{Sil} and \cite[V.3.1]{Sil}, we have:

\begin{proposition}\label{height-torsion}
	The height of $\Gamma_C$ is either $1$ or $2$.
	Moreover, the height is $2$ if and only if the only point of $C\left[p^r\right]$ is $O$ for all $r \geq 1$.
\end{proposition}

In fact, there are many more equivalent conditions to the above, and we make this into a definition.

\begin{definition}
	$C$ is called \emph{supersingular} if $\Gamma_C$ is of height $2$.
\end{definition}

Similarly to the Lubin-Tate deformation theory of formal group laws described in \ref{LT-def}, there is a deformation theory for elliptic curves.
Denote by $C_U$ the universal deformation of $C$, which is an elliptic curve over some ring $R_U$.
This has a corresponding formal group law $\Gamma_{C_U}$ which represents a formal group $\fG_{C_U}$.
We then have the following theorem, implied by the Serre-Tate theorem \cite[1.2.1]{ST}.

\begin{theorem}[``$\Gamma_{C_U} = \left({\Gamma_C}\right)_U$"]\label{serre-tate}
	The formal group law $\Gamma_{C_U}$ over $R_U$, is a universal deformation of $\Gamma_C$ over $k$, in the sense of \ref{LT-def}.
\end{theorem}

In this case we get a Lubin-Tate spectrum $E = \E{k, \Gamma_C}$.
We recall from \ref{LT-spectrum} that the coefficients can be taken to be $E_* = \Wk\formal{u_1}\left[u^{\pm 1}\right]$ where $\left|u\right| = 2$, and the formal group law is $u\left(\Gamma_C\right)_U$, which by Serre-Tate can be described by $u \Gamma_{C_U}$.



\subsection{HKR From Elliptic Curves}

Recall that the our main goal in HKR theory was to compute $p^{-1} E^*\left(\BG\right)$ for some Lubin-Tate spectrum $E$.
We shall focus only the $0$-th level, as this is $2$-periodic in the usual way.
The main result \ref{theorem-c-pt} for us was (stated here only for the $0$-th level, and with $L_r$ for $r \geq r_0$)
$$
p^{-1} E^0\left(\BG\right)
\cong \prod_{\left[\alpha\right] \in \Gnp/{\left(G \times \aut \left(\Lambda_r\right)\right)}}
\left(L_r^0\right)^{\stab_{\aut \left(\Lambda_r\right)}\left(\alpha\right)}.
$$
That is, in order to compute $p^{-1} E^0\left(\BG\right)$, we need to compute $L_r^0$, and various fixed-points sub-rings thereof.

We fix some elliptic curve $C$ over a finite field $k$ of characteristic $p$, of height $n=2$, and consider $E = \E{k, \Gamma_C}$.
In \ref{alg-geo-Lr}, we have seen that $L_r^0$ can be described as follows.
First, $\spec E^0\left(\BB[\Lambda_r]\right) = \left(\fG_{C_U}\left[p^r\right]\right)^2$.
Second, $S_r^0 = \left\{ [k_1]\left(t_1\right) +_{\Gamma_{C_U}} [k_2]\left(t_2\right) \mid \left(k_1, k_2\right) \neq 0 \mod p^r\right\}$.
And $L_r^0 = \left(S_r^0\right)^{-1} E^0\left(\BB[\Lambda_r]\right)$.

Now, since $\fG_C$ is the formal neighborhood of $O$ in $C$, we have a map $\fG_C \to C$.
Since the multiplication on $\fG_C$ comes from the multiplication on $C$, we have the commutative square:
$$
\begin{tikzcd}
	\fG_C \arrow{r}{} \arrow{d}{\left[p^r\right]} & C \arrow{d}{\left[p^r\right]} \\
	\fG_C \arrow{r}{} & C
\end{tikzcd}
$$
Taking the kernels of both vertical maps, we get a map $\fG_C\left[p^r\right] \to C\left[p^r\right]$.
Since $C$ is supersingular, by \ref{height-torsion}, the only point of $C\left[p^r\right]$ is $O$, i.e. it is a nilpotent thickening of the point $O$, which means that the map $\fG_C\left[p^r\right] \to C\left[p^r\right]$ is an isomorphism.

In the same way as above, we have a map between the $p^r$-torsion of the universal deformations, $\fG_{C_U}\left[p^r\right] \to C_U\left[p^r\right]$.
Reducing modulo the maximal ideal, i.e. the map $\Wk\formal{u_1} \to k$, gives the map above $\fG_C\left[p^r\right] \to C\left[p^r\right]$, which is an isomorphism.
By Nakayama's lemma we see that the map $\fG_{C_U}\left[p^r\right] \to C_U\left[p^r\right]$ is also an isomorphism.

This means that in our computations of $L_r^0$, we can use the elliptic curve rather then its formal group law.
This has the advantage that the operations on the elliptic curve are given by polynomials, rather then formal power series.
More explicitly, we have that $\spec E^0\left(\BB[\Lambda_r]\right) \cong \left(C_U\left[p^r\right]\right)^2$, i.e. the scheme-theoretic kernel of the multiplication-by-$p^r$ map on the elliptic curve, squared.
We then need to localize away from the zeros of the functions $[k_1]\left(-\right) +_{C_U} [k_2]\left(-\right)$, for $\left(k_1, k_2\right) \neq 0 \mod p^r$.
As we have seen in \ref{alg-geo-Lr}, we can actually consider only $k_i$'s which are a multiple of $p^{r-1}$, which means that we need to consider only $p^2-1$ pairs.



\subsection{Specific Elliptic Curve}

We now restrict ourselves to a special case.
Take $p = 2$ and $k = \mbb{F}_4$.
We take $C$ to be the elliptic curve given by the Weierstrass equation $y^2 + y = x^3$.
It is supersingular as follows from \cite[exercise V.5.7 combined with proposition A.1.1.c]{Sil}.
Another way to see that is by a direct computation of the terms of the formal group law, which show that the $2$-series $\left[2\right]\left(x\right)$ has $2=0$ as the coefficient of $x$, as in \cite[6.1.4]{Bea}.

Furthermore, $y^2 + u_1 xy + y = x^3$ over $\Wk\formal{u_1} = \ZZ_2\left[\zeta_2\right]\formal{u_1}$ is a universal deformation $C_U$.
This is proven in \cite[3.5]{LT}.
Specifically, there the ring $\ZZ_2\left[\zeta_2\right]$ is denoted by $R$, and $u_1$ by $t$.
It is claimed that the formal group law (up to order 2) is given by $x+y+u_1 xy$.
Then, by \cite[1.1]{LT}, it is the universal deformation, for $C_2 = \frac{1}{2}\left(\left(x+y\right)^2-x^2-y^2\right)=xy$.
\todo{I don't even understand why this means that it is a univ defo of the fgl, and especially not why it is for the elliptic curve...}
