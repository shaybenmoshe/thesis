\section{Elliptic Curves}

At this point, one may wonder how we can find interesting pairs $\left(k, \Gamma\right)$, of a perfect field and a formal group law over it, to obtain Lubin-Tate spectra.
Two simple examples which we have already seen are the additive formal group law over $\Fp$, of height $\infty$, which gives rise to $\HH{\Fp}$, and the multiplicative formal group law over $\Fp$, of height $1$, which gives rise to $\K_p^\wedge$.
Elliptic curves are another source for formal group laws.



\subsection{Formal Group Laws From Elliptic Curves}

Let $C$ be an elliptic curve over a ring $R$, with $O$ the point at infinity.
In \cite[IV]{Sil}, there is a construction of a formal group law $\Gamma_C$, which represents a formal group $\fG_C$, obtained by considering the infinitesimal neighborhood of $O$ in $C$.

Now, assume that $R = k$ is a finite field of characteristic $p$.
We denote by $C\left[p^r\right]$ the $p^r$-torsion, i.e. the kernel of the multiplication-by-$p^r$ map.
By \cite[IV.7.5]{Sil} and \cite[V.3.1]{Sil}, we have:

\begin{proposition}\label{height-torsion}
	The height of $\Gamma_C$ is either $1$ or $2$.
	Moreover, the height is $2$ if and only if the only point of $C\left[p^r\right]$ is $O$ for all $r \geq 1$.
\end{proposition}

In fact, there are many more equivalent conditions to the above, and we make this into a definition.

\begin{definition}
	$C$ is called \emph{supersingular} if $\Gamma_C$ is of height $2$.
\end{definition}

Similarly to the Lubin-Tate deformation theory of formal group laws described in \ref{LT-def}, there is a deformation theory for elliptic curves.
Recall that $\Gamma_C$ over $k$ has a universal deformation.
The Serre-Tate theorem \cite[2.9.1]{KM} then implies the following:

\begin{theorem}\label{serre-tate}
	There exists a deformation $C_U$ over $\Wk\formal{u_1}$ of $C$, whose formal group law $\Gamma_{C_U}$ is a universal deformation of $\Gamma_C$.
\end{theorem}

In this case we get a Lubin-Tate spectrum $E = \E{k, \Gamma_C}$.
We recall from \ref{LT-spectrum} that the coefficients can be taken to be $E_* = \Wk\formal{u_1}\left[u^{\pm 1}\right]$ where $\left|u\right| = 2$, and the formal group law is $u\left(\Gamma_C\right)_U$, which by the above can be described by $u \Gamma_{C_U}$.



\subsection{HKR From Elliptic Curves}

Recall that the our main goal in HKR theory was to compute $p^{-1} E^*\left(\BG\right)$ for some Lubin-Tate spectrum $E$.
We shall focus only the $0$-th level, as this is $2$-periodic in the usual way.
The main result \ref{theorem-c-pt} for us was (stated here only for the $0$-th level, and with $L_r$ for $r \geq r_0$)
$$
p^{-1} E^0\left(\BG\right)
\cong \prod_{\left[\alpha\right] \in \Gnp/{\left(G \times \aut \left(\Lambda_r\right)\right)}}
\left(L_r^0\right)^{\stab_{\aut \left(\Lambda_r\right)}\left(\alpha\right)}.
$$
That is, in order to compute $p^{-1} E^0\left(\BG\right)$, we need to compute $L_r^0$, as defined in \ref{Lr}, and various fixed-points sub-rings thereof.

We fix some elliptic curve $C$ over a finite field $k$ of characteristic $p$, of height $n=2$, and consider $E = \E{k, \Gamma_C}$.
As we saw in \ref{serre-tate}, we can use the deformation $C_U$ to describe the universal deformation of the formal group.
In \ref{alg-geo-Lr}, we have seen that $L_r^0$ can be described as follows.
First, $\spec E^0\left(\BB[\Lambda_r]\right) = \left(\fG_{C_U}\left[p^r\right]\right)^2$.
Second, $S_r^0 = \left\{ [k_1]\left(t_1\right) +_{\Gamma_{C_U}} [k_2]\left(t_2\right) \mid \left(k_1, k_2\right) \neq 0 \mod p^r\right\}$.
And $L_r^0 = \left(S_r^0\right)^{-1} E^0\left(\BB[\Lambda_r]\right)$.

Now, since $\fG_C$ is the formal neighborhood of $O$ in $C$, we have a map $\fG_C \to C$.
Since the multiplication on $\fG_C$ comes from the multiplication on $C$, we have the commutative square:
$$
\begin{tikzcd}
	\fG_C \arrow{r}{} \arrow{d}{\left[p^r\right]} & C \arrow{d}{\left[p^r\right]} \\
	\fG_C \arrow{r}{} & C
\end{tikzcd}
$$
Taking the kernels of both vertical maps, we get a map $\fG_C\left[p^r\right] \to C\left[p^r\right]$.
Since $C$ is supersingular, by \ref{height-torsion}, the only point of $C\left[p^r\right]$ is $O$, i.e. it is a nilpotent thickening of the point $O$, which means that the map $\fG_C\left[p^r\right] \to C\left[p^r\right]$ is an isomorphism.

In the same way as above, we have a map between the $p^r$-torsion of the deformations, $\fG_{C_U}\left[p^r\right] \to C_U\left[p^r\right]$.
Reducing modulo the maximal ideal, i.e. the map $\Wk\formal{u_1} \to k$, gives the map above $\fG_C\left[p^r\right] \to C\left[p^r\right]$, which is an isomorphism.
By Nakayama's lemma we see that the map $\fG_{C_U}\left[p^r\right] \to C_U\left[p^r\right]$ is also an isomorphism.

This means that in our computations of $L_r^0$, we can use the elliptic curve rather then its formal group law.
This has the advantage that the operations on the elliptic curve are given by polynomials, rather then formal power series.
More explicitly, we have that $\spec E^0\left(\BB[\Lambda_r]\right) \cong \left(C_U\left[p^r\right]\right)^2$, i.e. the scheme-theoretic kernel of the multiplication-by-$p^r$ map on the elliptic curve, squared.
We then need to localize away from the zeros of the functions $[k_1]\left(-\right) +_{C_U} [k_2]\left(-\right)$, for $\left(k_1, k_2\right) \neq 0 \mod p^r$.
As we have seen in \ref{alg-geo-Lr}, we can actually consider only $k_i$'s which are a multiple of $p^{r-1}$, which means that we need to consider only $p^2-1$ pairs.



\subsection{Specific Elliptic Curve}

We now restrict ourselves to a special case.
Take $p = 2$ and $k = \mbb{F}_4$.
We take $C$ to be the elliptic curve given by the Weierstrass equation $y^2 + y = x^3$.
It is supersingular as follows from \cite[exercise V.5.7 combined with proposition A.1.1.c]{Sil}.
Another way to see that is by a direct computation of the terms of the formal group law, which show that the $2$-series $\left[2\right]\left(x\right)$ has $2=0$ as the coefficient of $x$, as in \cite[6.1.4]{Bea}.

Furthermore, denote by $C_U$ the elliptic curve given by $y^2 + u_1 xy + y = x^3$ over $\Wk\formal{u_1} = \ZZ_2\left[\zeta_3\right]\formal{u_1}$.
It is clear that modulo $\left(2, \zeta_3\right)$, this reduces to $C$.
Furthermore, in \cite[3.5]{LT}, it is proven that the formal group law of $C_U$ is indeed a universal deformation of that of $C$.
Specifically, there the ring $\ZZ_2\left[\zeta_2\right]$ is denoted by $R$, and $u_1$ by $t$.
It is claimed that the formal group law (up to order 2) is given by $x+y+u_1 xy$.
Then, by \cite[1.1]{LT}, it is the universal deformation, because $C_2 = \frac{1}{2}\left(\left(x+y\right)^2-x^2-y^2\right)=xy$.

Our next goal is to compute the ring $L_r^0$ corresponding to $E = \E{\mbb{F}_4, \Gamma_C}$.
To that end we first need to compute $E^0\left(\BB[\Lambda_r]\right)$, which as we saw is given by $\OO\left(\left(C_U\left[2^r\right]\right)^2\right) = \left(\OO\left(C_U\left[2^r\right]\right)\right)^{\otimes 2}$.
We then need to localize away from the zeros of $[k_1]\left(-\right) +_{C_U} [k_2]\left(-\right)$.
We have only $2^2-1=3$ pairs, which are $\left(2^{r-1},0\right), \left(0,2^{r-1}\right), \left(2^{r-1},2^{r-1}\right)$.
Note that the first two are symmetric, and can be computed even before taking the tensor product.
That is, $L_r^0$ is given by computing $\OO\left(C_U\left[2^r\right]\right)$ (i.e. $2^r$-torsion), localizing away from $[2^{r-1}]\left(-\right)$ (to get the points of order exactly $2^r$), tensoring with itself (to get pairs of such points), and localizing away from $[2^{r-1}]\left(-\right) +_{C_U} [2^{r-1}]\left(-\right)$ (to get such pairs that span).

Furthermore, we recall that by \ref{Lr-fixed-points}, $2$ is invertible in $L_r^0$, so the whole computation can be carried with $C_U$ base changed to $\QQ_2\left[\zeta_3\right]\formal{u_1}$, and we will obtain the same result.
Moreover, we note that the elliptic curve, and all the operations described above, are defined already over $\QQ\left[u_1\right]$, so we can carry the whole computation over $\QQ\left[u_1\right]$, and tensor in the end with $\QQ_2\left[\zeta_3\right]\formal{u_1}$ to get $L_r^0$.
