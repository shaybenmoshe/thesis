\section{Elliptic Curves}

At this point, one may wonder how we can find interesting pairs $\left(k, \Gamma\right)$, of a perfect field and a formal group law over it, to obtain Lubin-Tate spectra.
Two simple examples which we have already seen are the additive formal group law over $\Fp$, of height $\infty$, which gives rise to $\HH{\Fp}$, and the multiplicative formal group law over $\Fp$, of height $1$, which gives rise to $\K_p^\wedge$.
Elliptic curves are another source for formal group laws.



\subsection{Formal Group Laws From Elliptic Curves}

Let $C$ be an elliptic curve over a ring $R$, with $O$ the point at infinity.
In \cite[IV]{Sil}, there is a construction of a formal group law $\Gamma_C$, which represents a formal group $\fG_C$, obtained by considering the infinitesimal neighborhood of $O$ in $C$.

Now, assume that $R = k$ is a finite field of characteristic $p$.
We denote by $C\left[p^r\right]$ the $p^r$-torsion, i.e. the kernel of the multiplication-by-$p^r$ map.
By \cite[IV.7.5]{Sil} and \cite[V.3.1]{Sil}, we have:

\begin{proposition}\label{height-torsion}
	The height of $\Gamma_C$ is either $1$ or $2$.
	Moreover, the height is $2$ if and only if the only point of $C\left[p^r\right]$ is $O$ for all $r \geq 1$.
\end{proposition}

In fact, there are many more equivalent conditions to the above, and we make this into a definition.

\begin{definition}
	$C$ is called \emph{supersingular} if $\Gamma_C$ is of height $2$.
\end{definition}

Similarly to the Lubin-Tate deformation theory of formal group laws described in \ref{LT-def}, there is a deformation theory for elliptic curves.
Recall that $\Gamma_C$ over $k$ has a universal deformation.
The Serre-Tate theorem \cite[2.9.1]{KM} then implies the following:

\begin{theorem}\label{serre-tate}
	There exists a deformation $C_U$ over $\Wk\formal{u_1}$ of $C$, whose formal group law $\Gamma_{C_U}$ is a universal deformation of $\Gamma_C$.
\end{theorem}

In this case we get a Lubin-Tate spectrum $E = \E{k, \Gamma_C}$.
We recall from \ref{LT-spectrum} that the coefficients can be taken to be $E_* = \Wk\formal{u_1}\left[u^{\pm 1}\right]$ where $\left|u\right| = 2$, and the formal group law is $u\left(\Gamma_C\right)_U$, which by the above can be described by $u \Gamma_{C_U}$.



\subsection{HKR From Elliptic Curves}

Recall that the our main goal in HKR theory was to compute $p^{-1} E^*\left(\BG\right)$ for some Lubin-Tate spectrum $E$.
We shall focus only the $0$-th level, as this is $2$-periodic in the usual way.
The main result \ref{theorem-c-pt} for us was (stated here only for the $0$-th level, and with $L_r$ for $r \geq r_0$)
$$
p^{-1} E^0\left(\BG\right)
\cong \prod_{\left[\alpha\right] \in \Gnp/{\left(G \times \aut \left(\Lambda_r\right)\right)}}
\left(L_r^0\right)^{\stab_{\aut \left(\Lambda_r\right)}\left(\alpha\right)}.
$$
That is, in order to compute $p^{-1} E^0\left(\BG\right)$, we need to compute $L_r^0$, as defined in \ref{Lr}, and various fixed-points sub-rings thereof.

We fix some elliptic curve $C$ over a finite field $k$ of characteristic $p$, of height $n=2$, and consider $E = \E{k, \Gamma_C}$.
As we saw in \ref{serre-tate}, we can use the deformation $C_U$ to describe the universal deformation of the formal group.
In \ref{alg-geo-Lr}, we have seen that $L_r^0$ can be described as follows.
First, $\spec E^0\left(\BB[\Lambda_r]\right) = \left(\fG_{C_U}\left[p^r\right]\right)^2$.
Second, $S_r^0 = \left\{ [k_1]\left(t\right) +_{\Gamma_{C_U}} [k_2]\left(s\right) \mid \left(k_1, k_2\right) \neq 0 \mod p^r\right\}$.
And $L_r^0 = \left(S_r^0\right)^{-1} E^0\left(\BB[\Lambda_r]\right)$.

Now, since $\fG_C$ is the formal neighborhood of $O$ in $C$, we have a map $\fG_C \to C$.
Since the multiplication on $\fG_C$ comes from the multiplication on $C$, we have the commutative square:
$$
\begin{tikzcd}
	\fG_C \arrow{r}{} \arrow{d}{\left[p^r\right]} & C \arrow{d}{\left[p^r\right]} \\
	\fG_C \arrow{r}{} & C
\end{tikzcd}
$$
Taking the kernels of both vertical maps, we get a map $\fG_C\left[p^r\right] \to C\left[p^r\right]$.
Since $C$ is supersingular, by \ref{height-torsion}, the only point of $C\left[p^r\right]$ is $O$, i.e. it is a nilpotent thickening of the point $O$, which means that the map $\fG_C\left[p^r\right] \to C\left[p^r\right]$ is an isomorphism.

In the same way as above, we have a map between the $p^r$-torsion of the deformations, $\fG_{C_U}\left[p^r\right] \to C_U\left[p^r\right]$.
Reducing modulo the maximal ideal, i.e. the map $\Wk\formal{u_1} \to k$, gives the map above $\fG_C\left[p^r\right] \to C\left[p^r\right]$, which is an isomorphism.
By Nakayama's lemma we see that the map $\fG_{C_U}\left[p^r\right] \to C_U\left[p^r\right]$ is also an isomorphism.

This means that in our computations of $L_r^0$, we can use the elliptic curve rather then its formal group law.
This has the advantage that the operations on the elliptic curve are given by polynomials, rather then formal power series.
More explicitly, we have that $\spec E^0\left(\BB[\Lambda_r]\right) \cong \left(C_U\left[p^r\right]\right)^2$, i.e. the scheme-theoretic kernel of the multiplication-by-$p^r$ map on the elliptic curve, squared.
We then need to localize away from the zeros of the functions $[k_1]\left(-\right) +_{C_U} [k_2]\left(-\right)$, for $\left(k_1, k_2\right) \neq 0 \mod p^r$.
As we have seen in \ref{alg-geo-Lr}, we can actually consider only $k_i$'s which are a multiple of $p^{r-1}$, which means that we need to consider only $p^2-1$ pairs.



\subsection{Specific Elliptic Curve}

We now restrict ourselves to a special case.
Take $p = 2$ and $k = \mbb{F}_4$.
We take $C$ to be the elliptic curve given by the Weierstrass equation $y^2 + y = x^3$.
It is supersingular as follows from \cite[exercise V.5.7 combined with proposition A.1.1.c]{Sil}.
Another way to see that is by a direct computation of the terms of the formal group law, which show that the $2$-series $\left[2\right]\left(x\right)$ has $2=0$ as the coefficient of $x$, as in \cite[6.1.4]{Bea}.

Furthermore, denote by $C_U$ the elliptic curve given by $y^2 + u_1 xy + y = x^3$ over $\Wk\formal{u_1} = \ZZ_2\left[\zeta_3\right]\formal{u_1}$.
It is clear that modulo $\left(2, \zeta_3\right)$, this reduces to $C$.
Furthermore, in \cite[3.5]{LT}, it is proven that the formal group law of $C_U$ is indeed a universal deformation of that of $C$.
Specifically, there the ring $\ZZ_2\left[\zeta_2\right]$ is denoted by $R$, and $u_1$ by $t$.
It is claimed that the formal group law (up to order 2) is given by $x+y+u_1 xy$.
Then, by \cite[1.1]{LT}, it is the universal deformation, because $C_2 = \frac{1}{2}\left(\left(x+y\right)^2-x^2-y^2\right)=xy$.

Our next goal is to compute the ring $L_r^0$ corresponding to $E = \E{\mbb{F}_4, \Gamma_C}$.
To that end we first need to compute $E^0\left(\BB[\Lambda_r]\right)$, which as we saw is given by $\OO\left(\left(C_U\left[2^r\right]\right)^2\right) = \left(\OO\left(C_U\left[2^r\right]\right)\right)^{\otimes 2}$.
We then need to localize away from the zeros of $[k_1]\left(-\right) +_{C_U} [k_2]\left(-\right)$.
We have only $2^2-1=3$ pairs, which are $\left(2^{r-1},0\right), \left(0,2^{r-1}\right), \left(2^{r-1},2^{r-1}\right)$.
Note that the first two are symmetric, and can be computed even before taking the tensor product.
That is, $L_r^0$ is given by computing $\OO\left(C_U\left[2^r\right]\right)$ (i.e. $2^r$-torsion), localizing away from $[2^{r-1}]\left(-\right)$ (to get the points of order exactly $2^r$), tensoring with itself (to get pairs of such points), and localizing away from $[2^{r-1}]\left(-\right) +_{C_U} [2^{r-1}]\left(-\right)$ (to get such pairs that span).

Furthermore, we recall that by \ref{Lr-fixed-points}, $2$ is invertible in $L_r^0$, so the whole computation can be carried with $C_U$ base changed to $\QQ_2\left[\zeta_3\right]\formal{u_1}$, and we will obtain the same result.
Moreover, we note that the elliptic curve, and all the operations described above, are defined already over $\QQ\left[u_1\right]$, so we can carry the whole computation over $\QQ\left[u_1\right]$, and tensor in the end with $\QQ_2\left[\zeta_3\right]\formal{u_1}$ to get $L_r^0$.



\subsection{Concrete Computations}

We describe a way to do the calculation described above, namely over $\QQ\left[u_1\right]$.
We will display this along the Macaulay 2 code, that carries out the computation.

The basic operation on the elliptic curve are developed in projective coordinates.
Therefore, we homogenize the elliptic curve above to $X^3 = Y^2 Z + u_1 X Y Z + Y Z^2$.
When we need to compute the ring, we will work with the affine patch where $Y = 1$, which contains the origin $O = \left[0;1;0\right]$.
Note that in this patch we remove exactly one points from the patch, for if $Y = 0$, then $X^3 = 0 + 0 + 0 = 0$, i.e. $X = 0$, so we remove only the point $\left[0;0;1\right]$.
This point will be useful to give a computation of the addition map, and more specifically for the multiplication-by-$2^r$ map.

The code is given below, and what follows is an explanation of the code.

\subsubsection{General Remarks}

We will use in the code matrices (rather then other data types which store a list of values), as Macaulay 2 has the best support for matrices.

\subsubsection{Basic Objects}

We define the basic objects concerning the computation.
Note that \texttt{r = 2} can be replaced in principal by any value.	
Moreover, we can change \texttt{R = QQ[u1]} to \texttt{QQ} or even \texttt{GF(p)} if we want to work over them, and then we also need to remove the \texttt{u1 * X * Y * Z} term from \texttt{F}.

In the code \texttt{RemovedP} is the point described above which is not in the affine patch we will be using ($Y = 1$).

\subsubsection{Util Functions}

The function \texttt{DivideGcd} has a matrix as its input, which will always be a list of polynomials, and outputs the matrix divided by the $\GCD$ of all of the elements.
The function \texttt{comp} receives two matrices of polynomials, and computes the composition by substituting the variables.
It also divides by the $\GCD$, which does not affect the function in projective coordinates, but is essential in some instances to get the correct affine functions.

\subsubsection{Functions on the Elliptic Curve}

The first function calculated is what we call \texttt{Star}.
This operation is used to define the addition (as explained below), and it satisfies $P_1 \star P_2 = - \left(P_1 + P_2\right)$, in other words, $\left(P_1 \star P_2\right) + P_1 + P_2 = O$.
Geometrically, $P_1 \star P_2$ is the third intersection point of the line through $P_1$ and $P_2$ and the elliptic curve.
Equivalently, projectively, the line is given by $t P_1 + s P_2$, and we are looking for the places where $F\left(t P_1 + s P_2\right) = 0$.
The two trivial solutions are where $t = 0$ or $s = 0$.
We think of $F\left(t P_1 + s P_2\right)$ as a polynomial in $t,s$.
Since $F$ is homogenus of degree $3$, all terms will have total degree $3$.
The cubic terms $t^3$ and $s^3$ are then precisely those that are in $F\left(t P_1\right) + F\left(s P_2\right)$.
We assumed that $P_1,P_2$ are on the curve, so we can subtract this, and look for the solution for the solution of $F\left(t P_1 + s P_2\right) - F\left(t P_1\right) - F\left(s P_2\right)$.
This is now a homogenus polynomial of degree $3$ without $t^3$ and $s^3$, that is, it is of the form $t s \left(c_t t + c_s s\right)$.
Therefore a solution is for $t = -c_s$ and $s = c_t$, i.e. $P_1 \star P_2 = -c_s P_1 + c_t P_2$.

Now, taking $P_1 = P, P_2 = O$ we get $\left(P \star O\right) + P + O = O$, so $P \star O = -P$.
This explains the introduction of \texttt{Neg}.

Now that we have \texttt{Star} and \texttt{Neg} we can define the addition by $-\left(P_1 \star P_2\right)$.
We call this \texttt{AddCalc}, as this is not quiet the addition function we will be using.
This function has a problem, it vanishes on the diagonal $P_1 = P_2$, which is precisely what we need to multiply by $2$.
The reason is that already the function we denoted by \texttt{Star} vanishes on the diagonal, because when $P_1 = P_2$, the line through the points $t P_1 + s P_2$ is singular and we just get the point.
However, this function is defined everywhere else.
Luckily we can use the following trick, $P_1 + P_2 = \left(P_1 - Q\right) + \left(P_2 + Q\right)$.
Specifically, we take $Q$ to be the removed point $\left[0;0;1\right]$.
\todo{why does this actually work then? do we use the fact that we only care about points in the formal neighborhood of $O$?}

At this point we can introduce \texttt{Mul2} simply by $P+P$.
We further define the function \texttt{MulNTwoDiv} which computes $\left[n\right]$ by expanding $n$ in binary form (if it is divisible by $2$, then $\left[n\right] = \left[2\right]\left[\frac{n}{2}\right]$, otherwise $\left[n\right] = \left[n-1\right]+\id$).
This gives us $\left[p^{r-1}\right],\left[p^r\right]$ denoted by \texttt{Mulprm1} and \texttt{Mulpr} respectively.

\subsubsection{The Code}

\begin{lstlisting}
-------- Basic Objects --------

p = 2;
r = 2;
R = QQ[u1];

A3 = R[X,Y,Z];
A33 = R[X1,Y1,Z1,X2,Y2,Z2];
A332 = A33[t,s];

F = X^3 - (Y^2 * Z + u1 * X * Y * Z + Y * Z^2);

Mt = matrix{{t}};
Ms = matrix{{s}};

O = matrix{{0, 1, 0}};
P = matrix{{X, Y, Z}};
P1 = matrix{{X1, Y1, Z1}};
P2 = matrix{{X2, Y2, Z2}};
RemovedP = matrix{{0, 0, 1}};



-------- Util Functions --------

DivideGcd = mat -> (
	g := gcd flatten entries mat;
	matrix {(flatten entries mat) // g}
);

comp = (a, b) -> (
	DivideGcd(sub(a, b))
);



-------- Functions on the Elliptic Curve --------

StarCalc = sub(F, Mt * P1 + Ms * P2) - sub(F, Mt * P1) - sub(F, Ms * P2);
StarCt = StarCalc_(t^2 * s);
StarCs = StarCalc_(t * s^2);
Star = -StarCs * P1 + StarCt * P2;

Neg = comp(Star, P|O);

AddCalc = comp(Neg, sub(Star, P1|P2));
MovedP1 = comp(AddCalc, P1|sub(Neg, RemovedP));
MovedP2 = comp(AddCalc, P2|RemovedP);
Add = comp(AddCalc, MovedP1|MovedP2);

Mul2 = comp(Add, P|P);

MulNTwoDiv = n -> (
	if n == 1 then P
	else if n % 2 == 0 then comp(Mul2, MulNTwoDiv(n / 2))
	else comp(Add, P|MulNTwoDiv(n-1))
);

Mulprm1 = MulNTwoDiv(p^(r-1));

Mulpr = MulNTwoDiv(p^r);



-------- Exactly p^r Torsion --------

BasePr = R[x,z,d, MonomialOrder=>{Weights => {1, 1000 ,1000000}, Lex}];
p0 = matrix{{x, 1, z}};

quoPr = {};

quoPr = append(quoPr, sub(F, p0));

mulpr = sub(Mulpr, p0);
quoPr = append(quoPr, mulpr_(0,0));
quoPr = append(quoPr, mulpr_(0,2));

mulprm1 = sub(Mulprm1, p0);

quoPr = append(quoPr, 1 - d * (mulprm1_(0,0) + mulprm1_(0,2)));

IPr = ideal quoPr;

gbIPr = flatten entries gens gb IPr;



-------- Spanning Pairs --------

BaseLr0 = R[x1,z1,x2,z2,d1,d2,d3, MonomialOrder=>{Weights => {1,1000,1,1000,1000000,1000000,1000000}, Lex}];
p01d = matrix{{x1, z1, d1}};
p02d = matrix{{x2, z2, d2}};
p1 = matrix{{x1, 1, z1}};
p2 = matrix{{x2, 1, z2}};

quo = {};

quo = quo | (flatten entries sub(matrix{gbIPr}, p01d));
quo = quo | (flatten entries sub(matrix{gbIPr}, p02d));

Mulprm1 = MulNTwoDiv(p^(r-1));

mulprm11 = sub(Mulprm1, p1);
mulprm12 = sub(Mulprm1, p2);
addMulprm11and2 = sub(Add, mulprm11|mulprm12)
quo = append(quo, 1 - d3 * (addMulprm11and2_(0,0) + addMulprm11and2_(0,2)));

I = ideal quo
\end{lstlisting}
Here \texttt{A3} is where affine patches of the elliptic curve will be defined.
