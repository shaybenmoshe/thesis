\section{Atiyah-Segal}

We now leave the realm of chromatic homotopy theory.
One aspect of algebraic topology is to try to capture properties of spaces using algebraic invariants.
One of the most fruitful such invariants is complex K-theory $\K$, and one of the most important spaces in homotopy theory is $\BG$, so it is natural to ask for a description of $\K\left(\BG\right)$ (by Bott periodicity, we will consider only $\K = \K^0$).
Atiyah and Segal \cite{AS} gave a description of this, and more, in the case that $G$ is a compact Lie group, in terms of representations.

From now we fix a compact Lie group $G$.
Also, a representation means a unitary representation.

\subsection{The Atiyah-Segal Theorem}

We denote by $\RG$ the \emph{representation ring} of $G$, that is the collection of virtual representations of $G$ (which can be written is as a formal difference $V - U$) where the addition is given by direct sum and the product is given by tensor product.
This is an augmented ring $\varepsilon: \RG \to \ZZ$ by the virtual dimension (i.e. $\varepsilon\left(V-U\right) = \dim\left(V-U\right) = \dim V - \dim U$).
The \emph{augmentation ideal} is $I = \ker \varepsilon = \left\{ V - U \in \RG \mid 0 = \dim(V-U) \right\}$.

\begin{theorem}[{\cite{AS}}]
	$\K\left(\BG\right) \cong \RGI$.
\end{theorem}

We will not prove the theorem, but we will indicate some of the key ingredients.

First of all, to show that objects are isomorphic, we need a map.
Before giving the map actually used in the proof, we describe an easier way to see where this map comes from.
Recall that $\K\left(X\right) \cong \left[X, \BU \times \ZZ\right]$.
We write $\BU \to \BU \times \ZZ$ for the map that is the identity on the first coordinate and $0$ on the second.
The data of a representation of $G$ is the same thing as a homomorphism $G \to \UU$.
Since $\BB$ is a functor, we get a map $\BG \to \BU$, and by composing with $\BU \to \BU \times \ZZ$, we indeed get a map $\BG \to \BU \times \ZZ$, that is, and element of $\K\left(\BG\right)$.
Therefore we get a map $\RG \to \K\left(\BG\right)$.
The theorem is that it is a ring homomorphism which exhibits $\K\left(\BG\right)$ as the completion of $\RG$ at $I$.

There is an alternative description of this map.
In \cite{Seg}, Segal described equivariant K-theory $\KG$.
This is a variant of K-theory, which assigns to a $G$-space the ring of virtual $G$-bundles, that is bundles equipped with an action of $G$, compatible with the action on the base $G$-space.
Note that this is no longer a homotopy invariant, since it also takes into account the action of $G$.
First we note the following:

\begin{proposition}
	$\KG\left(*\right) = \RG$ (where $*$ denotes the trivial $G$-space)
\end{proposition}

\begin{proof}
	This is by definition, since a vector bundle over a point is just a vector space, and it is equipped with a $G$ action.
\end{proof}

For any $G$-space $X$, the projection map $\mrm{pr}: X \to X/G$ allows us to pullback vector bundles on $X/G$, to $G$-bundles on $X$, that is it induces a map $\mrm{pr}^*: \K\left(X/G\right) \to \KG\left(X\right)$.

\begin{proposition}[{\cite[2.1]{Seg}}]
	If $G$ acts \emph{freely} on $X$, then there is an inverse to $\mrm{pr}^*$, so $\K\left(X/G\right) \cong \KG\left(X\right)$.
	The inverse is given by taking a bundle $E \to X$ to $E/G \to X/G$.
\end{proposition}

Now, we have a map of $G$-spaces given by $\EG \to *$.
By the above we get:
$$
\RG
\cong \KG\left(*\right)
\to \KG\left(\EG\right)
\cong \K\left(\EG/G\right)
= \K\left(\BG\right)
$$
This is again the map we need, which exhibits $\K\left(\BG\right)$ as the $I$-completion of $\RG$.
Atiyah and Segal use this map and variants to show the theorem.
