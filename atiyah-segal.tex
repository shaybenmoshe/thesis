\section{Atiyah-Segal}

We now leave the realm of chromatic homotopy theory.
One aspect of algebraic topology is to try to capture properties of spaces using algebraic invariants.
One of the most fruitful such invariants is complex K-theory, denoted $\K$, and one of the most important spaces in homotopy theory is $\BG$, so it is natural to ask for a description of $\K\left(\BG\right)$ (by Bott periodicity, we will consider only $\K = \K^0$).
Atiyah and Segal \cite{AS} gave a description of this, and more, in the case that $G$ is a compact Lie group, in terms of representations.

From now we fix a compact Lie group $G$.
Also, a representation means a finite dimensional unitary representation.
We should also note that beyond this part, we will be mostly interested in finite groups.



\subsection{The Atiyah-Segal Theorem}

We denote by $\RG$ the \emph{representation ring} of $G$, that is the collection of virtual representations of $G$ (which can be written as a formal difference $V - U$) up to isomorphism, where the addition is given by direct sum and the product is given by tensor product.
This is an augmented ring $\varepsilon: \RG \to \ZZ$ by the virtual dimension (i.e. $\varepsilon\left(V-U\right) = \dim\left(V-U\right) = \dim V - \dim U$).
The \emph{augmentation ideal} is $I = \ker \varepsilon = \left\{ V - U \in \RG \mid 0 = \dim(V-U) \right\}$.

Atiyah and Segal showed that one can describe $\K\left(\BG\right)$ in these terms, namely, it is the completion of $\RG$ at the ideal $I$:

\begin{theorem}[{\cite{AS}}]\label{AS-private}
	$\K\left(\BG\right) \cong \RGI$.
\end{theorem}

We will not prove the theorem, but we will indicate some of the key ingredients.

First of all, to show that objects are isomorphic, we need a map.
Before giving the map actually used in the proof, we describe an easier way to see where this map comes from.
Recall that $\K\left(X\right) \cong \left[X, \BU \times \ZZ\right]$.
The data of a representation of $G$ is the same thing as a homomorphism $G \to \UU[n]$.
Since $\BB$ is a functor, we get a map $\BG \to \BU[n]$, and by composing with the injection $\BU[n] \cong \BU[n] \times \left\{n\right\} \to \BU \times \ZZ$, we get a map $\BG \to \BU \times \ZZ$, that is, an element of $\K\left(\BG\right)$.
Therefore we get a map $\RG \to \K\left(\BG\right)$.
The theorem shows that it is a ring homomorphism which exhibits $\K\left(\BG\right)$ as the completion of $\RG$ at $I$.

There is an alternative description of this map.
In \cite{Seg}, Segal described a variant of$\K$ theory, called equivariant K-theory $\KG$.
This variant assigns to a $G$-space the ring of virtual $G$-bundles, that is, bundles equipped with an action of $G$ which is compatible with the action on the base $G$-space.
Note that $\K_G$ is no longer homotopy invariant, since it also takes into account the action of $G$.
First we note the following:

\begin{proposition}
	$\KG\left(*\right) = \RG$ (where $*$ is equipped with a trivial $G$-action).
\end{proposition}

\begin{proof}
	This follows from the definitions, since a vector bundle over a point is just a vector space, and it is equipped with a $G$-action over the point, which is just a $G$ representation.
\end{proof}

For any $G$-space $X$, the projection map $\mrm{pr}: X \to X/G$ allows us to pullback vector bundles on $X/G$ to $G$-bundles on $X$.
In other words, it induces a map $\mrm{pr}^*: \K\left(X/G\right) \to \KG\left(X\right)$.

\begin{proposition}[{\cite[2.1]{Seg}}]
	Suppose the action of $G$ on $X$ is \emph{free}. Then $\mrm{pr}^*$ admits an inverse, given by taking a bundle $E \to X$ to $E/G \to X/G$.
	In particular, $\K\left(X/G\right) \cong \KG\left(X\right)$.
\end{proposition}

Now, we have a map of $G$-spaces given by $\EG \to *$.
By the above we get:
$$
\RG
\cong \KG\left(*\right)
\to \KG\left(\EG\right)
\cong \K\left(\EG/G\right)
= \K\left(\BG\right)
$$
It can be shown that this is the same map $\RG \to \K\left(\BG\right)$ described before, which exhibits $\K\left(\BG\right)$ as the $I$-completion of $\RG$.
Atiyah and Segal use this map and variants to prove the theorem.

Here is a sketch of the proof given by Atiyah and Segal.
First of all, we note that the theorem is proven for the entire $\K^*$ rather than just for $\K = \K^0$.
Also note that $\RgG = \KG^*\left(*\right)$ is a $2$-periodic version of the representation ring (because $\KG^*$ also satisfies Bott periodicity).
We have the corresponding $2$-periodic version of the augmentation ideal, which we denote by $I^*$.
They use the Milnor join construction $\EG_n = \underbrace{G * \cdots * G}_{n \text{ times}}$ and $\BG_n = \EG_n/G$, which has the property that $\colim \EG_n \to \colim \BG_n$ is a model for $\EG \to \BG$.
Then, for any compact $G$-space $X$ there is a similar map to the map above: using $X \times \EG_n \to X$ we get a map $\KG^*\left(X\right) \to \KG^*\left(X \times \EG_n\right)$.
All of these are $\RgG$-modules, and Atiyah and Segal show that this map factors through the quotient by $\left(I^*\right)^n$, to give a map $\KG^*\left(X\right)/\left(I^*\right)^n \to \KG^*\left(X \times \EG_n\right)$.
The two sides assemble into pro-rings, and the maps assemble to a map between the pro-rings:
$$
\left\{\KG^*\left(X\right)/\left(I^*\right)^n\right\}_n
\to \left\{\KG^*\left(X \times \EG_n\right)\right\}_n
$$
What they actually prove is the strong form:

\begin{theorem}[{\cite{AS}}]\label{AS-full}
	If $\KG^*\left(X\right)$ is finite over $\RgG$, then the above map of pro-rings is an isomorphism.
\end{theorem}

Their proof has another interesting aspect.
Although it is a statement about the $\K_G$ of some class of $G$-spaces, for one specific group $G$, their proof involves the $\K_G$'s of several groups.
In particular, to prove the result for example for a finite group, their proof involves more general compact Lie groups.
The proof consists of four steps.
In every step we show the theorem holds for a more general type of group:
\begin{itemize}
	\item $G = \UU[1]$ (circle group),
	\item $G = \UU[1]^n$ (torus group),
	\item $G = \UU[n]$,
	\item $G$ a general compact Lie group (this step is proven by $G$ embedding in $\UU[n]$).
\end{itemize}

We note that the first formulation of the Atiyah-Segal theorem \ref{AS-private}, is indeed a private case of the second formulation \ref{AS-full}.
Take the case $X = *$.
By definition, $\KG^*\left(*\right)$ is finite over $\RgG = \KG^*\left(*\right)$, so \ref{AS-full} holds and we have an isomorphism of pro-rings $\left\{\KG^*\left(*\right)/\left(I^*\right)^n\right\}_n \to \left\{\KG^*\left(\EG_n\right)\right\}_n$.
In particular, after computing the limits $\lim: \pro \Ring \to \Ring$, we get an isomorphism of rings $\KG^*\left(*\right)_{I^*}^\wedge \to \KG^*\left(\EG\right)$.
Taking only the $0$-th cohomology gives the desired isomorphism:
$$
\RG_I^\wedge
\cong \KG\left(*\right)_I^\wedge
\to \KG\left(\EG\right)
\cong \K\left(\EG/G\right)
= \K\left(\BG\right)
$$




\subsection{Examples}

We compute a few examples in detail, to make the isomorphism more vivid.

\subsubsection{\texorpdfstring{$\UU[1]$}{U(1)}, the Circle Group}

Take $G = \UU[1]$, the circle group.
It is known that the irreducible representations are of dimension $1$ and labeled by an integer $m \in \ZZ$, i.e. they are homomorphisms $\rho_m = \UU[1] \to \UU[1]$ given by $\rho_m\left(e^{i \theta}\right) = e^{m i \theta}$.
In particular, $\rho_0 = 1$ is the trivial representation.
It is then clear that for $m \geq 0$, $\rho_1^{\otimes_m} = \rho_m$ and $\rho_{-1}^{\otimes_{-m}} = \rho_{-m}$.
Therefore the representation ring generated under (virtual) direct sums and tensor products by $\rho_1$ and $\rho_{-1}$.
Moreover, $\rho_1 \otimes \rho_{-1} = 1$.
Therefore we conclude that $\RG[\UU[1]] = \ZZ\left[\rho, \rho^{-1}\right]$.

The augmentation map is the homomorphism $\varepsilon: \RG[\UU[1]] \to \ZZ$ which sends $1,\rho$ and $\rho^{-1}$ to $1$.
Recall that the augmentation ideal is $I = \ker \varepsilon$.
We set $t = \rho-1$, which clearly belongs to $I$.
We can also write then $\RG[\UU[1]] = \ZZ\left[t, \left(1+t\right)^{-1}\right]$.
Note that $\varepsilon$ factors to a map $\RG[\UU[1]]/\left(t\right) \to \ZZ$, which is already an isomorphism, so by the first isomorphism theorem indeed $I = \left(t\right)$.

We compute the completion $\RGI[{\UU[1]}]$.
Note that in $\ZZ\left[t\right]/t^n$, $1+t$ is already invertible.
The reason is that the formal power series for the inverse is finite since large enough powers of $t$ are zero, $\frac{1}{1-\left(-t\right)} = \sum_{m=0}^{n-1} \left(-t\right)^m$ is an inverse.
Therefore we see that $\RG[\UU[1]]/I^n \cong \ZZ\left[t\right]/t^n$, and clearly the maps the in the limit diagram send $t$ to $t$.
We get that $\RGI[{\UU[1]}] = \lim \ZZ\left[t\right]/t^n \cong \ZZ\left[\left[t\right]\right]$.

In \cite[proposition 2.24]{VB}, it is shown that $\K\left(\CP{n}\right) \cong \ZZ\left[L\right] / \left(L-1\right)^{n+1}$, where $L$ is the canonical line bundle on $\CP{n}$.
In \ref{k-thy-oriented} we denoted $t = L-1$ (warning: there we looked at $\K^*$, now we focus on $\K$), which allows us to rewrite this as $\K\left(\CP{n}\right) \cong \ZZ\left[t\right] / t^{n+1}$.
As we noted, the limit is $\K\left(\CP{\infty}\right) \cong \ZZ\left[\left[t\right]\right]$.
We thus see that $\K\left(\BU[1]\right) \cong \ZZ\left[\left[t\right]\right]$ where $t = L - 1$ is the canonical line bundle minus $1$.

The identity map $\rho: \UU[1] \to \UU[1]$, is mapped to the identity $\BU[1] \to \BU[1]$ (by functoriality of $\BB$), which tautologically corresponds the universal line bundle $L$ on $\BU[1]$.
We therefore see that the Atiyah-Segal map $\RG[\UU[1]] \to \K\left(\BU[1]\right)$, sends $\rho$ to $L$, and therefore $t = \rho - 1$ to $t = L - 1$.
Which indeed shows that the map admits $\K\left(\BU[1]\right) \cong \ZZ\left[\left[t\right]\right]$ as the $I = \left(t\right)$-completion of $\RG[\UU[1]]$.


\subsubsection{\texorpdfstring{$\ZZ/2$}{Z/2}, Cyclic Group of Order \texorpdfstring{$2$}{2}}

Take $G = \ZZ/2$.
Here we have only two irreducible representations, the trivial, and $\rho\left(0\right) = 1, \rho\left(1\right) = -1$.
Also, it is clear that $\rho \otimes \rho$ is the trivial representation.
Therefore, $\RG[\ZZ/2] = \ZZ[\rho]/\left(\rho^2-1\right)$.
Similarly to before, the augmentation $\varepsilon: \RG[\ZZ/2] \to \ZZ$ sends $1,\rho$ to $1$, so clearly $\left(\rho-1\right) \subseteq I$, and for the same reasoning as in the previous example this is actually an equality.
We change coordinates to $t = \rho-1$, and we have $\RG[\ZZ/2] = \ZZ[t]/\left(t^2+2t\right)$, and $I = \left(t\right)$.

We move to computing the completion $\RG[\ZZ/2]_I^\wedge$.
Modulo $t^2+2t$, i.e. $t^2 = -2t$, we have that $t^n = \left(-2\right)^{n-1} t$.
Thus $I^n = \left(\left(-2\right)^{n-1} t\right) = \left(2^{n-1} t\right)$, so $\RG[\ZZ/2]/I^n = \ZZ[t]/\left(t^2+2t, 2^{n-1} t\right)$.
We first compute the limit of $\RG[\ZZ/2]/I^n$ in abelian groups.
Since the forgetful functor from rings to abelian groups is right adjoint, it commutes with limits, so this will give us the abelian group structure.
As an abelian group, $\RG[\ZZ/2]/I^n$ is isomorphic to $\ZZ \oplus \ZZ/2^{n-1}\left\{t\right\}$.
It is then clear that as an abelian group, $\lim \RG[\ZZ/2]/I^n$ is isomorphic to $\ZZ \oplus \ZZ_2\left\{t\right\}$.

We now define a multiplication on $\ZZ \oplus \ZZ_2\left\{t\right\}$ abelian group, given by $\left(a+bt\right) * \left(c+dt\right) = ac + (ad+bc-2bd)t$.
It can be checked that it is associative and commutative.
We have homomorphisms of groups $\ZZ \oplus \ZZ_2\left\{t\right\} \to \ZZ[t]/\left(t^2+2t, 2^{n-1} t\right)$, admitting it as the limit in groups, which are given by sending $a+bt$ to $a+\left(b \mod 2^{n-1}\right) t$.
By construction this homomorphism is actually a homomorphism of rings (the $-2bdt$ term is explained by the relation $t^2+2t$).
Therefore, by the universal property of the limit, we get a map $\ZZ \oplus \ZZ_2\left\{t\right\} \to \lim \RG[\ZZ/2]/I^n$ in rings.
After taking the forgetful we know that it becomes an isomorphism, but the forgetful reflects isomorphisms, so this is also an isomorphism in rings.

Using the Atiyah-Segal theorem we conclude that
$$
\K\left({\mbb{R}\mrm{P}}^\infty\right)
= \K\left(\BB{\ZZ/2}\right)
\cong \ZZ \oplus \ZZ_2\left\{t\right\},
$$
with multiplication given by $\left(a+bt\right) * \left(c+dt\right) = ac + (ad+bc-2bd)t$.




\subsection{Recollections from Character Theory}

We restrict ourselves to the case of finite groups $G$.
We recall that representations of groups can be studied by their characters.
Specifically the character map $\chi: \RG \to \ZZ\left[\chi_{\rho_i}\right]$, defined by $\chi_\rho = \tr \rho$, is an isomorphism, where the ring on the right is the ring of functions generated by the irreducible characters (the multiplication of two characters is a character so it is indeed closed under multiplication).

We also recall that characters are class functions, that is, they are constant on conjugacy classes.
Let $L$ be some field containing all the values of all characters.
Then a natural place to study characters is in the ring of class functions with values in $L$, denote by $\cl{G;L}$.
Let us phrase this in a way that will be useful in the next section.
$G$ is equipped with a $G$-action by conjugation, $\gamma.g = \gamma g \gamma^{-1}$.
Equip $L$ with the trivial $G$-action.
Then $\cl{G;L} = \hom_{\GSet}\left(G, L\right)$.

We can of course extend the range of the character map to get an injection $\chi: \RG \to \cl{G;L}$.
The first classical theorem regarding the relationship between characters and class functions is:

\begin{theorem}\label{char-1}
	After tensoring with $L$, the character map $\chi \otimes L: \RG \otimes L \to \cl{G; L}$ becomes an isomorphism.
\end{theorem}

\begin{proof}
	Similarly to the proof in \cite[9.1]{Ser} for $L = \mbb{C}$, we can view $\cl{G;L}$ as a vector space over $L$, and the characters are linearly independent, so by counting them we see that the image of $\chi \otimes L$ has the dimension of the whole vector space and we are done.
\end{proof}

By definition the value of a character is the trace of a linear transformation $\chi_\rho\left(g\right) = \tr \rho\left(g\right) = \sum \lambda_i$ where $\lambda_i$ are the eigenvalues (which exist since the representation is unitary).
Since $g^{\left|G\right|} = e$, we get $\rho\left(g^{\left|G\right|}\right) = \rho\left(e\right) = \id$, but then we get that the eigenvalues of $\rho\left(g^{\left|G\right|}\right)$ are on the one hand $\lambda_i^{\left|G\right|}$ and on the other hand they are all $1$, which means that all the eigenvalues are roots of unity.
Therefore $L = \mbb{Q}^\mrm{ab} = \mbb{Q}\left(\zeta_\infty\right)$ is always a valid choice for $L$ (regardless of $G$).
To be concrete, we will take this choice.

The Galois group of $\mbb{Q}^\mrm{ab}$ is $\Gal\left(\Qab / \mbb{Q}\right) \cong {\hat\ZZ}^\times$.
For every $m \in {\hat\ZZ}^\times$ we also denote by $m \in \Gal\left(\Qab / \mbb{Q}\right)$ the corresponding element, which can be described as the homomorphism which raises a root of unity to the power of $m$.
Similarly it acts on $G$ by taking $g$ to $g^m$.
Then, for every such $m$ and $g$ we have that
$
\chi_\rho\left(g^m\right)
= \tr \rho\left(g\right)
= \sum \lambda_i^m
= m. \left(\sum \lambda_i\right)
= m. \left(\chi_\rho\left(g\right)\right)
$.
We replace $g$ with $g^{m^{-1}}$ ($m$ is invertible), and rewrite this as $\chi_\rho\left(g\right) = m. \left(\chi_\rho\left(g^{m^{-1}}\right)\right)$.
Similarly to this equality, we can define an action of $\Gal\left(\Qab / \mbb{Q}\right)$ on $\cl{G; \Qab}$, by taking a class function $f$ to $m.f$ defined by $\left(m.f\right)\left(g\right) = m. \left(f\left(g^{m^{-1}}\right)\right)$.

Let us rewrite this action in another way, which will be helpful in the next section.
We note that $G \cong \hom_{\TopGrp}\left(\hat\ZZ, G\right)$ (continuous homomorphisms).
We get an action of $\aut\left(\hat\ZZ\right) \cong {\hat\ZZ}^\times$ on $G$ by pre-composition.
Concretely, $m \in {\hat\ZZ}^\times$ acts by sending $g \in G$ to $g^m$.
Since ${\hat\ZZ}^\times$ acts on $\Qab$, we get an action on $\cl{G; \Qab} = \hom_{\GSet}\left(G,  \Qab\right)$ by acting with $m^{-1}$ in the source and with $m$ in the target.
It is evident that this is the same action from the previous paragraph.

As we just saw, the characters are in the fixed points $\cl{G; \Qab}^{\Gal\left(\Qab / \mbb{Q}\right)}$.
Also, since the rationals are fixed by the action of the Galois group, rational linear combinations of characters are in the fixed points.
We therefore conclude that the character map after tensoring with $\mbb{Q}$ lands in the fixed points, i.e. $\chi \otimes \mbb{Q}: \RG \otimes \mbb{Q} \to \cl{G; \Qab}^{\Gal\left(\Qab / \mbb{Q}\right)}$.
Moreover, the second classical theorem is:

\begin{theorem}[{\cite[Theorem 25]{Ser}}]\label{char-2}
	The map $\chi \otimes \mbb{Q}: \RG \otimes \mbb{Q} \to \cl{G; \Qab}^{\Gal\left(\Qab / \mbb{Q}\right)}$ is an isomorphism.
\end{theorem}

To conclude, \ref{char-1} tells us that $\RG \otimes \Qab \cong \cl{G; \Qab}$, and \ref{char-2} tells us that $\RG \otimes \mbb{Q} \cong \cl{G; \Qab}^{\Gal\left(\Qab / \mbb{Q}\right)}$.
