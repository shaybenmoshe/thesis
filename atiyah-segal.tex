\section{Atiyah-Segal}

We now leave the realm of chromatic homotopy theory.
One aspect of algebraic topology is to try to capture properties of spaces using algebraic invariants.
One of the most fruitful such invariants is complex K-theory $\K$, and one of the most important spaces in homotopy theory is $\BG$, so it is natural to ask for a description of $\K\left(\BG\right)$ (by Bott periodicity, we will consider only $\K = \K^0$).
Atiyah and Segal \cite{AS} gave a description of this, and more, in the case that $G$ is a compact Lie group, in terms of representations.

From now we fix a compact Lie group $G$.
Also, a representation means a unitary representation.
We should also note that beyond this part, we will be mostly interested in finite groups.

\subsection{The Atiyah-Segal Theorem}

We denote by $\RG$ the \emph{representation ring} of $G$, that is the collection of virtual representations of $G$ (which can be written is as a formal difference $V - U$) where the addition is given by direct sum and the product is given by tensor product.
This is an augmented ring $\varepsilon: \RG \to \ZZ$ by the virtual dimension (i.e. $\varepsilon\left(V-U\right) = \dim\left(V-U\right) = \dim V - \dim U$).
The \emph{augmentation ideal} is $I = \ker \varepsilon = \left\{ V - U \in \RG \mid 0 = \dim(V-U) \right\}$.

\begin{theorem}[{\cite{AS}}]
	$\K\left(\BG\right) \cong \RGI$.
\end{theorem}

We will not prove the theorem, but we will indicate some of the key ingredients.

First of all, to show that objects are isomorphic, we need a map.
Before giving the map actually used in the proof, we describe an easier way to see where this map comes from.
Recall that $\K\left(X\right) \cong \left[X, \BU \times \ZZ\right]$.
We write $\BU \to \BU \times \ZZ$ for the map that is the identity on the first coordinate and $0$ on the second.
The data of a representation of $G$ is the same thing as a homomorphism $G \to \UU$.
Since $\BB$ is a functor, we get a map $\BG \to \BU$, and by composing with $\BU \to \BU \times \ZZ$, we indeed get a map $\BG \to \BU \times \ZZ$, that is, and element of $\K\left(\BG\right)$.
Therefore we get a map $\RG \to \K\left(\BG\right)$.
The theorem is that it is a ring homomorphism which exhibits $\K\left(\BG\right)$ as the completion of $\RG$ at $I$.

There is an alternative description of this map.
In \cite{Seg}, Segal described equivariant K-theory $\KG$.
This is a variant of K-theory, which assigns to a $G$-space the ring of virtual $G$-bundles, that is bundles equipped with an action of $G$, compatible with the action on the base $G$-space.
Note that this is no longer a homotopy invariant, since it also takes into account the action of $G$.
First we note the following:

\begin{proposition}
	$\KG\left(*\right) = \RG$ (where $*$ denotes the trivial $G$-space)
\end{proposition}

\begin{proof}
	This is by definition, since a vector bundle over a point is just a vector space, and it is equipped with a $G$ action.
\end{proof}

For any $G$-space $X$, the projection map $\mrm{pr}: X \to X/G$ allows us to pullback vector bundles on $X/G$, to $G$-bundles on $X$, that is it induces a map $\mrm{pr}^*: \K\left(X/G\right) \to \KG\left(X\right)$.

\begin{proposition}[{\cite[2.1]{Seg}}]
	If $G$ acts \emph{freely} on $X$, then there is an inverse to $\mrm{pr}^*$, so $\K\left(X/G\right) \cong \KG\left(X\right)$.
	The inverse is given by taking a bundle $E \to X$ to $E/G \to X/G$.
\end{proposition}

Now, we have a map of $G$-spaces given by $\EG \to *$.
By the above we get:
$$
\RG
\cong \KG\left(*\right)
\to \KG\left(\EG\right)
\cong \K\left(\EG/G\right)
= \K\left(\BG\right)
$$
This is again the map we need, which exhibits $\K\left(\BG\right)$ as the $I$-completion of $\RG$.
Atiyah and Segal use this map and variants to show the theorem.

The way Atiyah and Segal's proof works is as follows.
They use the Milnor join construction $\EG_n = \underbrace{G * \cdots * G}_{n \text{ times}}$ and $\BG_n = \EG_n/G$, which has the property that $\colim \EG_n \to \colim \BG_n$ is a model for $\EG \to \BG$.
Then, for any compact $G$-space $X$ there is a similar map to the map above, using $X \times \EG_n \to X$ we get a map $\KG\left(X\right) \to \KG\left(X \times \EG_n\right)$.
All of these are $\RG \cong \KG\left(*\right)$-modules, and they show that this map factors through the quotient $I^n$, to give a map $\KG\left(X\right)/I^n \to \KG\left(X \times \EG_n\right)$.
The two sides assemble into pro-rings, and the maps assemble to a map between the pro-rings.
What they actually prove is the strong form:

\begin{theorem}[{\cite{AS}}]
	If $\KG\left(X\right)$ is finite over $\RG$, then the above map of pro-rings is an isomorphism.
\end{theorem}

Their proof is special in a way, since it uses different groups to deduce the result for $G$.
In particular, to prove the result for example for a finite group, they have to deal with more general compact Lie groups.

Their proof works in 4 steps:
\begin{itemize}
	\item Prove for $G = \UU[1]$ (circle group),
	\item Prove for $G = \UU[1]^n$ (torus group),
	\item Prove for $G = \UU[n]$,
	\item Prove for general compact Lie group $G$ by embedding in $\UU[n]$.
\end{itemize}



\subsection{Recollections from Character Theory}

We restrict ourselves to the case of finite groups $G$.
We recall that representations of groups can be studied by their characters.
Specifically the trace map $\chi: \RG \to \ZZ\left[\chi_{\rho_i}\right]$, defined by $\chi_\rho = \tr \rho$, is an isomorphism, where the ring on the right is the ring of functions generated by the irreducible characters (the multiplication of two characters is a character so it is indeed closed under multiplication).

We also recall that characters are class functions, that is, they are constant on conjugacy classes.
Let $L$ be some field containing all the values of all characters.
Then a natural place to study characters is in the ring of class functions $\cl{G;L}$, the functions $G \to L$ which are constant.
Therefore we get that there is an injection $\chi: \RG \to \cl{G;L}$.
The first classical theorem regarding the relationship between characters and class functions is:

\begin{theorem}\label{char-1}
	After tensoring with $L$, the character map $\chi \otimes L: \RG \otimes L \to \cl{G; L}$ becomes an isomorphism.
\end{theorem}

\begin{proof}
	Similarly to the proof in \cite[9.1]{Ser} for $L = \mbb{C}$, we can view $\cl{G;L}$ as a vector space over $L$, and the characters are linearly independent, so by counting them we see that the image of $\chi \otimes L$ has the dimension of the whole vector space and we are done.
\end{proof}

By definition the value of a character is the trace of a linear transformation $\chi_\rho\left(g\right) = \tr \rho\left(g\right) = \sum \lambda_i$ where $\lambda_i$ are the eigenvalues (which exist since the representation is unitary).
Since $g^{\left|G\right|} = e$, we get $\rho\left(g^{\left|G\right|}\right) = \rho\left(e\right) = \id$, but then we get that the eigenvalues of $\rho\left(g^{\left|G\right|}\right)$ are on the one hand $\lambda_i^{\left|G\right|}$ and on the other hand they are all $1$.
Therefore $L = \mbb{Q}^\mrm{ab} = \mbb{Q}\left(\zeta_\infty\right)$ is always a valid choice for $L$.
To be concrete, we will take this choice.

The Galois group of $\mbb{Q}^\mrm{ab}$ is $\Gal\left(\Qab / \mbb{Q}\right) \cong {\hat\ZZ}^\times$.
For every $m \in {\hat\ZZ}^\times$ we also denote by $m \in \Gal\left(\Qab / \mbb{Q}\right)$ the corresponding element, which can be described as the homomorphism which raises a root of unity to the power of $m$.
Then, for every such $m$ we have that
$
\chi_\rho\left(g^m\right)
= \tr \rho\left(g\right)
= \sum \lambda_i^m
= m. \left(\sum \lambda_i\right)
= m. \left(\chi_\rho\left(g\right)\right)
$.
We replace $g$ with $g^{m^{-1}}$ ($m$ is invertible), and rewrite this as $\chi_\rho\left(g^m\right) = m. \left(\chi_\rho\left(g^{m^{-1}}\right)\right)$.
Similarly, to this we can define an action of $\Gal\left(\Qab / \mbb{Q}\right)$ on $\cl{G; \Qab}$, by taking a class function $f$ to $m.f$ defined by $\left(m.f\right)\left(g\right) = m. \left(f\left(g^{m^{-1}}\right)\right)$.
As we just saw, the characters are in the fixed points $\cl{G; \Qab}^{\Gal\left(\Qab / \mbb{Q}\right)}$.
Also, since the rationals are fixed by the action of the Galois group, rational linear combinations of characters are in the fixed points.
We therefore conclude that the character map after tensoring with $\mbb{Q}$ lands in the fixed points, i.e. $\chi \otimes \mbb{Q}: \RG \otimes \mbb{Q} \to \cl{G; \Qab}^{\Gal\left(\Qab / \mbb{Q}\right)}$.
Moreover, the second classical theorem is:

\begin{theorem}\label{char-2}
	The map $\chi \otimes \mbb{Q}: \RG \otimes \mbb{Q} \to \cl{G; \Qab}^{\Gal\left(\Qab / \mbb{Q}\right)}$ is an isomorphism.
\end{theorem}

The proof is essentially the same as in \cite[13.1, theorem 29a]{Ser}.

To conclude, \ref{char-1} tells us that $\RG \otimes \Qab \cong \cl{G; \Qab}$, and \ref{char-2} tells us that $\RG \otimes \mbb{Q} \cong \cl{G; \Qab}^{\Gal\left(\Qab / \mbb{Q}\right)}$.
